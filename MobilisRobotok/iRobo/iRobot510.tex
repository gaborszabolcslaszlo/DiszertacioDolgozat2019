\renewcommand{\xname}{iRobo510}
\renewcommand{\x}{0.521}
\renewcommand{\y}{0.686}
\renewcommand{\z}{0.178}
\renewcommand{\weight}{20}
\renewcommand{\img}{MobilisRobotok/iRobo/iRobo510.jpg}
\renewcommand{\sources}{Forrás: https://www.army-technology.com}
\renewcommand{\captionn}{iRobo 510 lánctalpas packboot}
\renewcommand{\watherProf}{Igen -3m ig.}
\renewcommand{\sebesseg}{9.3}
\renewcommand{\weight}{10.89}
\renewcommand{\AcAndGy}{Igen}
\renewcommand{\GPS}{Igen}
\subsection*{iRobo510}
 A robot külterén és belterén is egyaránt használható, felderítésekre és kisebb beavatkozásra alkalmas a manipulátor karát használva.A robotot 2007-ben dobták piacra. A katonaság használta bombák hatástalanítására vagy felderítésre. 

\renewcommand{\aspectratioPic}{0.5}
\begin{kep}
\begin{figure}[H]
\centering
\ifthenelse{\equal{\svg}{*}}
{
    \includegraphics[width=\aspectratioPic\textwidth,angle=\rotationAnglePic]{\img}
}
{
    \includesvg[width=\aspectratioPic\textwidth,angle=\rotationAnglePic]{\img}
}

 \ifthenelse{\equal{\sources}{*}}
    { \captionof{figure}{ \captionn}}
    { \captionof{figure}{ \mand{\mand{\captionn}{Forrás:}}{}} }
  	

\ifthenelse{\equal{\figlabel}{*}}
    {}
    {\label{fig:\figlabel}}%
    
\renewcommand{\figlabel}{*}



\end{figure}
\end{kep}
\renewcommand{\aspectratioPic}{1}
\renewcommand{\rotationAnglePic}{0}
\renewcommand{\svg}{*}


\begin{center}
\center
\begin{tabular}{ llll  }
 \hline
 Tulajdonság              &           &          & Mértékegység \\ \hline
\multirow{3}{*}{Méretek} & X         &    \x      & m  \\
                         & Y         &    \y      & m  \\
                         & Z         &    \z      & m  \\
Önsuly                   & \multicolumn{2}{l}{\weight} & kg \\
Max Sebesség             & \multicolumn{2}{l}{\sebesseg} & km/h \\
Gyorsulásmérő és Giroszkóp & \multicolumn{2}{l}{\AcAndGy} & 
\end{tabular}
\end{center}

