\renewcommand{\xname}{Husky}
\renewcommand{\x}{0.99}
\renewcommand{\y}{0.67}
\renewcommand{\z}{0.39}
\renewcommand{\weight}{50 + 75}
\renewcommand{\img}{MobilisRobotok/clearpathrobotics/husky.jpg}
\renewcommand{\sources}{Forrás: https://robots.ros.org/husky}
\renewcommand{\captionn}{Négykerekű mobilis platform.}
\renewcommand{\watherProf}{Igen}
\renewcommand{\sebesseg}{3.6}
\renewcommand{\AcAndGy}{Igen}
\renewcommand{\GPS}{Igen}

\subsection*{Husky}
A nagy teherbírású kültéri mobilis robotot leginkább kutatási célokra használják, nagy keréknyomatéka miatt nehéz terepen is jól boldogul. Nagy felbontású inkrementális szenzorokkal szerelték fel, alacsony mozgási sebességre is képes. Átlagos használat mellett három órát működik, támogatja teljes mértékben ROS-t.

\renewcommand{\aspectratioPic}{0.6}
\begin{kep}
\begin{figure}[H]
\centering
\ifthenelse{\equal{\svg}{*}}
{
    \includegraphics[width=\aspectratioPic\textwidth,angle=\rotationAnglePic]{\img}
}
{
    \includesvg[width=\aspectratioPic\textwidth,angle=\rotationAnglePic]{\img}
}

 \ifthenelse{\equal{\sources}{*}}
    { \captionof{figure}{ \captionn}}
    { \captionof{figure}{ \mand{\mand{\captionn}{Forrás:}}{}} }
  	

\ifthenelse{\equal{\figlabel}{*}}
    {}
    {\label{fig:\figlabel}}%
    
\renewcommand{\figlabel}{*}



\end{figure}
\end{kep}
\renewcommand{\aspectratioPic}{1}
\renewcommand{\rotationAnglePic}{0}
\renewcommand{\svg}{*}

\begin{center}
\center
\begin{tabular}{ llll  }
 \hline
 Tulajdonság              &           &          & Mértékegység \\ \hline
\multirow{3}{*}{Méretek} & X         &    \x      & m  \\
                         & Y         &    \y      & m  \\
                         & Z         &    \z      & m  \\
Önsuly                   & \multicolumn{2}{l}{\weight} & kg \\
Max Sebesség             & \multicolumn{2}{l}{\sebesseg} & km/h \\
Gyorsulásmérő és Giroszkóp & \multicolumn{2}{l}{\AcAndGy} & 
\end{tabular}
\end{center}

