\renewcommand{\xname}{Spirit}
\renewcommand{\x}{1.6}
\renewcommand{\y}{2.3}
\renewcommand{\z}{1.5}
\renewcommand{\weight}{35 (felszereléssel 180)}
\renewcommand{\img}{MobilisRobotok/Spirit/spirit.jpg}
\renewcommand{\sources}{Forrás:
https://en.wikipedia.org/wiki/Mars_Exploration_Rover}
\renewcommand{\captionn}{Spirit nevű Mars járó robot.}
\renewcommand{\watherProf}{Igen -3m ig.}
\renewcommand{\sebesseg}{0.05(avg 0.01)}
\renewcommand{\AcAndGy}{Igen}
\renewcommand{\GPS}{Igen}
\subsection*{Spirit}
 Marsi körülményekre tervezett robot 6 kerekkel rendelkezett, amelyből 4 kormányzott
 négy sztereó kamerapárral ellátva,30\degree lejtőt volt képes megmászni. 
Működési ideje 6 év és 2 hónap volt a Marson, a végen homokba süllyedve egy sziklára akadva
érte a marsi tél, amely ahhoz vezetett, hogy a nap elemei nem szolgáltattak elegendő
energiát és így végleg leállt és kihűlt.

\renewcommand{\aspectratioPic}{0.5}
\begin{kep}
\begin{figure}[H]
\centering
\ifthenelse{\equal{\svg}{*}}
{
    \includegraphics[width=\aspectratioPic\textwidth,angle=\rotationAnglePic]{\img}
}
{
    \includesvg[width=\aspectratioPic\textwidth,angle=\rotationAnglePic]{\img}
}

 \ifthenelse{\equal{\sources}{*}}
    { \captionof{figure}{ \captionn}}
    { \captionof{figure}{ \mand{\mand{\captionn}{Forrás:}}{}} }
  	

\ifthenelse{\equal{\figlabel}{*}}
    {}
    {\label{fig:\figlabel}}%
    
\renewcommand{\figlabel}{*}



\end{figure}
\end{kep}
\renewcommand{\aspectratioPic}{1}
\renewcommand{\rotationAnglePic}{0}
\renewcommand{\svg}{*}


\begin{center}
\center
\begin{tabular}{ llll  }
 \hline
 Tulajdonság              &           &          & Mértékegység \\ \hline
\multirow{3}{*}{Méretek} & X         &    \x      & m  \\
                         & Y         &    \y      & m  \\
                         & Z         &    \z      & m  \\
Önsuly                   & \multicolumn{2}{l}{\weight} & kg \\
Max Sebesség             & \multicolumn{2}{l}{\sebesseg} & km/h \\
Gyorsulásmérő és Giroszkóp & \multicolumn{2}{l}{\AcAndGy} & 
\end{tabular}
\end{center}

