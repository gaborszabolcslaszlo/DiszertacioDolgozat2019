%\lhead{ROS sajat robot}
\section{ROS}

A roboton levő számítógépen Ubuntu 16.4 linux operációs rendszer fut, és ROS lunar keret rendszerben fejlesztet programok megoldjak:
-FPGA ROS kommunikációt, -távirányítás, -heartbeat biztonsági funkcionalitás,-IMU szenzor adatainak a beategyüttese,-térképezés és lokalizálás.

Az \ref{fig:ROSgraph} láthatóak a főbb nodok, topikok és a köztük levő relációk, a nodok és a topikok leírását a 

\begin{table}[H]
\centering
\begin{tabular}{lp{3cm}p{8cm}}
\hline Node Nev & Tipus & Leiras \\ \hline
  /R es /L      &  FPGA
  Communication
  Modul     &  Jobboldali FPGA-val való kommunikációért felelős. Adatokat fogad/küld  a \ref{FPGAcomuSection} fejezetben leírt protokoll alapján, amelyeket továbbít a ROS operációs rendszerben működő nodoknak.     \\
  /ImuInBoxA       &    Imuxxxx   &  Feladata szöveges formában érkező adatok feldolgozása és a ROS keretrendszerbe integrálását oldja meg. Mert fizikai menyiségek:       \\
  /WheelOdometry       &       &   Kerekek mozgásából számolt robot elméleti pozíciója a térben     \\
  /TeleopJot      &       & Feladata a joystick-tol érkező parancsok fogadása. Megvalósítja a $Dead Man Switch$ gomb kezelését létrehoz egy globálisan engedélyező jelet a /setEMS-t és a /TeleopSig-t amely előírja a robot linearis mozgási sebességét és a forgási sebességét a Z tengely körül.  \\
  /rplidarNode & & A lidar merési adatait olvassa ki és továbbítja a /scan topikban.\\
  /rosserial\_server & & A megvalósítja a kommunikációt egy esp8266 fejlesztőlap és a ROS között amely az akkumulátorok feszültségeinek a merését végzi \cite{RosSerial}.\\
  /Joy & & Joystick eszközök integrációját valósítja meg \cite{rosjoy}.\\
  /hector\_mapping & & Lidar merései alapján 2D térképet keszit a környezetről, miközben lokalizálja a robotot ezen a térképen \cite{roshectormap}\\
  /Measurement & & merésék elvégzésére szolgáló node, amely egy élőre beállított intervallumokban az megadat referenciaértékeket küld ki az FPGA ban levő szabályzóknak.\\
  /TelepoController & & Átalakítja a robot sebesség állapotait és kiszámolja nyílt hurokban a szabályzok előírt értékeit. Abban az esetben ha a /movebase nodot használjuk, ez megoldja a robot pozíció szabályzását a térképen, így a $/cmd_vel$ csomagot csak átalakítja /refVals csomaggá azáltal hogy azonos oldalon levő kereke ugyanazt a referencia értéket kell követniük.
  
  
\end{tabular}
\end{table}

\subsection{Uzenet tipusok (.msg)}
Az alábbi táblázatban láthatjuk a ROS operációs rendszer által szolgáltatott .msg üzenetek kiterjesztését amelyek lehetővé teszik az FPGA integrációját a ROS környezethez.
\begin{table}[H]
\centering
\begin{tabular}{lllp{6cm}}
\hline \textbf{Üzenet típus}              & \textbf{Értékek}   & \textbf{Erkek típusa} & \textbf{Leírás} \\ \hline
 header                   &     -     &      std\_msgs/Header      &   Minden üzenet tartalmaz egy fejlécét amely információkat tartalmaz az üzenetről. \\   
                          & seq       &      uint32        &   minden üzenetet egyedileg beazonosító szám     \\
                          & stamp     &      time        &  idő-bélyeg amely a küldés idő-pillanatát tárolja.      \\
                          & frame\_id  &      string        &        \\  \hline
                          
\hline  \multirow{1}{*}{/GlobalEnable}  &   systemIsOk        &    int16          &    
                                                                          =0 - a HLC működik.
                                                                          
                                                                          <>0 -  a HLC nem nem működik      \\    \hline                    
\hline\multirow{4}{*}{/refVals} & names     & string{[}{]} &        \\
                          & ref\_position  & float{[}{]}  & előírt szög pozíció       \\
                          & ref\_velocity & float{[}{]}  &  előírt szög sebesség      \\ 
                          & ref\_effort &    float{[}{]}  &  előírt forgatónyomaték    \\\hline
 \hline \multirow{1}{*}{/setEMS}  &   value        &     int16         &      =0 - Vészleállító aktív.
                                                                          
                                                                       <>0 -  Vészleállító nem aktív  \\     \hline 
\hline\multirow{3}{*}{/joyControll} & vx     & float64 &    A robot X tengely mentén előírt sebessége m/s-ban.    \\
                          & omega  & float64  & előírt szög pozíció   A robot Z tengely körüli forgása \degree/s-ban.    \\
                          & ControlMode & int64  &  Választhatunk a HLC szabályzok típusa vagy a manuális irányítás közül 
                           =0 move\_base szabályzó, =1 manuális irányítás joystick segítségével.\\   \hline                                                                       
\end{tabular}
\end{table}



\renewcommand{\img}{SajatRobot/ROS/rosgraph.svg}
\renewcommand{\sources}{*}
\renewcommand{\svg}{svg}
\renewcommand{\aspectratioPic}{1.4}
\renewcommand{\rotationAnglePic}{90}
\renewcommand{\captionn}{ROS graph}
\renewcommand{\figlabel}{ROSgraph}
\input{picture.tex}

\subsection{FPGA komunikacios modul ROS oldali integracio}

A ROS biztosít a fejlesztőknek egy megoldást amelyek kepések újonnan létrehozót robot integrálását ROS környezetben \cite{RosSerial}. Előnye hogy gyorsan látványos eredményt érhetünk el, de a működés sebességben és az üzenetek méretében is korlátozott. Ezen hátrányokból kifolyólag sajátos integráció volt szükséges amely integrálta az FPGA UART kommunikáció protokollt a ROS keretrendszerben működő más modulokhoz.

A \ref{fig:ROStoUart} diagram a kommunikáció node technikai megvalósítását, különálló szál gondoskodik az UART adatok lóvasasáról és írásáról. Az üzenetek értelemizéset egy külön szál végzi és hívja fel a kiszolgáló fövenyeket.
A paraméterek helyes beállításáról a ParamThread szál gondoskodik, paraméterek helyes beállításáról FPGA oldalon. Abban az esetben ha a hardver kap egy uj paramétert a \ref{fig:MicroblazeSoft} alapján az FPGA visszaküldi a kapott paramétert, a visszajelzésből a eldönthető hogy a paraméter  a hardverben helyesen beállítódott-e. Abban az esetben ha nem megfelelő újraküldődik mindaddig amíg nem sikeres a beállítás.

A paraméterek kezelésére a ROS paraméter szerver a felelős \cite{parameterserver}, abban az esetben ha egy paraméter megváltozott amely az illető nodehoz köthető akkor a \ref{fig:ROStoUart} ábrán látható ParameterValtozott esemény előidézi a  megváltozott paraméter elküldését az FPGAnak.

A globális engedélyező jel a \ref{fig:ROStoUart} MasterLive Enable,  /globalEnable típusú üzenettel engedélyezhetjük a szabályzok működését, a folyamatos működéshez 500ms periódussal kell érkeznie. Abban az esetben ha a központi számítógép valami okból leállna akkor a hardveres szabályzok is leállnak. A node 300ms periódussal küldi tovább az engedélyező jelet az FPGA modulnak. A HardverLive jel információt szolgáltat a többi ROS környezetben futó és a működés szempontjából kritikus nodenak hogy az adott modul megfelelően működik e. Ezen információ birtokában a HeartBeat node leállítja a rendszert ha egyik FPGA modul nem válaszol.

\subsection{Előirt értékek}
A /refVals tipusu üzenetben megkapjuk minden egyes motor előírt érteket annak fövenyében $\degree/s$ ha sebesség alapján szabályzunk vagy $N/m$ előírt nyomaték alapján szabályzunk.

\renewcommand{\img}{SajatRobot/ROS/NodeUML.jpg}
\renewcommand{\sources}{*}
\renewcommand{\captionn}{ROS integrálása Uart protokolhoz.}
\renewcommand{\figlabel}{ROStoUart}
\input{picture.tex}

\subsection{Vonatkoztatási Rendszerek }
A  vonatkoztatási rendszereket szükségesek mert a szenzorok és beavatkozó eszközök pozíciója és egymáshoz viszonyított helyzete is változhat. Sok esetben szükséges ismernünk egy adott eszköznek a múltbeli helyzetét, vagy egy másik vonatkoztatási rendszerhez képest a pozícióját vagy irányát. A ROS biztosít egy tf \cite{rosTF} nevű csomagot amely megvalósítja a szűkséges eszközöket. 
A \ref{fig:ROSframes} látható a kialakított vonatkoztatási rendszerek a roboton amely hűen modellezi a fizikai robot kialakítását.
A vonatkoztatási rendszereket két csoportba oszthatok:

\begin{enumerate}[label=(\alph*)]
\item rögzített pozíció és szögek, szabadságok 0:
Szenzorok laser, BODY\_link, wheel\_odom, ImuALink VNR je a base\_link a globális robot VNR höz:
\item rögzített pozíció csak szögek változnak, szabadságok 1: Kerekek VNR je: FL\_link, BL\_link, FR\_link, BR\_link a BODY\_link hez képest csak Y korul foroghat.
\item pozíció és szög is változik, szabadságok 6:
A robot base\_link az helymeghatározás odom, és az odometria a térképhez map viszonyítva.
\end{enumerate}


A robot modellt ROS környezetben URDF robot leíró, xml alapú fájlal tehetjük meg, \cite{rosURDF}
\cite{rosJoint} \cite{rosLink}.
Az <origin> tag az xzy paraméter alatt megadhatjuk a csukló pozícióját mindhárom tengelyen méterben kifejezve a <parent> tagban szereplő linkhez képest. Az rpy paraméter alatt az elfordulásokat rendre x, y, z tengelyek mentén radiánban.
Az <axis> tagban beállíthatjuk a kényszereket, jelen esetben csak az y tengely körüli forgás engedélyezett. 
Az <link> tagban robot elemeket hozhatunk létre. Az alábbi XML-ben látható a robot fizikai leírása amely megfelel a valós szerkezetnek.
\begin{lstlisting}[language=XML]
<robot name="mobile_robot_platform_4Wheel">
	<link name="base_link" > </link>
	<link name="FL_link" > </link>
	<link name="BR_link" > </link>
	<link name="BL_link" > </link>
	<link name="BODY_link"> </link>  
	<link name="ImuALink"> </link>  
	<link name="laser"> </link> 

	<joint name="FL" type="continuous">
		<parent link="BODY_link"/>
		<child link="FL_link"/>    
		<origin xyz="0.29 -0.33 0" rpy="0 0 0" />
		<axis xyz="0 1 0" />
	</joint>

	<joint name="FR" type="continuous">
		<parent link="BODY_link"/>
		<child link="FR_link"/>    
		<origin xyz="0.29 0.330 0" rpy="0 0 0" />
		<axis xyz="0 1 0" />
	</joint>

	<joint name="BL" type="continuous">
		<parent link="BODY_link"/>
		<child link="BL_link"/>
		<origin xyz="-0.29 -0.330 0" rpy="0 0 0" />
		<axis xyz="0 1 0" />
	</joint>

	<joint name="BR" type="continuous">
		<parent link="BODY_link"/>
		<child link="BR_link"/>     
		<origin xyz="-0.29 0.330 0" rpy="0 0 0" />
		<axis xyz="0 1 0"/>
	</joint>

	<joint name="imuAandGPS" type="fixed">
		<parent link="base_link"/>
		<child link="ImuALink"/>
		<origin xyz="0.125 0.03 0.11" rpy="0 0 0" />
	</joint>

	<joint name="laserAJoin" type="fixed">
		<parent link="base_link"/>
		<child link="laser"/>
		<origin xyz="0.39 -0.02 0.23" rpy="0 0 3.14" />
	</joint>  

	<joint name="contact" type="fixed">
		<parent link="base_link"/>
		<child link="BODY_link"/>
	</joint> 
</robot>
\end{lstlisting}

\renewcommand{\img}{SajatRobot/ROS/frames.svg}
\renewcommand{\sources}{*}
\renewcommand{\svg}{svg}
\renewcommand{\aspectratioPic}{1.5}
\renewcommand{\rotationAnglePic}{90}
\renewcommand{\captionn}{A megvalositott robot VNR-k közti reláció }
\renewcommand{\figlabel}{ROSframes}
\input{picture.tex}


\section{Kerekek Pid Szabalyzo hangolas}

A pid a legelterjedtebb szabályozó egyszerű feladatok elvégzésére, esetünkben is elegendő a kerekek szögsebesség szabályzására kerekenkénti egy PID szabályzóval. A PID szoftveresen fut a uBlaze processzoron. Bemenete egy előírt forgási sebesség \degree/s ban és kimenete egy -32000 és 32000 egész típusú értek. A kimenti érteke a PWM kitöltési tényezőt jeleni, az előjel pedig a beavatkozás irányát.
Matlab/Simulink környezetben használva a Robotix Toolbox segítségével direkben pwm beavatkozo referenica erteket irtam elo a motroknak. A beavatkozó jel előállítása és elküldési a fizikai eszköznek 0-100\%-ig 10\% lépcsőkben amelyek 0\% kitoltesekkel vanak megszakitva. A mert adatokat rosbag csomagba mentve majd a System Identification Toolbox használatával identifikáljuk a rendszer modellt. A rendszer bemenete egy beavatkozójel ami fizikailag feszültségnek fele meg 0V és 12V között. A kimenetek a forgási sebesseg.
A mert adatokat Matlab/System Identification hasznalataval megbecsuljuk a rendszer modeleket. Nemlinearis modelt becslunk 
Hammerstein-Wiener model \cite{matlabhwmmodel} hasznalva, 1 kimenet es 1 bemenet, a linearis atviteli fugveny fokszama:
zerusok nb = 2, polusok nf = 3, keses a bemenet es a kimenet kozott nk = 1. A becsult adatok 94\% ban megfelelnek a mert rendszernek.          

A mereseket a robot kerekei es a talaj erintekzese nelkul vegeztem.
A becsult modelt a bemenet 50/\%  korul linearizaljuk es a linearizalt modelbol atviteli fugvenyt kesztunk. 
$tf = tf(linearize(model,16000));$ utasutast hasznalva Matlab kornyezetben. A linearizalt modelt Matlab/PidTuning eszkozt hasznalva behangolunk kiszamitjuk a megfelelo PID szabalyzo parametereit.


\tikzstyle{block} = [draw, fill=white, rectangle, 
    minimum height=3em, minimum width=6em]
\tikzstyle{sum} = [draw, fill=white, circle, node distance=1cm]
\tikzstyle{matlab} = [draw, fill=white, circle, node distance=1cm]
\tikzstyle{input} = [coordinate]
\tikzstyle{output} = [coordinate]
\tikzstyle{pinstyle} = [pin edge={to-,thin,black}]
\begin{center}
\begin{tikzpicture}[auto, node distance=2cm,>=latex']

   \node [block] (matlab) {Matlab/Simulink};
   \node [block, right of=matlab, node distance=4cm] (system) {System};
   \node [block, below of=system] (rosbag) {ROS bag};

    \node [right of=matlab,fill=black,inner sep=1.2pt,node distance=62] (umid) {};
    \draw [->] (matlab) -- (system);
%---------------- 
    \path (system.north east)--(system.south east) foreach \j in {1,...,2} {%
        coordinate [pos=1/3*\j] (z\j)
    };
    
    \foreach \i/\name/\nameNode/\text  in {1/{Omega}/{Om}/{$    \Omega$},2/{Current}/{Cu}/{I}}
    {  
        \node [output, right of=system, node distance=\i*15] (\nameNode) at (z\i) {};
        \node [fill=black,inner sep=1.2pt, right of=system, node distance=\i*15] (\nameNode) at (z\i) {};
        \node [output, right of=system, node distance=0] (\nameNode1) at (z\i) {};
        \node [output, right of=system, node distance=50] (\nameNode2) at (z\i) {};
        \draw [->] (\nameNode1) -- node[name=eu1,right of=Om2,node distance=33] {\text} (\nameNode2);
    }


%----------------
    \path (rosbag.north east)--(rosbag.south east) foreach \k in {1,...,2} {%
        coordinate [pos=1/3*\k] (z\k)
    };
    \foreach \i/\name  in {1/{inRbagOmeg},2/{inRbagCur}}
    {  
        \node [input, left of=rosbag, node distance=1.5] (\name) at (z\i) {};
    }
    

    \draw [<-] (inRbagCur) -| (Cu); crossing over
    \draw [<-] (inRbagOmeg) -| (Om);
    \draw [->] (umid) |-(rosbag);
    
\end{tikzpicture}
\end{center}

A becsült rendszer átviteli függvénye $H_s(z)$, mintavetelezesi periódus Ts: = 0.05s.

\subsection*{Nagyobik fokozatban}

A becsult modelt oszehasonlitva a mert ertekekkel a \ref{fig:NFsysIdent}, a model nemlinearis becsult model megfelel a mert ertekeknek.

\begin{figure}[H]
  \includegraphics{tikz/NFsysIdent.pdf}
  \caption{Nagy fokozat Hammerstein-Wiener becsult model valasza, es a mert ertekek.}
  \label{fig:NFsysIdent}
\end{figure}


Az atviteli fuggveny a bemenet 50/\% korul linearizalva.

\begin{equation}
    H_s(z)=\frac{-0.07017z^{-2} -0.053z^{-1}}{-0.2117^{-3}+0.7321z^{-2} -1.393z^{-1} +1}
\end{equation}

A tervezett PID szabályozó paramétere Kp: 7.11 , Ti: 23.66 , Td: 0.43

\subsection{Kisebik fokozatban}

A becsult modelt oszehasonlitva a mert ertekekkel a \ref{fig:KFsysIdent}, a model nemlinearis becsult model megfelel a mert ertekeknek.


\begin{figure}[H]
  \includegraphics{tikz/KFsysIdent.pdf}
  \caption{Kis fokozat Hammerstein-Wiener becsult model valasza, es a mert ertekek.}
  \label{fig:KFsysIdent}
\end{figure}

Az atviteli fuggveny a bemenet 50/\% korul linearizalva.

\begin{equation}
    H_s(z)=\frac{-0.0291z^{-2} -0.009263z^{-1}}{-0.198z^{-3}+0.7058z^{-2} -1.394z^{-1} +1}
\end{equation}

A tervezett PID szabályozó paraméterek: Kp: 15.96 , Ti:51.51 , Td:1.237 

.