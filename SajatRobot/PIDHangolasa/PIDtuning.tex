\section{Kerekek Pid Szabalyzo hangolas}

A pid a legelterjedtebb szabályozó egyszerű feladatok elvégzésére, esetünkben is elegendő a kerekek szögsebesség szabályzására kerekenkénti egy PID szabályzóval. A PID szoftveresen fut a uBlaze processzoron. Bemenete egy előírt forgási sebesség \degree/s ban és kimenete egy -32000 és 32000 egész típusú értek. A kimenti érteke a PWM kitöltési tényezőt jeleni, az előjel pedig a beavatkozás irányát.
Matlab/Simulink környezetben használva a Robotix Toolbox segítségével direkben pwm beavatkozo referenica erteket irtam elo a motroknak. A beavatkozó jel előállítása és elküldési a fizikai eszköznek 0-100\%-ig 10\% lépcsőkben amelyek 0\% kitoltesekkel vanak megszakitva. A mert adatokat rosbag csomagba mentve majd a System Identification Toolbox használatával identifikáljuk a rendszer modellt. A rendszer bemenete egy beavatkozójel ami fizikailag feszültségnek fele meg 0V és 12V között. A kimenetek a forgási sebesseg.
A mert adatokat Matlab/System Identification hasznalataval megbecsuljuk a rendszer modeleket. Nemlinearis modelt becslunk 
Hammerstein-Wiener model \cite{matlabhwmmodel} hasznalva, 1 kimenet es 1 bemenet, a linearis atviteli fugveny fokszama:
zerusok nb = 2, polusok nf = 3, keses a bemenet es a kimenet kozott nk = 1. A becsult adatok 94\% ban megfelelnek a mert rendszernek.          

A mereseket a robot kerekei es a talaj erintekzese nelkul vegeztem.
A becsult modelt a bemenet 50/\%  korul linearizaljuk es a linearizalt modelbol atviteli fugvenyt kesztunk. 
$tf = tf(linearize(model,16000));$ utasutast hasznalva Matlab kornyezetben. A linearizalt modelt Matlab/PidTuning eszkozt hasznalva behangolunk kiszamitjuk a megfelelo PID szabalyzo parametereit.


\tikzstyle{block} = [draw, fill=white, rectangle, 
    minimum height=3em, minimum width=6em]
\tikzstyle{sum} = [draw, fill=white, circle, node distance=1cm]
\tikzstyle{matlab} = [draw, fill=white, circle, node distance=1cm]
\tikzstyle{input} = [coordinate]
\tikzstyle{output} = [coordinate]
\tikzstyle{pinstyle} = [pin edge={to-,thin,black}]
\begin{center}
\begin{tikzpicture}[auto, node distance=2cm,>=latex']

   \node [block] (matlab) {Matlab/Simulink};
   \node [block, right of=matlab, node distance=4cm] (system) {System};
   \node [block, below of=system] (rosbag) {ROS bag};

    \node [right of=matlab,fill=black,inner sep=1.2pt,node distance=62] (umid) {};
    \draw [->] (matlab) -- (system);
%---------------- 
    \path (system.north east)--(system.south east) foreach \j in {1,...,2} {%
        coordinate [pos=1/3*\j] (z\j)
    };
    
    \foreach \i/\name/\nameNode/\text  in {1/{Omega}/{Om}/{$    \Omega$},2/{Current}/{Cu}/{I}}
    {  
        \node [output, right of=system, node distance=\i*15] (\nameNode) at (z\i) {};
        \node [fill=black,inner sep=1.2pt, right of=system, node distance=\i*15] (\nameNode) at (z\i) {};
        \node [output, right of=system, node distance=0] (\nameNode1) at (z\i) {};
        \node [output, right of=system, node distance=50] (\nameNode2) at (z\i) {};
        \draw [->] (\nameNode1) -- node[name=eu1,right of=Om2,node distance=33] {\text} (\nameNode2);
    }


%----------------
    \path (rosbag.north east)--(rosbag.south east) foreach \k in {1,...,2} {%
        coordinate [pos=1/3*\k] (z\k)
    };
    \foreach \i/\name  in {1/{inRbagOmeg},2/{inRbagCur}}
    {  
        \node [input, left of=rosbag, node distance=1.5] (\name) at (z\i) {};
    }
    

    \draw [<-] (inRbagCur) -| (Cu); crossing over
    \draw [<-] (inRbagOmeg) -| (Om);
    \draw [->] (umid) |-(rosbag);
    
\end{tikzpicture}
\end{center}

A becsült rendszer átviteli függvénye $H_s(z)$, mintavetelezesi periódus Ts: = 0.05s.

\subsection*{Nagyobik fokozatban}

A becsult modelt oszehasonlitva a mert ertekekkel a \ref{fig:NFsysIdent}, a model nemlinearis becsult model megfelel a mert ertekeknek.

\begin{figure}[H]
  \includegraphics{tikz/NFsysIdent.pdf}
  \caption{Nagy fokozat Hammerstein-Wiener becsult model valasza, es a mert ertekek.}
  \label{fig:NFsysIdent}
\end{figure}


Az atviteli fuggveny a bemenet 50/\% korul linearizalva.

\begin{equation}
    H_s(z)=\frac{-0.07017z^{-2} -0.053z^{-1}}{-0.2117^{-3}+0.7321z^{-2} -1.393z^{-1} +1}
\end{equation}

A tervezett PID szabályozó paramétere Kp: 7.11 , Ti: 23.66 , Td: 0.43

\subsection{Kisebik fokozatban}

A becsult modelt oszehasonlitva a mert ertekekkel a \ref{fig:KFsysIdent}, a model nemlinearis becsult model megfelel a mert ertekeknek.


\begin{figure}[H]
  \includegraphics{tikz/KFsysIdent.pdf}
  \caption{Kis fokozat Hammerstein-Wiener becsult model valasza, es a mert ertekek.}
  \label{fig:KFsysIdent}
\end{figure}

Az atviteli fuggveny a bemenet 50/\% korul linearizalva.

\begin{equation}
    H_s(z)=\frac{-0.0291z^{-2} -0.009263z^{-1}}{-0.198z^{-3}+0.7058z^{-2} -1.394z^{-1} +1}
\end{equation}

A tervezett PID szabályozó paraméterek: Kp: 15.96 , Ti:51.51 , Td:1.237 

.