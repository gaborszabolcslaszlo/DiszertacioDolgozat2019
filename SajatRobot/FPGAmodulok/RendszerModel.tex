%\lhead{Modulok es kapcsolatok}

\section{Alacsony szintu modulok} 

A roboton alacsonyszintű feladatait amelyek a motor hajtásokhoz kapcsolódó szenzorokat és vezérlő jelek elő allitasara hivatott a kétdaram CmodA7 20T FPGA fejlesztőlap. Négy tengely szogsebesseget kel szabályozni, egy FPGA két hajtást valósít meg. A \ref{fig:CmodA7architektura} alapján egy hajtáshoz tartozó I/O-k:

\begin{table}[H]
\center
\begin{tabular}{lll}
\hline Név                            & Darab              & Típus  \\ \hline
Inkrementális enkoder           & 2           & Digitális Input \\
Árammérő szenzor                & 1                  & Analóg Input \\
Motor vezérlő                   & 3                  & Digitális Output \\
Végálas kapcsoló                 & 1                  & Digitális Input        
\end{tabular}
\end{table}




\renewcommand{\img}{SajatRobot/SzerkAbrak/cmoda7modulok.jpg}
\renewcommand{\sources}{*}
\renewcommand{\captionn}{CmodA7 FPGA-ban kialakított architektúra amely a szenzorok és motor hajtások kezelésre hivatott }
\renewcommand{\figlabel}{CmodA7architektura}
\begin{kep}
\begin{figure}[H]
\centering
\ifthenelse{\equal{\svg}{*}}
{
    \includegraphics[width=\aspectratioPic\textwidth,angle=\rotationAnglePic]{\img}
}
{
    \includesvg[width=\aspectratioPic\textwidth,angle=\rotationAnglePic]{\img}
}

 \ifthenelse{\equal{\sources}{*}}
    { \captionof{figure}{ \captionn}}
    { \captionof{figure}{ \mand{\mand{\captionn}{Forrás:}}{}} }
  	

\ifthenelse{\equal{\figlabel}{*}}
    {}
    {\label{fig:\figlabel}}%
    
\renewcommand{\figlabel}{*}



\end{figure}
\end{kep}
\renewcommand{\aspectratioPic}{1}
\renewcommand{\rotationAnglePic}{0}
\renewcommand{\svg}{*}


A rendszer tervezesenel a fo koncepcio az volt hogy a rendszer arhitekturaja dinamikus legyen a fejleszhetoseget tekitve, igy a \ref{fig:CmodA7architektura} levo arhitekurat kaptuk. 


\renewcommand{\img}{SajatRobot/FPGAmodulok/uBlazeAndFpgaUML.jpg}
\renewcommand{\sources}{*}
\renewcommand{\captionn}{MicroBlaze proceszoron futo szoftver diagramja}
\renewcommand{\figlabel}{MicroblazeSoft}
\begin{kep}
\begin{figure}[H]
\centering
\ifthenelse{\equal{\svg}{*}}
{
    \includegraphics[width=\aspectratioPic\textwidth,angle=\rotationAnglePic]{\img}
}
{
    \includesvg[width=\aspectratioPic\textwidth,angle=\rotationAnglePic]{\img}
}

 \ifthenelse{\equal{\sources}{*}}
    { \captionof{figure}{ \captionn}}
    { \captionof{figure}{ \mand{\mand{\captionn}{Forrás:}}{}} }
  	

\ifthenelse{\equal{\figlabel}{*}}
    {}
    {\label{fig:\figlabel}}%
    
\renewcommand{\figlabel}{*}



\end{figure}
\end{kep}
\renewcommand{\aspectratioPic}{1}
\renewcommand{\rotationAnglePic}{0}
\renewcommand{\svg}{*}





\subsection{FPGA alapu uart komunikacio}
\label{FPGAcomuSection}
Megvalositva a komunikaciot a kiszolgalo ROS noddal, amely UART protokolra epitett sajat uzenetekbol all \ref{fig:FPGAComCsomagAlt}.

\renewcommand{\img}{SajatRobot/FPGAmodulok/ProtokolDiagram.jpg}
\renewcommand{\sources}{*}
\renewcommand{\captionn}{FPGA komunikacios protokol altalanos csomag szerkezet}
\renewcommand{\figlabel}{FPGAComCsomagAlt}
\begin{kep}
\begin{figure}[H]
\centering
\ifthenelse{\equal{\svg}{*}}
{
    \includegraphics[width=\aspectratioPic\textwidth,angle=\rotationAnglePic]{\img}
}
{
    \includesvg[width=\aspectratioPic\textwidth,angle=\rotationAnglePic]{\img}
}

 \ifthenelse{\equal{\sources}{*}}
    { \captionof{figure}{ \captionn}}
    { \captionof{figure}{ \mand{\mand{\captionn}{Forrás:}}{}} }
  	

\ifthenelse{\equal{\figlabel}{*}}
    {}
    {\label{fig:\figlabel}}%
    
\renewcommand{\figlabel}{*}



\end{figure}
\end{kep}
\renewcommand{\aspectratioPic}{1}
\renewcommand{\rotationAnglePic}{0}
\renewcommand{\svg}{*}



A modul megvalositja mindet iranyban a keretezest, parameterkent atadhato egy-egy karaketr jelzi az uzenet kezdetet es veget jelen esetben az $S$ csomag kezdetet, mig a $P$ a csomag veget jelzik.
A proceszor cimtartomanyaba ilesztet modul rendelkezik egy 400byte karakterbol allo memoriaval, amelyhez egyidoban a proceszor es a hardver is hozaferhet \cite{DualPortRam}.
A kezdo keret beerkezte utan minden karakter bekerul a 400byte hoszu memoriaba sorba, kivetelt kepez ezalol a binaris adat tomb ahol a skip\footnote{specialis karakterek amelyek jelzik az ertelmezo szamara hogy olyan karakter kovetkezik amelyet a protokol ertelmezeseben nem kell vegrehajtani.} karakterek kikerulnek a tombol.Jelen esetben a \{ es \} kozt levo karakterek nem kerulnek ertelmezesre. A \{ utani elso karakter megadja a binaris adat hoszat igy tudja az ertelmezo hogy hol kell varnia a lezaro karaktert.

\subsection{Paramterek FPGA modul}
% Please add the following required packages to your document preamble:
% \usepackage{multirow}
% \usepackage[normalem]{ulem}
% \useunder{\uline}{\ul}{}

\begin{table}[H]
\begin{tabular}{lllllp{6cm}}
\hline \multirow{2}{*}{Id} & \multirow{2}{*}{\begin{tabular}[c]{@{}l@{}}Nev \\ (X lehet A vagy B)\end{tabular}} & \multicolumn{2}{l}{Ertekek} & \multirow{2}{*}{Tipus} & \multirow{2}{*}{Leiras}                                                                                                                      \\
                    &                                                                                    & Min         & Max           &                        &                                                                                                                                              \\ \hline
1                   & TsTimerPeriod                                                                      & 1           & 1000          & int16                  &  Mintavetelezesi periodus {[}ms{]} ban.                                                                                                      \\
2                   & GetDataPeriodical                                                                  & 0           & 1             & int16                  &  Kapcsolo ha 0 akkor nem kuld az FPGA mert ertekeket, kulomben a TsTimerPeriod periodusi mintavetellel kuld.                                  \\
3                   & TorqueCoefX                                                                        &             &               & float16                &  Motor aram es nyomatek kozti egyuthato.                                                                                                      \\
4                   & ActiveControllerX                                                                  & 0           & 65535         & int16                  &  Valaszhato szabalyzo tipusok hajtasonkent 0=Szoftvare PID szogsebesseg, 1=Hardver PID szogsebesseg, 2=Szoftver PID aram, 3=Hardver PID aram  \\
5                   & MaxControlSiggnalX                                                                 & 0           & 32760         & sint16                 &  A beavatkozo PWM jel maximalis kitoltesi tenyezoje, linearisan 0-\textgreater{}0\% -tol 32760-\textgreater{}100\% -ig.                       \\
6                   & IncSenzResX                                                                        & 0           & 65535         & int16                  &  Inkrementalis szenzor altal generalt inpulzusok szama egy teljes kerekfordulatra.FPGA oldalon ez a szam 10-el szorzodik.                     \\
7                   & IncSenzCountDirectionX                                                             & -1          & 1             & sint16                 &  inkrementalis szenzor jeleit feldogozo mudul szamolasi iranyanak valtoztatasara szolgalo egyuthato.                                          \\
8                   & Kp\_Whell\_PidX                                                                    & 0           &               & float16                &  szogsebeseg szabalyzo, PID erositesi parametere.                                                                                             \\
9                   & Ti\_Whell\_PidX                                                                    &             &               & float16                &  szogsebeseg szabalyzo, PID integralasi ido.                                                                                                  \\
10                  & Td\_Whell\_PidX                                                                    &             &               & float16                &  szogsebeseg szabalyzo, PID derivalasi ido.                                                                                                   
\end{tabular}
\end{table}


\renewcommand{\img}{SajatRobot/FPGAmodulok/UartUML.jpg}
\renewcommand{\sources}{*}
\renewcommand{\captionn}{FPGA hardver/MicroBlaze proceszor es ROS node kozti komunikacio megvalistiasa UART protokol alapjan }
\renewcommand{\figlabel}{FPGAuartRos}
\begin{kep}
\begin{figure}[H]
\centering
\ifthenelse{\equal{\svg}{*}}
{
    \includegraphics[width=\aspectratioPic\textwidth,angle=\rotationAnglePic]{\img}
}
{
    \includesvg[width=\aspectratioPic\textwidth,angle=\rotationAnglePic]{\img}
}

 \ifthenelse{\equal{\sources}{*}}
    { \captionof{figure}{ \captionn}}
    { \captionof{figure}{ \mand{\mand{\captionn}{Forrás:}}{}} }
  	

\ifthenelse{\equal{\figlabel}{*}}
    {}
    {\label{fig:\figlabel}}%
    
\renewcommand{\figlabel}{*}



\end{figure}
\end{kep}
\renewcommand{\aspectratioPic}{1}
\renewcommand{\rotationAnglePic}{0}
\renewcommand{\svg}{*}



Az uzenet veget jelzo karakter az \ref{fig:FPGAuartRos} lathato folyamat alapjan a modul general egy hardveres megszakitast a procesornak, a proceszor kiolvassa az adat kezdocimet, es elkezdi kiolvasni az adatokat a sajat adatmemoriajaba. Ekozben ha uj csomag erkezik a hardver komunikacios modulhoz, elkezdodik annak beirasa a memoriaba, az elozo adatok felulnemirasaval. A megszakitas ertelemzi a csomag tipusat es annak megfeleloen vegrehajtja a muveleteket amelyek lehetnek: Parameter ertekenek a bealitasa,  Parameter ertekenek a lekerdezese, eloirt ertekek fogadasa, mert erteke kuldese, ACK megerosito uzenet kuldese/fogadasa.

Az uzenetek kuldese hasonlokepen mukodik az olvasashoz, a hardver megoldja a keretezest es a skip karakterek beszurasat is.




\subsubsection{Kommunikáció sebessége}

Az uart sebesege 1MBd \footnote{megabaud} ami megfelel 131072 byte/s adatforgalomnak. A valosagban a komunikacio hibatlanul mukodik 1ms periodussal kuldott 100byte hoszu uzeneteket.

\begin{equation}
    frekvencia = \frac{131072}{SizeOfPachage}=\frac{131072[byte/s]}{100[byte]}=1310,072 [Hz]
\end{equation}

\subsubsection{Biztonsagi megoldasok}

Abbana az esetben ha kikultuk az elirt ertekeket a modulnak es ezutan a komunikacio megszakadt a modullal akkor a szabalyzok probaljak tartani az elirt erteket annak elenere is hogy az mar lehet hogy megkelet volna valtozzon. Erre a celra bepitesre kerult egy uzenet es egy logika (HeartBeat) amely csak a komunikacios modulhoz erkezik meg, es jelzi hogy a ROS jolmukodik, es forditva is.Abban az esetben ha a komunikacio megszakad lealitja a motrokat es a szabalyzokat,  \ref{fig:FPGAuartRos} alapjan az Enable(EN) jel.

\begin{table}[H]
\center
\begin{tabular}{lll}
\hline Irany   & Uzenet & Periodis    \\ \hline
FPGA->ROS &  SEP        & 300 ms kotelezo         \\
ROS->FPGA &  mintavetelezett ertekek kuldese & dinamikusan modosithato                   
\end{tabular}
\end{table}















