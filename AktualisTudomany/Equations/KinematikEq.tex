\begin{equation}
\label{eq:allapot}
\dot{q} = 
\begin{bmatrix}
\dot{X}\\ 
\dot{Y}\\
\dot{\theta}
\end{bmatrix}= \begin{bmatrix}
\cos \theta  & -\sin \theta & 0 \\
\sin\theta & \cos \theta &  0\\ 
 0 & 0  & 1
\end{bmatrix}
\begin{bmatrix}
^RV^{COM}_x\\ 
^RV^{COM}_y\\ 
^R\omega^{COG}
\end{bmatrix}
\end{equation}

\begin{equation}
\label{eq:SebComY}
^RV^{COM}_y = d\omega
\end{equation}

\begin{equation}
\label{eq:nemholomonikusmegkotes}
\begin{bmatrix}
-\sin\theta  & -\cos\theta  & -d
\end{bmatrix}
\begin{bmatrix}
\dot{X}\\ 
\dot{Y}\\
\dot{\theta}
\end{bmatrix} = A(q) \dot{q} =0
\end{equation}

\begin{equation}
    \label{eq:allapotegyszeru}
    \dot q = S(q)\eta 
\end{equation}

\begin{equation*}
S(q)=\begin{bmatrix}
\cos \theta  & -d\sin \theta  \\
\sin\theta & d\cos \theta \\ 
 0  & 1
\end{bmatrix},
\eta=\begin{bmatrix}
^RV^{COM}_x\\ 
^R\omega^{COG}
\end{bmatrix}
\end{equation*}

\begin{equation}
\label{eq:SxAeqZero}
    S^TA^T=0
\end{equation}

Az ICR pont meghatározása a robot paraméterei és a mért sebességek alapján \cite{ICRposition}.

\begin{equation}
	^RV_{COG_X} = \frac{V_L + V_R}{2}
\end{equation}

\begin{equation}
	^R\Omega_{COG} = \frac{V_L - V_R}{b+a} 
\end{equation}

\begin{equation}
	^R{ICR_X} = -\frac{^R{V_{COG_Y}}}{^R\Omega_{COG}}
\end{equation}

\begin{equation}
	^R{ICR_Y} = \frac{^R{V_{COG_X}}}{^R\Omega_{COG}}
\end{equation}

\begin{equation}
	R=\sqrt{^R{ICR^2_X} + ^R{ICR^2_Y}}
\end{equation}


