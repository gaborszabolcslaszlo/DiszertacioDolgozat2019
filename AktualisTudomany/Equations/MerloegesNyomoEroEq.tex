
\iffalse
\begin{equation}
    AN=B \Rightarrow N=BA^{-1}
\end{equation}

\begin{equation}
G_xh=F_{difX}R_5\Rightarrow F_{difX}=\frac{G_{x}h}{R5}, \quad
G_yh=F_{difY}R_6\Rightarrow F_{difY}=\frac{G_{y}h}{R6}
\end{equation}

\begin{equation}
N_{comp} = \begin{bmatrix}
-\frac{1}{2}\\ 
\frac{1}{2}\\ 
-\frac{1}{2}\\ 
\frac{1}{2}
\end{bmatrix}F_{difX}
+
\begin{bmatrix}
-\frac{1}{2}\\ 
\frac{1}{2}\\ 
-\frac{1}{2}\\ 
\frac{1}{2}
\end{bmatrix}F_{difY}
\end{equation}

\begin{equation}
    N=BA^{-1} + N_{conp}(G_x,G_y)
\end{equation}

\begin{equation*}
A =\begin{pmatrix}
0 & \frac{R_5}{R_1} & \frac{R_7}{R_1} & \frac{R_6}{R_1}\\ 
\frac{R_5}{R_2} &  0&  \frac{R_6}{R_2}& \frac{R_7}{R_2} \\ 
\frac{R_7}{R_3} & \frac{R_6}{R_3} & 0 & \frac{R_5}{R_3} \\ 
\frac{R_6}{R_4} & \frac{R_7}{R_4} & \frac{R_5}{R_4}& 0
\end{pmatrix},\quad
N =\begin{bmatrix}
N_{FL} & N_{BL} & N_{FR} & N_{BR} 
\end{bmatrix}^T,
B=\begin{bmatrix}
G_z&& 
G_z&& 
G_z&& 
G_z
\end{bmatrix}^T
\end{equation*}

\begin{equation*}
    G=mg \textbf{ ahol m - a robot súlya.}
\end{equation*}

\begin{equation*}
R_1 = \sqrt{a^2+d^2},R_2 = \sqrt{a^2+c^2},\quad
R_3 = \sqrt{b^2+c^2}, R_4 = \sqrt{d^2+b^2}
\end{equation*}
\begin{equation*}
R_5 = d+c, R_6 = a+b,\quad
R_7 = \sqrt{(a+b^2)+(d+c)^2)} 
\end{equation*} 

\fi

Egy test nyugalomban van ha a rá ható erők eredője és a forgatónyomatékok eredője zero, ismerve a súlypont pozíciójának a koordinátáit a robot $VNR$-be akkor az \ref{eq:N1} egyenlettel meghatározzuk a $G_x$ erő által létrehozót nyomóerőket a $N_F$ és $N_B$ pontokban.

\begin{equation}
\label{eq:N1}
    N_{FGx}=\frac{hG_x}{c+d}
    ,\quad
    N_{BGx}= - N_{FGx}
\end{equation}

Meghatározzuk a $G_y$ erő által létrehozott nyomóerőket a $N_{RGy}$ és $N_{LGy}$ pontokban.

\begin{equation}
\label{eq:N2}
    N_{RGy}=\frac{hG_y}{a+b}
    ,\quad
    N_{LGy}= - N_{FGy}
\end{equation}


Ismerve a $N_{LGy}$ és $N_{RGy}$ pontokban ható erőket kiszámítjuk ezek eloszlását a robot kerekeire nézve, így megkapjuk azokat a nyomóerőket amelyet a \ref{fig:SMR4WLejtoSzembol} ábrán látható állapotban a $G_y$ gravitációból származó erő hoz létre.

\begin{equation}
\label{eq:N3}
N_{yBL}=\frac{dN_{LGy}}{c+d}
    ,\quad
F_{yFL}=-N_{yBL}
\end{equation}

\begin{equation}
\label{eq:N4}
N_{yBR}=\frac{dN_{RGy}}{c+d}
    ,\quad
F_{yFR}=-N_{yBR}
\end{equation}


Meghatározzuk a gravitáció Z komponense által létrehozót nyomóerőket a $F_F$ és a $F_B$ pontokban amelyhez hozzáadjuk a X komponens által létrehozót nyomóerőket ugyan ezekben a pontokban.

\begin{equation}
\label{eq:N5}
F_F = \frac{G_zd}{c+d} + N_{FGx}
    ,\quad
F_B = G_z-F_F + N_{BGx}
\end{equation}


Ismét kiszámoljuk a kerekekre nézve a nyomóerőket ismerve az $F_F$ és $F_B$ erőket.

\begin{equation}
\label{eq:N6}
F_{BR}=\frac{aF_B}{a+b}
    ,\quad
F_{BL}=F_{B}-F_{BR}
\end{equation}

\begin{equation}
\label{eq:N7}
F_{FR}=\frac{aF_F}{a+b}
    ,\quad
F_{FL}=F_{F}-F_{FR}
\end{equation}

A merőleges nyomóerő vektor az X,Y,Z gravitációs erők által létrehozott nyomóerők összegzéséből áll.


\begin{equation}
\label{eq:N8}
N_\perp =\begin{bmatrix}
F_{FL} + N_{yFL} & F_{BL} +  N_{yBL} & F_{FR} +  N_{yFR} & F_{BR} +  N_{yBR}
\end{bmatrix}^T
\end{equation}

\subsection{Súlypont (X,Y) komponensének a meghatározása}

Ismerve a robot méreteit $W$ jelölje a szélességét  míg a $L$ hosszúságát, kerék középpont között mérve.

A robotot vízszintes helyzetbe helyezzük, és minden kerek merőleges nyomóerejét megmérve mérleg segítségével megkapjuk a
$N_{FL},N_{FR},N_{BL},N_{BR}$ nyomóerőket.

\begin{equation}
\label{eq:N9}
W = a+b,\quad L = c+d
\end{equation}

Meghatározzuk a $a$ érteket ismerve a $F_{FR}$ pontban a nyomóerőt és kiszámolva a $F_F$ pontban a nyomóerőt a \ref{eq:N11} és \ref{eq:N12} egyenletek segítségével.

\begin{equation}
\label{eq:N10}
F_{FR}=\frac{aF_F}{W} \Rightarrow a = \frac{F_{FR}W}{F_F}
\end{equation}

\begin{equation}
\label{eq:N11}
G_{z}=F_{FR}+ F_{FL}+ F_{BR}+ F_{BL} 
\end{equation}

\begin{equation}
\label{eq:N12}
F_{F}=\frac{dG_z}{L} \Rightarrow d = \frac{F_{F}L}{G_z}
\end{equation}









