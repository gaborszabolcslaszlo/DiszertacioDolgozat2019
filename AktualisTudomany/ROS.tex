
%\lhead{Robot Operációs Rendszer}

\section{Robot Operációs Rendszer} 
A ROS 2007-ben jelent meg, gyorsan elterjedt az egész világon, manapság szinte minden robotokkal foglalkozó cég termékeit kapcsolódnak a ROS-hoz.

A \cite{lentin2015} \cite{NagyPirosKonyv} segítségével megalapozhatjuk a tudásunkat. Számos előnyel rendelkezik a ROS használata egy új robot fejlesztésében mert már elkészített csomagokat használhatunk pl: SLAM \footnote{ Simultaneous Localization and Mapping egyidejű térképezés és lokalizáció}, $AMCL$, vagy előre elkészített eszközök segítenek a fejlesztésben pl: Rviz  \footnote{ROS környezet vizualizációs eszköze}, Gazebo \footnote{ROS környezet szimulációs eszköze}, interfészt biztosít a szenzoroknak és beavatkozóknak pl: $LIDAR$, $IMU$ .

A \cite{lentin2015} említést tesz arról hogy más hasonló keret-rendszerekhez képest a ROS stabilabb pl: ha egy modulban futás-idejű hiba lép fel az nem terjed ki a többi csomópontra.
Egyszerűbb fejlesztést lehetséges azáltal hogy kisebb modulokat fejlesztünk és nem egy nagy több szálon futó kódot. Annak ellenére hogy a forrás-kódja nyílt a keret-rendszernek nagyon jó szupportja van, rengeteg fórumon keresztül kaphatunk megoldásokat az esetleges hibákra. Több teknologiát képes összekapcsolni mint pl.: tensorflow, matlab, simulink, opencv..., V-Rep

Hátrányai között említhető a Gazebo szimulátor mert a használata nem egyszerű ellentétben a V-Rep \footnote{Robot szimulációs szoftver amely támogatja a ROS-t} programmal. A robot modellezése nem egyszerű dolog, $URDF$ fájl szükséges hozzá, csak SolodWork \footnote{3D modellező szoftver} biztosít  lehetőséget arra hogy modellt exportálhassuk.

\subsubsection{Uj robot integrációja ROS hoz}
Egy új robot integrációja során meg kell vizsgálni hogy milyen mérési adatok állnak rendelkezésünkre alacsony szinten, a rotációs csukló  paraméterek lehetnek pl: szög pozíció, szög sebesség, kifejtett, nyomaték, ezeket a paramétereket mérhetjük, illetve referenciaként is előírhatjuk. 

Az integrációra több megoldás is lehetséges:
\begin{enumerate}[label=(\alph*)]
	\item ROS Serialon keresztül.
	\item ROS control használata.
	\item Osztott Rendszer.
\end{enumerate}

\subsubsection*{Ros Serial}
Egy megoldás a hardver integrációjára a ROS Serial amely UART, vagy TCP protokollra épulő, soros vagy hálózati kommunikációt használva. Korlátai miatt \cite{RosSerial} nem képes nagy méretű üzentek használatára, valamit a nodok száma is korlátozott lásd. Szükséges a ROS csomagok használata a hardveren ami nem mindig előnyös.

A \cite{ROSArduino2013} cikkben egy arduino típusú fejlesztő lappal valósítja meg a robot alacsony szintű szabályzását, megemlíti hogy a rosserial-t nem tudja használni a limitációk miatt. A feldolgozó oldalon elkészít egy saját szoftveres modult amely lényegében ROS kapcsolt, és ezen a modulon keresztül valósítja meg a kapcsolatot a hardverrel. A kommunikációra soros $UART$ protokollt használ.

A \cite{ROSRoboticsByExampleSecEd} könyv 8-ik fejezetben leírja hogyan lehet használni a rosserial-t, arduino valamint Raspberry Pi fejlesztő lapokon de viszont nem tesz említést a hátrányairól.

\subsubsection*{Ros Control}

A ros controller  \cite{roscontrol} használatával összeköthetjük az alacsony szintű hardvert a ROS keretrendszerben fejlesztett modulokkal, implementálva \cite{ROSControlExample} a hardware\_interface::RobotHW interfészt és létrehozva minden egyes rotációs csuklónak egy \newline hardware\_interface::JointStateHandle-t. A \ref{fig:RosControlDiagram} látható read() és write() függvényeken keresztül kell megvalósítani a kommunikációt a hardverrel, ez történhet hálózaton vagy soros porton keresztül.

\renewcommand{\img}{AktualisTudomany/roscontrol.png}
\renewcommand{\sources}{Forrás: https://www.army-technology.com}
\renewcommand{\captionn}{Ros Control modulok}
\renewcommand{\figlabel}{RosControlDiagram}
\begin{kep}
\begin{figure}[H]
\centering
\ifthenelse{\equal{\svg}{*}}
{
    \includegraphics[width=\aspectratioPic\textwidth,angle=\rotationAnglePic]{\img}
}
{
    \includesvg[width=\aspectratioPic\textwidth,angle=\rotationAnglePic]{\img}
}

 \ifthenelse{\equal{\sources}{*}}
    { \captionof{figure}{ \captionn}}
    { \captionof{figure}{ \mand{\mand{\captionn}{Forrás:}}{}} }
  	

\ifthenelse{\equal{\figlabel}{*}}
    {}
    {\label{fig:\figlabel}}%
    
\renewcommand{\figlabel}{*}



\end{figure}
\end{kep}
\renewcommand{\aspectratioPic}{1}
\renewcommand{\rotationAnglePic}{0}
\renewcommand{\svg}{*}


\subsubsection*{Osztott Rendszer}

A  \cite{DistributedRealTimeControlROS} cikkben leír egy megoldást arra hogyan lehetne smart eszközként tekinteni a szenzorokra és beavatkozokra. Osztott rendszert $DSC$ -t alkalmaz, ahol minden szenzornak saját mastere van, ezáltal minden node független lesz a hálózaton. A hálózat konfigurációját teljes mértékben ismerni kell minden nodenak a IP címét, de ezzel a megoldással futásidőben módosíthatjuk a hálózatot, $GUID$ -t használ a új maszter bejelentésre a hálózaton, valamit ezzel is oldja meg az információk áramlását is.
A kommunikációt a mesterek között ROS MultiPeer Architecture (RMPA) nevezet architektúrával oldja meg. Más rendszerekhez képest kétszer jobb időkésést valósított meg. 
Ami a hátránya, a szenzorok mellett egy mcu is szükséges amely kepés egy operációs rendszert futtatni és egy rós masztert. Valóban robusztusabb modulárisabb lesz a rendszer ezáltal de drágább is. Kisebb rendszereknél mint inkább hátrány mint elöny, de viszont komplex nagy területet lefedő rendszernél előnyös.

Egy másik hasonló megoldás ahol több ROS masztert lehet összekötni és felügyelni a multimaster\_fkie megoldja a futásidőben való új maszter szinkronizációját, a topikok és a szervizek kezelését is. 

\input{Meresek/Figure1.tex}


