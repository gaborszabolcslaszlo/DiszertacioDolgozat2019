
%\lhead{Mobilis robotok modellezése}

\section{Mobilis robotok modellezése} 

\subsection*{Négy kerekű modell}
A mobilis robotok leginkább kerekekkel oldjak meg a helyváltoztatásukat. Egy fontos probléma ezekkel a robotokkal az hogy milyen kölcsönhatások lepnek fel a kerék és a talaj között \cite{Torjancki4Mobilerobot} \cite{RobustMotionControl} \cite{Campa2014}, és ezeket az erőket hogyan lehet modellezni. A \cite{Torjancki4Mobilerobot} cikkben kidolgoz egy modellt amely segítségével képes meghatározni egy négykerekű robot pozícióját a kerekek forgási sebességéét felhasználva. A fent említett irodalmakban a $SSMR$  típusú $OMP$ vizsgálnak módélezés és pályakövetés szempontjából. Annak függvényében hogy a MR-t mozgató motorokat tekintve a következő variánsok lehetségesek: \textbf{4 kerék - 4 motor}, \textbf{4 kerék - 2 motor}, azonos oldalon levő kerekek összecsatolva fogas-szíjjal ez második megoldás egyszerűbb kevesebb szenzort és hajtó motort igényel.
A \cite{Torjancki4Mobilerobot} \cite{Campa2014} irodalmakban a $HLC$ sebesség referencia jelet ír elő a kerekeken, \cite{RobustMotionControl} a cikkben nyomatékot ír elő amelyet az alacsony szintű szabályozóknak.
Az $ICR$ meghatározásával választotta. Több dolgot is feltételez: a robot forgásközpontja a robot középpontjában van, a robot azonos oldalán levő kerekek ugyanazzal a szögsebességgel forognak, a négy kerék mindig érintkezik a talajjal és méretben is megegyeznek.

A \cite{RobustMotionControl} kifejezetten a SSMR  jól ismert a robusztus félépítése miatt, nagyon jól alkalmazható terepen. Általában $DDV$ mert a fordulást azáltal oldják meg hogy a jobb és bal oldali kerekek különböző sebességgel vagy különböző kifejtett nyomatékot fejtenek ki a talajra.
A \cite{RobustMotionControl} cikkben alkalmazott technológiáiak: $VFO$
$CTM$ a robot kerekeit szabályzó motoroknak nyomaték van előríva amit követniük kell.

\subsection{Merőleges Nyomóerő (N)} 

A kerekek és a talaj egy pontban érintkeznek, ezeket a pontokat jelölje a BR,BL,FL,FR a \ref{fig:SMR4WMerolegesNyomoero} ábrán. Jelölje S a robot súlypontkát, G - a súlyából származó erők a robot alapjához rendelt kordinát rendszerben felbontva a három tengely menten, N a merőleges nyomóerő a talajra az adott pontban.

\renewcommand{\img}{SajatRobot/SzerkAbrak/MerolegesNyomoEro.jpg}
\renewcommand{\sources}{*}
\renewcommand{\captionn}{Merőleges nyomóerő a talajra $4W-SSMR$ típusú robot esetében.}
\renewcommand{\figlabel}{SMR4WMerolegesNyomoero}
\begin{kep}
\begin{figure}[H]
\centering
\ifthenelse{\equal{\svg}{*}}
{
    \includegraphics[width=\aspectratioPic\textwidth,angle=\rotationAnglePic]{\img}
}
{
    \includesvg[width=\aspectratioPic\textwidth,angle=\rotationAnglePic]{\img}
}

 \ifthenelse{\equal{\sources}{*}}
    { \captionof{figure}{ \captionn}}
    { \captionof{figure}{ \mand{\mand{\captionn}{Forrás:}}{}} }
  	

\ifthenelse{\equal{\figlabel}{*}}
    {}
    {\label{fig:\figlabel}}%
    
\renewcommand{\figlabel}{*}



\end{figure}
\end{kep}
\renewcommand{\aspectratioPic}{1}
\renewcommand{\rotationAnglePic}{0}
\renewcommand{\svg}{*}


Jelölje az $\alpha$ \ szög ha a lejtőn felfele halad a robotref{fig:SMR4WLejtoOldalrol}, $\beta$  ha a robot a lejtőn oldalra halad \ref{fig:SMR4WLejtoSzembol}. 
Ha ismerjük a robot súlypontjának a pozícióját mindhárom tengelyen akkor kicsuszamolhatjuk a kerekek merőleges nyomóerejét a talajra a következő módszerrel.



\renewcommand{\img}{SajatRobot/SzerkAbrak/LejtoOldalrol.jpg}
\renewcommand{\sources}{*}
\renewcommand{\captionn}{$4W-SSMR$ típusú robot lejtőn felfele oldal nézetből.}
\renewcommand{\aspectratioPic}{0.5}
\renewcommand{\figlabel}{SMR4WLejtoOldalrol}
\begin{kep}
\begin{figure}[H]
\centering
\ifthenelse{\equal{\svg}{*}}
{
    \includegraphics[width=\aspectratioPic\textwidth,angle=\rotationAnglePic]{\img}
}
{
    \includesvg[width=\aspectratioPic\textwidth,angle=\rotationAnglePic]{\img}
}

 \ifthenelse{\equal{\sources}{*}}
    { \captionof{figure}{ \captionn}}
    { \captionof{figure}{ \mand{\mand{\captionn}{Forrás:}}{}} }
  	

\ifthenelse{\equal{\figlabel}{*}}
    {}
    {\label{fig:\figlabel}}%
    
\renewcommand{\figlabel}{*}



\end{figure}
\end{kep}
\renewcommand{\aspectratioPic}{1}
\renewcommand{\rotationAnglePic}{0}
\renewcommand{\svg}{*}


\renewcommand{\img}{SajatRobot/SzerkAbrak/LejtoSzembol.png}
\renewcommand{\sources}{*}
\renewcommand{\captionn}{$4W-SSMR$ típusú robot lejtőn első nézetből.}
\renewcommand{\aspectratioPic}{0.5}
\renewcommand{\figlabel}{SMR4WLejtoSzembol}
\begin{kep}
\begin{figure}[H]
\centering
\ifthenelse{\equal{\svg}{*}}
{
    \includegraphics[width=\aspectratioPic\textwidth,angle=\rotationAnglePic]{\img}
}
{
    \includesvg[width=\aspectratioPic\textwidth,angle=\rotationAnglePic]{\img}
}

 \ifthenelse{\equal{\sources}{*}}
    { \captionof{figure}{ \captionn}}
    { \captionof{figure}{ \mand{\mand{\captionn}{Forrás:}}{}} }
  	

\ifthenelse{\equal{\figlabel}{*}}
    {}
    {\label{fig:\figlabel}}%
    
\renewcommand{\figlabel}{*}



\end{figure}
\end{kep}
\renewcommand{\aspectratioPic}{1}
\renewcommand{\rotationAnglePic}{0}
\renewcommand{\svg}{*}



\iffalse
\begin{equation}
    AN=B \Rightarrow N=BA^{-1}
\end{equation}

\begin{equation}
G_xh=F_{difX}R_5\Rightarrow F_{difX}=\frac{G_{x}h}{R5}, \quad
G_yh=F_{difY}R_6\Rightarrow F_{difY}=\frac{G_{y}h}{R6}
\end{equation}

\begin{equation}
N_{comp} = \begin{bmatrix}
-\frac{1}{2}\\ 
\frac{1}{2}\\ 
-\frac{1}{2}\\ 
\frac{1}{2}
\end{bmatrix}F_{difX}
+
\begin{bmatrix}
-\frac{1}{2}\\ 
\frac{1}{2}\\ 
-\frac{1}{2}\\ 
\frac{1}{2}
\end{bmatrix}F_{difY}
\end{equation}

\begin{equation}
    N=BA^{-1} + N_{conp}(G_x,G_y)
\end{equation}

\begin{equation*}
A =\begin{pmatrix}
0 & \frac{R_5}{R_1} & \frac{R_7}{R_1} & \frac{R_6}{R_1}\\ 
\frac{R_5}{R_2} &  0&  \frac{R_6}{R_2}& \frac{R_7}{R_2} \\ 
\frac{R_7}{R_3} & \frac{R_6}{R_3} & 0 & \frac{R_5}{R_3} \\ 
\frac{R_6}{R_4} & \frac{R_7}{R_4} & \frac{R_5}{R_4}& 0
\end{pmatrix},\quad
N =\begin{bmatrix}
N_{FL} & N_{BL} & N_{FR} & N_{BR} 
\end{bmatrix}^T,
B=\begin{bmatrix}
G_z&& 
G_z&& 
G_z&& 
G_z
\end{bmatrix}^T
\end{equation*}

\begin{equation*}
    G=mg \textbf{ ahol m - a robot súlya.}
\end{equation*}

\begin{equation*}
R_1 = \sqrt{a^2+d^2},R_2 = \sqrt{a^2+c^2},\quad
R_3 = \sqrt{b^2+c^2}, R_4 = \sqrt{d^2+b^2}
\end{equation*}
\begin{equation*}
R_5 = d+c, R_6 = a+b,\quad
R_7 = \sqrt{(a+b^2)+(d+c)^2)} 
\end{equation*} 

\fi

Egy test nyugalomban van ha a rá ható erők eredője és a forgatónyomatékok eredője zero, ismerve a súlypont pozíciójának a koordinátáit a robot $VNR$-be akkor az \ref{eq:N1} egyenlettel meghatározzuk a $G_x$ erő által létrehozót nyomóerőket a $N_F$ és $N_B$ pontokban.

\begin{equation}
\label{eq:N1}
    N_{FGx}=\frac{hG_x}{c+d}
    ,\quad
    N_{BGx}= - N_{FGx}
\end{equation}

Meghatározzuk a $G_y$ erő által létrehozott nyomóerőket a $N_{RGy}$ és $N_{LGy}$ pontokban.

\begin{equation}
\label{eq:N2}
    N_{RGy}=\frac{hG_y}{a+b}
    ,\quad
    N_{LGy}= - N_{FGy}
\end{equation}


Ismerve a $N_{LGy}$ és $N_{RGy}$ pontokban ható erőket kiszámítjuk ezek eloszlását a robot kerekeire nézve, így megkapjuk azokat a nyomóerőket amelyet a \ref{fig:SMR4WLejtoSzembol} ábrán látható állapotban a $G_y$ gravitációból származó erő hoz létre.

\begin{equation}
\label{eq:N3}
N_{yBL}=\frac{dN_{LGy}}{c+d}
    ,\quad
F_{yFL}=-N_{yBL}
\end{equation}

\begin{equation}
\label{eq:N4}
N_{yBR}=\frac{dN_{RGy}}{c+d}
    ,\quad
F_{yFR}=-N_{yBR}
\end{equation}


Meghatározzuk a gravitáció Z komponense által létrehozót nyomóerőket a $F_F$ és a $F_B$ pontokban amelyhez hozzáadjuk a X komponens által létrehozót nyomóerőket ugyan ezekben a pontokban.

\begin{equation}
\label{eq:N5}
F_F = \frac{G_zd}{c+d} + N_{FGx}
    ,\quad
F_B = G_z-F_F + N_{BGx}
\end{equation}


Ismét kiszámoljuk a kerekekre nézve a nyomóerőket ismerve az $F_F$ és $F_B$ erőket.

\begin{equation}
\label{eq:N6}
F_{BR}=\frac{aF_B}{a+b}
    ,\quad
F_{BL}=F_{B}-F_{BR}
\end{equation}

\begin{equation}
\label{eq:N7}
F_{FR}=\frac{aF_F}{a+b}
    ,\quad
F_{FL}=F_{F}-F_{FR}
\end{equation}

A merőleges nyomóerő vektor az X,Y,Z gravitációs erők által létrehozott nyomóerők összegzéséből áll.


\begin{equation}
\label{eq:N8}
N_\perp =\begin{bmatrix}
F_{FL} + N_{yFL} & F_{BL} +  N_{yBL} & F_{FR} +  N_{yFR} & F_{BR} +  N_{yBR}
\end{bmatrix}^T
\end{equation}

\subsection{Súlypont (X,Y) komponensének a meghatározása}

Ismerve a robot méreteit $W$ jelölje a szélességét  míg a $L$ hosszúságát, kerék középpont között mérve.

A robotot vízszintes helyzetbe helyezzük, és minden kerek merőleges nyomóerejét megmérve mérleg segítségével megkapjuk a
$N_{FL},N_{FR},N_{BL},N_{BR}$ nyomóerőket.

\begin{equation}
\label{eq:N9}
W = a+b,\quad L = c+d
\end{equation}

Meghatározzuk a $a$ érteket ismerve a $F_{FR}$ pontban a nyomóerőt és kiszámolva a $F_F$ pontban a nyomóerőt a \ref{eq:N11} és \ref{eq:N12} egyenletek segítségével.

\begin{equation}
\label{eq:N10}
F_{FR}=\frac{aF_F}{W} \Rightarrow a = \frac{F_{FR}W}{F_F}
\end{equation}

\begin{equation}
\label{eq:N11}
G_{z}=F_{FR}+ F_{FL}+ F_{BR}+ F_{BL} 
\end{equation}

\begin{equation}
\label{eq:N12}
F_{F}=\frac{dG_z}{L} \Rightarrow d = \frac{F_{F}L}{G_z}
\end{equation}











\subsubsection{Meroleges nyomoero alakulasa a sulypont fugvenyeben}

A tételezzük fel hogy a robot súlya 28kg, a $BR$ kerek közepe legyen a (0,0) pont, $W\in(0m,0.6m)$ és $L\in(0m,0.6m)$ értéket vehet fel.
A \ref{fig:FRnyomoeroszim} ábran látható a $FR$ pontban a merőleges nyomóerő változása a súlypont pozíciójának a függvényben.


\pgfplotstableset{%
    x index=0,
    y index=1,
    header=true
}%

\pgfplotsset{every axis/.append style={
font=\large,
line width=1.2pt,
tick style={line width=0.8pt}}}

\pgfplotsset{
contour/every contour label/.style={
sloped,
transform shape,
inner sep=1pt,
every node/.style={mapped color!50!black,fill=white},
/pgf/number format/relative={\pgfplotspointmetarangeexponent},
},
}





\begin{kep}

\begin{figure}[H]
\centering

\begin{filecontents}{MerolegesNyomoeroCalc.dat}
   x            y       F_xFL        F_xBL       F_xFR       F_xBR   F_yFL       F_yBL       F_yFR       F_yBR       F_zFL   F_zBL       F_zFR   F_zBR
   0.00000    0.00000    0.00000    0.00000    3.13433   -3.13433   -0.00000    0.00000    0.00000   -0.00000    0.00000    0.00000    0.00000   31.00000
   0.00000    0.05000    0.00000    0.00000    3.13433   -3.13433   -0.23391    0.23391    0.23391   -0.23391    0.00000    0.00000    2.31343   28.68657
   0.00000    0.10000    0.00000    0.00000    3.13433   -3.13433   -0.46781    0.46781    0.46781   -0.46781    0.00000    0.00000    4.62687   26.37313
   0.00000    0.15000    0.00000    0.00000    3.13433   -3.13433   -0.70172    0.70172    0.70172   -0.70172    0.00000    0.00000    6.94030   24.05970
   0.00000    0.20000    0.00000    0.00000    3.13433   -3.13433   -0.93562    0.93562    0.93562   -0.93562    0.00000    0.00000    9.25373   21.74627
   0.00000    0.25000    0.00000    0.00000    3.13433   -3.13433   -1.16953    1.16953    1.16953   -1.16953    0.00000    0.00000   11.56716   19.43284
   0.00000    0.30000    0.00000    0.00000    3.13433   -3.13433   -1.40343    1.40343    1.40343   -1.40343    0.00000    0.00000   13.88060   17.11940
   0.00000    0.35000    0.00000    0.00000    3.13433   -3.13433   -1.63734    1.63734    1.63734   -1.63734    0.00000    0.00000   16.19403   14.80597
   0.00000    0.40000    0.00000    0.00000    3.13433   -3.13433   -1.87124    1.87124    1.87124   -1.87124    0.00000    0.00000   18.50746   12.49254
   0.00000    0.45000    0.00000    0.00000    3.13433   -3.13433   -2.10515    2.10515    2.10515   -2.10515    0.00000    0.00000   20.82090   10.17910
   0.00000    0.50000    0.00000    0.00000    3.13433   -3.13433   -2.33905    2.33905    2.33905   -2.33905    0.00000    0.00000   23.13433    7.86567
   0.00000    0.55000    0.00000    0.00000    3.13433   -3.13433   -2.57296    2.57296    2.57296   -2.57296    0.00000    0.00000   25.44776    5.55224
   0.00000    0.60000    0.00000    0.00000    3.13433   -3.13433   -2.80686    2.80686    2.80686   -2.80686    0.00000    0.00000   27.76119    3.23881
   0.00000    0.65000    0.00000    0.00000    3.13433   -3.13433   -3.04077    3.04077    3.04077   -3.04077    0.00000    0.00000   30.07463    0.92537
   0.05000    0.00000    0.23391   -0.23391    2.90042   -2.90042   -0.00000    0.00000    0.00000   -0.00000    0.00000    2.31343    0.00000   28.68657
   0.05000    0.05000    0.23391   -0.23391    2.90042   -2.90042   -0.23391    0.23391    0.23391   -0.23391    0.17264    2.14079    2.14079   26.54578
   0.05000    0.10000    0.23391   -0.23391    2.90042   -2.90042   -0.46781    0.46781    0.46781   -0.46781    0.34529    1.96814    4.28158   24.40499
   0.05000    0.15000    0.23391   -0.23391    2.90042   -2.90042   -0.70172    0.70172    0.70172   -0.70172    0.51793    1.79550    6.42237   22.26420
   0.05000    0.20000    0.23391   -0.23391    2.90042   -2.90042   -0.93562    0.93562    0.93562   -0.93562    0.69058    1.62286    8.56315   20.12341
   0.05000    0.25000    0.23391   -0.23391    2.90042   -2.90042   -1.16953    1.16953    1.16953   -1.16953    0.86322    1.45021   10.70394   17.98262
   0.05000    0.30000    0.23391   -0.23391    2.90042   -2.90042   -1.40343    1.40343    1.40343   -1.40343    1.03587    1.27757   12.84473   15.84184
   0.05000    0.35000    0.23391   -0.23391    2.90042   -2.90042   -1.63734    1.63734    1.63734   -1.63734    1.20851    1.10492   14.98552   13.70105
   0.05000    0.40000    0.23391   -0.23391    2.90042   -2.90042   -1.87124    1.87124    1.87124   -1.87124    1.38115    0.93228   17.12631   11.56026
   0.05000    0.45000    0.23391   -0.23391    2.90042   -2.90042   -2.10515    2.10515    2.10515   -2.10515    1.55380    0.75963   19.26710    9.41947
   0.05000    0.50000    0.23391   -0.23391    2.90042   -2.90042   -2.33905    2.33905    2.33905   -2.33905    1.72644    0.58699   21.40789    7.27868
   0.05000    0.55000    0.23391   -0.23391    2.90042   -2.90042   -2.57296    2.57296    2.57296   -2.57296    1.89909    0.41435   23.54867    5.13789
   0.05000    0.60000    0.23391   -0.23391    2.90042   -2.90042   -2.80686    2.80686    2.80686   -2.80686    2.07173    0.24170   25.68946    2.99710
   0.05000    0.65000    0.23391   -0.23391    2.90042   -2.90042   -3.04077    3.04077    3.04077   -3.04077    2.24438    0.06906   27.83025    0.85632
   0.10000    0.00000    0.46781   -0.46781    2.66652   -2.66652   -0.00000    0.00000    0.00000   -0.00000    0.00000    4.62687    0.00000   26.37313
   0.10000    0.05000    0.46781   -0.46781    2.66652   -2.66652   -0.23391    0.23391    0.23391   -0.23391    0.34529    4.28158    1.96814   24.40499
   0.10000    0.10000    0.46781   -0.46781    2.66652   -2.66652   -0.46781    0.46781    0.46781   -0.46781    0.69058    3.93629    3.93629   22.43685
   0.10000    0.15000    0.46781   -0.46781    2.66652   -2.66652   -0.70172    0.70172    0.70172   -0.70172    1.03587    3.59100    5.90443   20.46870
   0.10000    0.20000    0.46781   -0.46781    2.66652   -2.66652   -0.93562    0.93562    0.93562   -0.93562    1.38115    3.24571    7.87258   18.50056
   0.10000    0.25000    0.46781   -0.46781    2.66652   -2.66652   -1.16953    1.16953    1.16953   -1.16953    1.72644    2.90042    9.84072   16.53241
   0.10000    0.30000    0.46781   -0.46781    2.66652   -2.66652   -1.40343    1.40343    1.40343   -1.40343    2.07173    2.55513   11.80887   14.56427
   0.10000    0.35000    0.46781   -0.46781    2.66652   -2.66652   -1.63734    1.63734    1.63734   -1.63734    2.41702    2.20985   13.77701   12.59612
   0.10000    0.40000    0.46781   -0.46781    2.66652   -2.66652   -1.87124    1.87124    1.87124   -1.87124    2.76231    1.86456   15.74515   10.62798
   0.10000    0.45000    0.46781   -0.46781    2.66652   -2.66652   -2.10515    2.10515    2.10515   -2.10515    3.10760    1.51927   17.71330    8.65984
   0.10000    0.50000    0.46781   -0.46781    2.66652   -2.66652   -2.33905    2.33905    2.33905   -2.33905    3.45288    1.17398   19.68144    6.69169
   0.10000    0.55000    0.46781   -0.46781    2.66652   -2.66652   -2.57296    2.57296    2.57296   -2.57296    3.79817    0.82869   21.64959    4.72355
   0.10000    0.60000    0.46781   -0.46781    2.66652   -2.66652   -2.80686    2.80686    2.80686   -2.80686    4.14346    0.48340   23.61773    2.75540
   0.10000    0.65000    0.46781   -0.46781    2.66652   -2.66652   -3.04077    3.04077    3.04077   -3.04077    4.48875    0.13812   25.58588    0.78726
   0.15000    0.00000    0.70172   -0.70172    2.43261   -2.43261   -0.00000    0.00000    0.00000   -0.00000    0.00000    6.94030    0.00000   24.05970
   0.15000    0.05000    0.70172   -0.70172    2.43261   -2.43261   -0.23391    0.23391    0.23391   -0.23391    0.51793    6.42237    1.79550   22.26420
   0.15000    0.10000    0.70172   -0.70172    2.43261   -2.43261   -0.46781    0.46781    0.46781   -0.46781    1.03587    5.90443    3.59100   20.46870
   0.15000    0.15000    0.70172   -0.70172    2.43261   -2.43261   -0.70172    0.70172    0.70172   -0.70172    1.55380    5.38650    5.38650   18.67320
   0.15000    0.20000    0.70172   -0.70172    2.43261   -2.43261   -0.93562    0.93562    0.93562   -0.93562    2.07173    4.86857    7.18200   16.87770
   0.15000    0.25000    0.70172   -0.70172    2.43261   -2.43261   -1.16953    1.16953    1.16953   -1.16953    2.58966    4.35063    8.97750   15.08220
   0.15000    0.30000    0.70172   -0.70172    2.43261   -2.43261   -1.40343    1.40343    1.40343   -1.40343    3.10760    3.83270   10.77300   13.28670
   0.15000    0.35000    0.70172   -0.70172    2.43261   -2.43261   -1.63734    1.63734    1.63734   -1.63734    3.62553    3.31477   12.56850   11.49120
   0.15000    0.40000    0.70172   -0.70172    2.43261   -2.43261   -1.87124    1.87124    1.87124   -1.87124    4.14346    2.79684   14.36400    9.69570
   0.15000    0.45000    0.70172   -0.70172    2.43261   -2.43261   -2.10515    2.10515    2.10515   -2.10515    4.66139    2.27890   16.15950    7.90020
   0.15000    0.50000    0.70172   -0.70172    2.43261   -2.43261   -2.33905    2.33905    2.33905   -2.33905    5.17933    1.76097   17.95500    6.10470
   0.15000    0.55000    0.70172   -0.70172    2.43261   -2.43261   -2.57296    2.57296    2.57296   -2.57296    5.69726    1.24304   19.75050    4.30920
   0.15000    0.60000    0.70172   -0.70172    2.43261   -2.43261   -2.80686    2.80686    2.80686   -2.80686    6.21519    0.72511   21.54600    2.51370
   0.15000    0.65000    0.70172   -0.70172    2.43261   -2.43261   -3.04077    3.04077    3.04077   -3.04077    6.73313    0.20717   23.34150    0.71820
   0.20000    0.00000    0.93562   -0.93562    2.19871   -2.19871   -0.00000    0.00000    0.00000   -0.00000    0.00000    9.25373    0.00000   21.74627
   0.20000    0.05000    0.93562   -0.93562    2.19871   -2.19871   -0.23391    0.23391    0.23391   -0.23391    0.69058    8.56315    1.62286   20.12341
   0.20000    0.10000    0.93562   -0.93562    2.19871   -2.19871   -0.46781    0.46781    0.46781   -0.46781    1.38115    7.87258    3.24571   18.50056
   0.20000    0.15000    0.93562   -0.93562    2.19871   -2.19871   -0.70172    0.70172    0.70172   -0.70172    2.07173    7.18200    4.86857   16.87770
   0.20000    0.20000    0.93562   -0.93562    2.19871   -2.19871   -0.93562    0.93562    0.93562   -0.93562    2.76231    6.49142    6.49142   15.25485
   0.20000    0.25000    0.93562   -0.93562    2.19871   -2.19871   -1.16953    1.16953    1.16953   -1.16953    3.45288    5.80085    8.11428   13.63199
   0.20000    0.30000    0.93562   -0.93562    2.19871   -2.19871   -1.40343    1.40343    1.40343   -1.40343    4.14346    5.11027    9.73714   12.00913
   0.20000    0.35000    0.93562   -0.93562    2.19871   -2.19871   -1.63734    1.63734    1.63734   -1.63734    4.83404    4.41969   11.35999   10.38628
   0.20000    0.40000    0.93562   -0.93562    2.19871   -2.19871   -1.87124    1.87124    1.87124   -1.87124    5.52462    3.72912   12.98285    8.76342
   0.20000    0.45000    0.93562   -0.93562    2.19871   -2.19871   -2.10515    2.10515    2.10515   -2.10515    6.21519    3.03854   14.60570    7.14057
   0.20000    0.50000    0.93562   -0.93562    2.19871   -2.19871   -2.33905    2.33905    2.33905   -2.33905    6.90577    2.34796   16.22856    5.51771
   0.20000    0.55000    0.93562   -0.93562    2.19871   -2.19871   -2.57296    2.57296    2.57296   -2.57296    7.59635    1.65738   17.85141    3.89485
   0.20000    0.60000    0.93562   -0.93562    2.19871   -2.19871   -2.80686    2.80686    2.80686   -2.80686    8.28692    0.96681   19.47427    2.27200
   0.20000    0.65000    0.93562   -0.93562    2.19871   -2.19871   -3.04077    3.04077    3.04077   -3.04077    8.97750    0.27623   21.09713    0.64914
   0.25000    0.00000    1.16953   -1.16953    1.96480   -1.96480   -0.00000    0.00000    0.00000   -0.00000    0.00000   11.56716    0.00000   19.43284
   0.25000    0.05000    1.16953   -1.16953    1.96480   -1.96480   -0.23391    0.23391    0.23391   -0.23391    0.86322   10.70394    1.45021   17.98262
   0.25000    0.10000    1.16953   -1.16953    1.96480   -1.96480   -0.46781    0.46781    0.46781   -0.46781    1.72644    9.84072    2.90042   16.53241
   0.25000    0.15000    1.16953   -1.16953    1.96480   -1.96480   -0.70172    0.70172    0.70172   -0.70172    2.58966    8.97750    4.35063   15.08220
   0.25000    0.20000    1.16953   -1.16953    1.96480   -1.96480   -0.93562    0.93562    0.93562   -0.93562    3.45288    8.11428    5.80085   13.63199
   0.25000    0.25000    1.16953   -1.16953    1.96480   -1.96480   -1.16953    1.16953    1.16953   -1.16953    4.31611    7.25106    7.25106   12.18178
   0.25000    0.30000    1.16953   -1.16953    1.96480   -1.96480   -1.40343    1.40343    1.40343   -1.40343    5.17933    6.38784    8.70127   10.73157
   0.25000    0.35000    1.16953   -1.16953    1.96480   -1.96480   -1.63734    1.63734    1.63734   -1.63734    6.04255    5.52462   10.15148    9.28135
   0.25000    0.40000    1.16953   -1.16953    1.96480   -1.96480   -1.87124    1.87124    1.87124   -1.87124    6.90577    4.66139   11.60169    7.83114
   0.25000    0.45000    1.16953   -1.16953    1.96480   -1.96480   -2.10515    2.10515    2.10515   -2.10515    7.76899    3.79817   13.05190    6.38093
   0.25000    0.50000    1.16953   -1.16953    1.96480   -1.96480   -2.33905    2.33905    2.33905   -2.33905    8.63221    2.93495   14.50212    4.93072
   0.25000    0.55000    1.16953   -1.16953    1.96480   -1.96480   -2.57296    2.57296    2.57296   -2.57296    9.49543    2.07173   15.95233    3.48051
   0.25000    0.60000    1.16953   -1.16953    1.96480   -1.96480   -2.80686    2.80686    2.80686   -2.80686   10.35865    1.20851   17.40254    2.03030
   0.25000    0.65000    1.16953   -1.16953    1.96480   -1.96480   -3.04077    3.04077    3.04077   -3.04077   11.22188    0.34529   18.85275    0.58008
   0.30000    0.00000    1.40343   -1.40343    1.73090   -1.73090   -0.00000    0.00000    0.00000   -0.00000    0.00000   13.88060    0.00000   17.11940
   0.30000    0.05000    1.40343   -1.40343    1.73090   -1.73090   -0.23391    0.23391    0.23391   -0.23391    1.03587   12.84473    1.27757   15.84184
   0.30000    0.10000    1.40343   -1.40343    1.73090   -1.73090   -0.46781    0.46781    0.46781   -0.46781    2.07173   11.80887    2.55513   14.56427
   0.30000    0.15000    1.40343   -1.40343    1.73090   -1.73090   -0.70172    0.70172    0.70172   -0.70172    3.10760   10.77300    3.83270   13.28670
   0.30000    0.20000    1.40343   -1.40343    1.73090   -1.73090   -0.93562    0.93562    0.93562   -0.93562    4.14346    9.73714    5.11027   12.00913
   0.30000    0.25000    1.40343   -1.40343    1.73090   -1.73090   -1.16953    1.16953    1.16953   -1.16953    5.17933    8.70127    6.38784   10.73157
   0.30000    0.30000    1.40343   -1.40343    1.73090   -1.73090   -1.40343    1.40343    1.40343   -1.40343    6.21519    7.66540    7.66540    9.45400
   0.30000    0.35000    1.40343   -1.40343    1.73090   -1.73090   -1.63734    1.63734    1.63734   -1.63734    7.25106    6.62954    8.94297    8.17643
   0.30000    0.40000    1.40343   -1.40343    1.73090   -1.73090   -1.87124    1.87124    1.87124   -1.87124    8.28692    5.59367   10.22054    6.89886
   0.30000    0.45000    1.40343   -1.40343    1.73090   -1.73090   -2.10515    2.10515    2.10515   -2.10515    9.32279    4.55781   11.49811    5.62130
   0.30000    0.50000    1.40343   -1.40343    1.73090   -1.73090   -2.33905    2.33905    2.33905   -2.33905   10.35865    3.52194   12.77567    4.34373
   0.30000    0.55000    1.40343   -1.40343    1.73090   -1.73090   -2.57296    2.57296    2.57296   -2.57296   11.39452    2.48608   14.05324    3.06616
   0.30000    0.60000    1.40343   -1.40343    1.73090   -1.73090   -2.80686    2.80686    2.80686   -2.80686   12.43039    1.45021   15.33081    1.78859
   0.30000    0.65000    1.40343   -1.40343    1.73090   -1.73090   -3.04077    3.04077    3.04077   -3.04077   13.46625    0.41435   16.60838    0.51103
   0.35000    0.00000    1.63734   -1.63734    1.49699   -1.49699   -0.00000    0.00000    0.00000   -0.00000    0.00000   16.19403    0.00000   14.80597
   0.35000    0.05000    1.63734   -1.63734    1.49699   -1.49699   -0.23391    0.23391    0.23391   -0.23391    1.20851   14.98552    1.10492   13.70105
   0.35000    0.10000    1.63734   -1.63734    1.49699   -1.49699   -0.46781    0.46781    0.46781   -0.46781    2.41702   13.77701    2.20985   12.59612
   0.35000    0.15000    1.63734   -1.63734    1.49699   -1.49699   -0.70172    0.70172    0.70172   -0.70172    3.62553   12.56850    3.31477   11.49120
   0.35000    0.20000    1.63734   -1.63734    1.49699   -1.49699   -0.93562    0.93562    0.93562   -0.93562    4.83404   11.35999    4.41969   10.38628
   0.35000    0.25000    1.63734   -1.63734    1.49699   -1.49699   -1.16953    1.16953    1.16953   -1.16953    6.04255   10.15148    5.52462    9.28135
   0.35000    0.30000    1.63734   -1.63734    1.49699   -1.49699   -1.40343    1.40343    1.40343   -1.40343    7.25106    8.94297    6.62954    8.17643
   0.35000    0.35000    1.63734   -1.63734    1.49699   -1.49699   -1.63734    1.63734    1.63734   -1.63734    8.45957    7.73446    7.73446    7.07151
   0.35000    0.40000    1.63734   -1.63734    1.49699   -1.49699   -1.87124    1.87124    1.87124   -1.87124    9.66808    6.52595    8.83939    5.96658
   0.35000    0.45000    1.63734   -1.63734    1.49699   -1.49699   -2.10515    2.10515    2.10515   -2.10515   10.87659    5.31744    9.94431    4.86166
   0.35000    0.50000    1.63734   -1.63734    1.49699   -1.49699   -2.33905    2.33905    2.33905   -2.33905   12.08510    4.10893   11.04923    3.75674
   0.35000    0.55000    1.63734   -1.63734    1.49699   -1.49699   -2.57296    2.57296    2.57296   -2.57296   13.29361    2.90042   12.15415    2.65182
   0.35000    0.60000    1.63734   -1.63734    1.49699   -1.49699   -2.80686    2.80686    2.80686   -2.80686   14.50212    1.69191   13.25908    1.54689
   0.35000    0.65000    1.63734   -1.63734    1.49699   -1.49699   -3.04077    3.04077    3.04077   -3.04077   15.71063    0.48340   14.36400    0.44197
   0.40000    0.00000    1.87124   -1.87124    1.26309   -1.26309   -0.00000    0.00000    0.00000   -0.00000    0.00000   18.50746    0.00000   12.49254
   0.40000    0.05000    1.87124   -1.87124    1.26309   -1.26309   -0.23391    0.23391    0.23391   -0.23391    1.38115   17.12631    0.93228   11.56026
   0.40000    0.10000    1.87124   -1.87124    1.26309   -1.26309   -0.46781    0.46781    0.46781   -0.46781    2.76231   15.74515    1.86456   10.62798
   0.40000    0.15000    1.87124   -1.87124    1.26309   -1.26309   -0.70172    0.70172    0.70172   -0.70172    4.14346   14.36400    2.79684    9.69570
   0.40000    0.20000    1.87124   -1.87124    1.26309   -1.26309   -0.93562    0.93562    0.93562   -0.93562    5.52462   12.98285    3.72912    8.76342
   0.40000    0.25000    1.87124   -1.87124    1.26309   -1.26309   -1.16953    1.16953    1.16953   -1.16953    6.90577   11.60169    4.66139    7.83114
   0.40000    0.30000    1.87124   -1.87124    1.26309   -1.26309   -1.40343    1.40343    1.40343   -1.40343    8.28692   10.22054    5.59367    6.89886
   0.40000    0.35000    1.87124   -1.87124    1.26309   -1.26309   -1.63734    1.63734    1.63734   -1.63734    9.66808    8.83939    6.52595    5.96658
   0.40000    0.40000    1.87124   -1.87124    1.26309   -1.26309   -1.87124    1.87124    1.87124   -1.87124   11.04923    7.45823    7.45823    5.03431
   0.40000    0.45000    1.87124   -1.87124    1.26309   -1.26309   -2.10515    2.10515    2.10515   -2.10515   12.43039    6.07708    8.39051    4.10203
   0.40000    0.50000    1.87124   -1.87124    1.26309   -1.26309   -2.33905    2.33905    2.33905   -2.33905   13.81154    4.69592    9.32279    3.16975
   0.40000    0.55000    1.87124   -1.87124    1.26309   -1.26309   -2.57296    2.57296    2.57296   -2.57296   15.19269    3.31477   10.25507    2.23747
   0.40000    0.60000    1.87124   -1.87124    1.26309   -1.26309   -2.80686    2.80686    2.80686   -2.80686   16.57385    1.93362   11.18735    1.30519
   0.40000    0.65000    1.87124   -1.87124    1.26309   -1.26309   -3.04077    3.04077    3.04077   -3.04077   17.95500    0.55246   12.11963    0.37291
   0.45000    0.00000    2.10515   -2.10515    1.02918   -1.02918   -0.00000    0.00000    0.00000   -0.00000    0.00000   20.82090    0.00000   10.17910
   0.45000    0.05000    2.10515   -2.10515    1.02918   -1.02918   -0.23391    0.23391    0.23391   -0.23391    1.55380   19.26710    0.75963    9.41947
   0.45000    0.10000    2.10515   -2.10515    1.02918   -1.02918   -0.46781    0.46781    0.46781   -0.46781    3.10760   17.71330    1.51927    8.65984
   0.45000    0.15000    2.10515   -2.10515    1.02918   -1.02918   -0.70172    0.70172    0.70172   -0.70172    4.66139   16.15950    2.27890    7.90020
   0.45000    0.20000    2.10515   -2.10515    1.02918   -1.02918   -0.93562    0.93562    0.93562   -0.93562    6.21519   14.60570    3.03854    7.14057
   0.45000    0.25000    2.10515   -2.10515    1.02918   -1.02918   -1.16953    1.16953    1.16953   -1.16953    7.76899   13.05190    3.79817    6.38093
   0.45000    0.30000    2.10515   -2.10515    1.02918   -1.02918   -1.40343    1.40343    1.40343   -1.40343    9.32279   11.49811    4.55781    5.62130
   0.45000    0.35000    2.10515   -2.10515    1.02918   -1.02918   -1.63734    1.63734    1.63734   -1.63734   10.87659    9.94431    5.31744    4.86166
   0.45000    0.40000    2.10515   -2.10515    1.02918   -1.02918   -1.87124    1.87124    1.87124   -1.87124   12.43039    8.39051    6.07708    4.10203
   0.45000    0.45000    2.10515   -2.10515    1.02918   -1.02918   -2.10515    2.10515    2.10515   -2.10515   13.98418    6.83671    6.83671    3.34239
   0.45000    0.50000    2.10515   -2.10515    1.02918   -1.02918   -2.33905    2.33905    2.33905   -2.33905   15.53798    5.28291    7.59635    2.58276
   0.45000    0.55000    2.10515   -2.10515    1.02918   -1.02918   -2.57296    2.57296    2.57296   -2.57296   17.09178    3.72912    8.35598    1.82312
   0.45000    0.60000    2.10515   -2.10515    1.02918   -1.02918   -2.80686    2.80686    2.80686   -2.80686   18.64558    2.17532    9.11562    1.06349
   0.45000    0.65000    2.10515   -2.10515    1.02918   -1.02918   -3.04077    3.04077    3.04077   -3.04077   20.19938    0.62152    9.87525    0.30385
   0.50000    0.00000    2.33905   -2.33905    0.79528   -0.79528   -0.00000    0.00000    0.00000   -0.00000    0.00000   23.13433    0.00000    7.86567
   0.50000    0.05000    2.33905   -2.33905    0.79528   -0.79528   -0.23391    0.23391    0.23391   -0.23391    1.72644   21.40789    0.58699    7.27868
   0.50000    0.10000    2.33905   -2.33905    0.79528   -0.79528   -0.46781    0.46781    0.46781   -0.46781    3.45288   19.68144    1.17398    6.69169
   0.50000    0.15000    2.33905   -2.33905    0.79528   -0.79528   -0.70172    0.70172    0.70172   -0.70172    5.17933   17.95500    1.76097    6.10470
   0.50000    0.20000    2.33905   -2.33905    0.79528   -0.79528   -0.93562    0.93562    0.93562   -0.93562    6.90577   16.22856    2.34796    5.51771
   0.50000    0.25000    2.33905   -2.33905    0.79528   -0.79528   -1.16953    1.16953    1.16953   -1.16953    8.63221   14.50212    2.93495    4.93072
   0.50000    0.30000    2.33905   -2.33905    0.79528   -0.79528   -1.40343    1.40343    1.40343   -1.40343   10.35865   12.77567    3.52194    4.34373
   0.50000    0.35000    2.33905   -2.33905    0.79528   -0.79528   -1.63734    1.63734    1.63734   -1.63734   12.08510   11.04923    4.10893    3.75674
   0.50000    0.40000    2.33905   -2.33905    0.79528   -0.79528   -1.87124    1.87124    1.87124   -1.87124   13.81154    9.32279    4.69592    3.16975
   0.50000    0.45000    2.33905   -2.33905    0.79528   -0.79528   -2.10515    2.10515    2.10515   -2.10515   15.53798    7.59635    5.28291    2.58276
   0.50000    0.50000    2.33905   -2.33905    0.79528   -0.79528   -2.33905    2.33905    2.33905   -2.33905   17.26442    5.86990    5.86990    1.99577
   0.50000    0.55000    2.33905   -2.33905    0.79528   -0.79528   -2.57296    2.57296    2.57296   -2.57296   18.99087    4.14346    6.45689    1.40878
   0.50000    0.60000    2.33905   -2.33905    0.79528   -0.79528   -2.80686    2.80686    2.80686   -2.80686   20.71731    2.41702    7.04389    0.82179
   0.50000    0.65000    2.33905   -2.33905    0.79528   -0.79528   -3.04077    3.04077    3.04077   -3.04077   22.44375    0.69058    7.63088    0.23480
   0.55000    0.00000    2.57296   -2.57296    0.56137   -0.56137   -0.00000    0.00000    0.00000   -0.00000    0.00000   25.44776    0.00000    5.55224
   0.55000    0.05000    2.57296   -2.57296    0.56137   -0.56137   -0.23391    0.23391    0.23391   -0.23391    1.89909   23.54867    0.41435    5.13789
   0.55000    0.10000    2.57296   -2.57296    0.56137   -0.56137   -0.46781    0.46781    0.46781   -0.46781    3.79817   21.64959    0.82869    4.72355
   0.55000    0.15000    2.57296   -2.57296    0.56137   -0.56137   -0.70172    0.70172    0.70172   -0.70172    5.69726   19.75050    1.24304    4.30920
   0.55000    0.20000    2.57296   -2.57296    0.56137   -0.56137   -0.93562    0.93562    0.93562   -0.93562    7.59635   17.85141    1.65738    3.89485
   0.55000    0.25000    2.57296   -2.57296    0.56137   -0.56137   -1.16953    1.16953    1.16953   -1.16953    9.49543   15.95233    2.07173    3.48051
   0.55000    0.30000    2.57296   -2.57296    0.56137   -0.56137   -1.40343    1.40343    1.40343   -1.40343   11.39452   14.05324    2.48608    3.06616
   0.55000    0.35000    2.57296   -2.57296    0.56137   -0.56137   -1.63734    1.63734    1.63734   -1.63734   13.29361   12.15415    2.90042    2.65182
   0.55000    0.40000    2.57296   -2.57296    0.56137   -0.56137   -1.87124    1.87124    1.87124   -1.87124   15.19269   10.25507    3.31477    2.23747
   0.55000    0.45000    2.57296   -2.57296    0.56137   -0.56137   -2.10515    2.10515    2.10515   -2.10515   17.09178    8.35598    3.72912    1.82312
   0.55000    0.50000    2.57296   -2.57296    0.56137   -0.56137   -2.33905    2.33905    2.33905   -2.33905   18.99087    6.45689    4.14346    1.40878
   0.55000    0.55000    2.57296   -2.57296    0.56137   -0.56137   -2.57296    2.57296    2.57296   -2.57296   20.88995    4.55781    4.55781    0.99443
   0.55000    0.60000    2.57296   -2.57296    0.56137   -0.56137   -2.80686    2.80686    2.80686   -2.80686   22.78904    2.65872    4.97215    0.58008
   0.55000    0.65000    2.57296   -2.57296    0.56137   -0.56137   -3.04077    3.04077    3.04077   -3.04077   24.68813    0.75963    5.38650    0.16574
   0.60000    0.00000    2.80686   -2.80686    0.32747   -0.32747   -0.00000    0.00000    0.00000   -0.00000    0.00000   27.76119    0.00000    3.23881
   0.60000    0.05000    2.80686   -2.80686    0.32747   -0.32747   -0.23391    0.23391    0.23391   -0.23391    2.07173   25.68946    0.24170    2.99710
   0.60000    0.10000    2.80686   -2.80686    0.32747   -0.32747   -0.46781    0.46781    0.46781   -0.46781    4.14346   23.61773    0.48340    2.75540
   0.60000    0.15000    2.80686   -2.80686    0.32747   -0.32747   -0.70172    0.70172    0.70172   -0.70172    6.21519   21.54600    0.72511    2.51370
   0.60000    0.20000    2.80686   -2.80686    0.32747   -0.32747   -0.93562    0.93562    0.93562   -0.93562    8.28692   19.47427    0.96681    2.27200
   0.60000    0.25000    2.80686   -2.80686    0.32747   -0.32747   -1.16953    1.16953    1.16953   -1.16953   10.35865   17.40254    1.20851    2.03030
   0.60000    0.30000    2.80686   -2.80686    0.32747   -0.32747   -1.40343    1.40343    1.40343   -1.40343   12.43039   15.33081    1.45021    1.78859
   0.60000    0.35000    2.80686   -2.80686    0.32747   -0.32747   -1.63734    1.63734    1.63734   -1.63734   14.50212   13.25908    1.69191    1.54689
   0.60000    0.40000    2.80686   -2.80686    0.32747   -0.32747   -1.87124    1.87124    1.87124   -1.87124   16.57385   11.18735    1.93362    1.30519
   0.60000    0.45000    2.80686   -2.80686    0.32747   -0.32747   -2.10515    2.10515    2.10515   -2.10515   18.64558    9.11562    2.17532    1.06349
   0.60000    0.50000    2.80686   -2.80686    0.32747   -0.32747   -2.33905    2.33905    2.33905   -2.33905   20.71731    7.04389    2.41702    0.82179
   0.60000    0.55000    2.80686   -2.80686    0.32747   -0.32747   -2.57296    2.57296    2.57296   -2.57296   22.78904    4.97215    2.65872    0.58008
   0.60000    0.60000    2.80686   -2.80686    0.32747   -0.32747   -2.80686    2.80686    2.80686   -2.80686   24.86077    2.90042    2.90042    0.33838
   0.60000    0.65000    2.80686   -2.80686    0.32747   -0.32747   -3.04077    3.04077    3.04077   -3.04077   26.93250    0.82869    3.14213    0.09668
   0.65000    0.00000    3.04077   -3.04077    0.09356   -0.09356   -0.00000    0.00000    0.00000   -0.00000    0.00000   30.07463    0.00000    0.92537
   0.65000    0.05000    3.04077   -3.04077    0.09356   -0.09356   -0.23391    0.23391    0.23391   -0.23391    2.24438   27.83025    0.06906    0.85632
   0.65000    0.10000    3.04077   -3.04077    0.09356   -0.09356   -0.46781    0.46781    0.46781   -0.46781    4.48875   25.58588    0.13812    0.78726
   0.65000    0.15000    3.04077   -3.04077    0.09356   -0.09356   -0.70172    0.70172    0.70172   -0.70172    6.73313   23.34150    0.20717    0.71820
   0.65000    0.20000    3.04077   -3.04077    0.09356   -0.09356   -0.93562    0.93562    0.93562   -0.93562    8.97750   21.09713    0.27623    0.64914
   0.65000    0.25000    3.04077   -3.04077    0.09356   -0.09356   -1.16953    1.16953    1.16953   -1.16953   11.22188   18.85275    0.34529    0.58008
   0.65000    0.30000    3.04077   -3.04077    0.09356   -0.09356   -1.40343    1.40343    1.40343   -1.40343   13.46625   16.60838    0.41435    0.51103
   0.65000    0.35000    3.04077   -3.04077    0.09356   -0.09356   -1.63734    1.63734    1.63734   -1.63734   15.71063   14.36400    0.48340    0.44197
   0.65000    0.40000    3.04077   -3.04077    0.09356   -0.09356   -1.87124    1.87124    1.87124   -1.87124   17.95500   12.11963    0.55246    0.37291
   0.65000    0.45000    3.04077   -3.04077    0.09356   -0.09356   -2.10515    2.10515    2.10515   -2.10515   20.19938    9.87525    0.62152    0.30385
   0.65000    0.50000    3.04077   -3.04077    0.09356   -0.09356   -2.33905    2.33905    2.33905   -2.33905   22.44375    7.63088    0.69058    0.23480
   0.65000    0.55000    3.04077   -3.04077    0.09356   -0.09356   -2.57296    2.57296    2.57296   -2.57296   24.68813    5.38650    0.75963    0.16574
   0.65000    0.60000    3.04077   -3.04077    0.09356   -0.09356   -2.80686    2.80686    2.80686   -2.80686   26.93250    3.14213    0.82869    0.09668
   0.65000    0.65000    3.04077   -3.04077    0.09356   -0.09356   -3.04077    3.04077    3.04077   -3.04077   29.17688    0.89775    0.89775    0.02762
\end{filecontents}
\pgfplotstableread{MerolegesNyomoeroCalc.dat}{\pistonkinetics}

\begin{tikzpicture}
\pgfplotsset{every axis plot/.append style={very thick}}
\setcaptionsubtype
\begin{axis}[grid,xlabel=m,ylabel=m,zlabel=N]

\addplot3 [surf,mesh/rows=196, mesh/cols=14,mesh/check=false ] table [header=true, x = x, y=y, z=F_zFL] {\pistonkinetics};
\end{axis}
\end{tikzpicture}
\caption[]{Kerek nyomoero valtozasa a sulypont fugvenyeben ha $\alpha=0 es \beta=0$}
\label{fig:FRnyomoeroszim}

\end{figure}




\begin{figure}[H]
\centering

\begin{filecontents}{MerolegesNyomoeroCalc.dat}
   x            y       F_xFL        F_xBL       F_xFR       F_xBR   F_yFL       F_yBL       F_yFR       F_yBR       F_zFL   F_zBL       F_zFR   F_zBR
   0.00000    0.00000    0.00000    0.00000    3.13433   -3.13433   -0.00000    0.00000    0.00000   -0.00000    0.00000    0.00000    0.00000   31.00000
   0.00000    0.05000    0.00000    0.00000    3.13433   -3.13433   -0.23391    0.23391    0.23391   -0.23391    0.00000    0.00000    2.31343   28.68657
   0.00000    0.10000    0.00000    0.00000    3.13433   -3.13433   -0.46781    0.46781    0.46781   -0.46781    0.00000    0.00000    4.62687   26.37313
   0.00000    0.15000    0.00000    0.00000    3.13433   -3.13433   -0.70172    0.70172    0.70172   -0.70172    0.00000    0.00000    6.94030   24.05970
   0.00000    0.20000    0.00000    0.00000    3.13433   -3.13433   -0.93562    0.93562    0.93562   -0.93562    0.00000    0.00000    9.25373   21.74627
   0.00000    0.25000    0.00000    0.00000    3.13433   -3.13433   -1.16953    1.16953    1.16953   -1.16953    0.00000    0.00000   11.56716   19.43284
   0.00000    0.30000    0.00000    0.00000    3.13433   -3.13433   -1.40343    1.40343    1.40343   -1.40343    0.00000    0.00000   13.88060   17.11940
   0.00000    0.35000    0.00000    0.00000    3.13433   -3.13433   -1.63734    1.63734    1.63734   -1.63734    0.00000    0.00000   16.19403   14.80597
   0.00000    0.40000    0.00000    0.00000    3.13433   -3.13433   -1.87124    1.87124    1.87124   -1.87124    0.00000    0.00000   18.50746   12.49254
   0.00000    0.45000    0.00000    0.00000    3.13433   -3.13433   -2.10515    2.10515    2.10515   -2.10515    0.00000    0.00000   20.82090   10.17910
   0.00000    0.50000    0.00000    0.00000    3.13433   -3.13433   -2.33905    2.33905    2.33905   -2.33905    0.00000    0.00000   23.13433    7.86567
   0.00000    0.55000    0.00000    0.00000    3.13433   -3.13433   -2.57296    2.57296    2.57296   -2.57296    0.00000    0.00000   25.44776    5.55224
   0.00000    0.60000    0.00000    0.00000    3.13433   -3.13433   -2.80686    2.80686    2.80686   -2.80686    0.00000    0.00000   27.76119    3.23881
   0.00000    0.65000    0.00000    0.00000    3.13433   -3.13433   -3.04077    3.04077    3.04077   -3.04077    0.00000    0.00000   30.07463    0.92537
   0.05000    0.00000    0.23391   -0.23391    2.90042   -2.90042   -0.00000    0.00000    0.00000   -0.00000    0.00000    2.31343    0.00000   28.68657
   0.05000    0.05000    0.23391   -0.23391    2.90042   -2.90042   -0.23391    0.23391    0.23391   -0.23391    0.17264    2.14079    2.14079   26.54578
   0.05000    0.10000    0.23391   -0.23391    2.90042   -2.90042   -0.46781    0.46781    0.46781   -0.46781    0.34529    1.96814    4.28158   24.40499
   0.05000    0.15000    0.23391   -0.23391    2.90042   -2.90042   -0.70172    0.70172    0.70172   -0.70172    0.51793    1.79550    6.42237   22.26420
   0.05000    0.20000    0.23391   -0.23391    2.90042   -2.90042   -0.93562    0.93562    0.93562   -0.93562    0.69058    1.62286    8.56315   20.12341
   0.05000    0.25000    0.23391   -0.23391    2.90042   -2.90042   -1.16953    1.16953    1.16953   -1.16953    0.86322    1.45021   10.70394   17.98262
   0.05000    0.30000    0.23391   -0.23391    2.90042   -2.90042   -1.40343    1.40343    1.40343   -1.40343    1.03587    1.27757   12.84473   15.84184
   0.05000    0.35000    0.23391   -0.23391    2.90042   -2.90042   -1.63734    1.63734    1.63734   -1.63734    1.20851    1.10492   14.98552   13.70105
   0.05000    0.40000    0.23391   -0.23391    2.90042   -2.90042   -1.87124    1.87124    1.87124   -1.87124    1.38115    0.93228   17.12631   11.56026
   0.05000    0.45000    0.23391   -0.23391    2.90042   -2.90042   -2.10515    2.10515    2.10515   -2.10515    1.55380    0.75963   19.26710    9.41947
   0.05000    0.50000    0.23391   -0.23391    2.90042   -2.90042   -2.33905    2.33905    2.33905   -2.33905    1.72644    0.58699   21.40789    7.27868
   0.05000    0.55000    0.23391   -0.23391    2.90042   -2.90042   -2.57296    2.57296    2.57296   -2.57296    1.89909    0.41435   23.54867    5.13789
   0.05000    0.60000    0.23391   -0.23391    2.90042   -2.90042   -2.80686    2.80686    2.80686   -2.80686    2.07173    0.24170   25.68946    2.99710
   0.05000    0.65000    0.23391   -0.23391    2.90042   -2.90042   -3.04077    3.04077    3.04077   -3.04077    2.24438    0.06906   27.83025    0.85632
   0.10000    0.00000    0.46781   -0.46781    2.66652   -2.66652   -0.00000    0.00000    0.00000   -0.00000    0.00000    4.62687    0.00000   26.37313
   0.10000    0.05000    0.46781   -0.46781    2.66652   -2.66652   -0.23391    0.23391    0.23391   -0.23391    0.34529    4.28158    1.96814   24.40499
   0.10000    0.10000    0.46781   -0.46781    2.66652   -2.66652   -0.46781    0.46781    0.46781   -0.46781    0.69058    3.93629    3.93629   22.43685
   0.10000    0.15000    0.46781   -0.46781    2.66652   -2.66652   -0.70172    0.70172    0.70172   -0.70172    1.03587    3.59100    5.90443   20.46870
   0.10000    0.20000    0.46781   -0.46781    2.66652   -2.66652   -0.93562    0.93562    0.93562   -0.93562    1.38115    3.24571    7.87258   18.50056
   0.10000    0.25000    0.46781   -0.46781    2.66652   -2.66652   -1.16953    1.16953    1.16953   -1.16953    1.72644    2.90042    9.84072   16.53241
   0.10000    0.30000    0.46781   -0.46781    2.66652   -2.66652   -1.40343    1.40343    1.40343   -1.40343    2.07173    2.55513   11.80887   14.56427
   0.10000    0.35000    0.46781   -0.46781    2.66652   -2.66652   -1.63734    1.63734    1.63734   -1.63734    2.41702    2.20985   13.77701   12.59612
   0.10000    0.40000    0.46781   -0.46781    2.66652   -2.66652   -1.87124    1.87124    1.87124   -1.87124    2.76231    1.86456   15.74515   10.62798
   0.10000    0.45000    0.46781   -0.46781    2.66652   -2.66652   -2.10515    2.10515    2.10515   -2.10515    3.10760    1.51927   17.71330    8.65984
   0.10000    0.50000    0.46781   -0.46781    2.66652   -2.66652   -2.33905    2.33905    2.33905   -2.33905    3.45288    1.17398   19.68144    6.69169
   0.10000    0.55000    0.46781   -0.46781    2.66652   -2.66652   -2.57296    2.57296    2.57296   -2.57296    3.79817    0.82869   21.64959    4.72355
   0.10000    0.60000    0.46781   -0.46781    2.66652   -2.66652   -2.80686    2.80686    2.80686   -2.80686    4.14346    0.48340   23.61773    2.75540
   0.10000    0.65000    0.46781   -0.46781    2.66652   -2.66652   -3.04077    3.04077    3.04077   -3.04077    4.48875    0.13812   25.58588    0.78726
   0.15000    0.00000    0.70172   -0.70172    2.43261   -2.43261   -0.00000    0.00000    0.00000   -0.00000    0.00000    6.94030    0.00000   24.05970
   0.15000    0.05000    0.70172   -0.70172    2.43261   -2.43261   -0.23391    0.23391    0.23391   -0.23391    0.51793    6.42237    1.79550   22.26420
   0.15000    0.10000    0.70172   -0.70172    2.43261   -2.43261   -0.46781    0.46781    0.46781   -0.46781    1.03587    5.90443    3.59100   20.46870
   0.15000    0.15000    0.70172   -0.70172    2.43261   -2.43261   -0.70172    0.70172    0.70172   -0.70172    1.55380    5.38650    5.38650   18.67320
   0.15000    0.20000    0.70172   -0.70172    2.43261   -2.43261   -0.93562    0.93562    0.93562   -0.93562    2.07173    4.86857    7.18200   16.87770
   0.15000    0.25000    0.70172   -0.70172    2.43261   -2.43261   -1.16953    1.16953    1.16953   -1.16953    2.58966    4.35063    8.97750   15.08220
   0.15000    0.30000    0.70172   -0.70172    2.43261   -2.43261   -1.40343    1.40343    1.40343   -1.40343    3.10760    3.83270   10.77300   13.28670
   0.15000    0.35000    0.70172   -0.70172    2.43261   -2.43261   -1.63734    1.63734    1.63734   -1.63734    3.62553    3.31477   12.56850   11.49120
   0.15000    0.40000    0.70172   -0.70172    2.43261   -2.43261   -1.87124    1.87124    1.87124   -1.87124    4.14346    2.79684   14.36400    9.69570
   0.15000    0.45000    0.70172   -0.70172    2.43261   -2.43261   -2.10515    2.10515    2.10515   -2.10515    4.66139    2.27890   16.15950    7.90020
   0.15000    0.50000    0.70172   -0.70172    2.43261   -2.43261   -2.33905    2.33905    2.33905   -2.33905    5.17933    1.76097   17.95500    6.10470
   0.15000    0.55000    0.70172   -0.70172    2.43261   -2.43261   -2.57296    2.57296    2.57296   -2.57296    5.69726    1.24304   19.75050    4.30920
   0.15000    0.60000    0.70172   -0.70172    2.43261   -2.43261   -2.80686    2.80686    2.80686   -2.80686    6.21519    0.72511   21.54600    2.51370
   0.15000    0.65000    0.70172   -0.70172    2.43261   -2.43261   -3.04077    3.04077    3.04077   -3.04077    6.73313    0.20717   23.34150    0.71820
   0.20000    0.00000    0.93562   -0.93562    2.19871   -2.19871   -0.00000    0.00000    0.00000   -0.00000    0.00000    9.25373    0.00000   21.74627
   0.20000    0.05000    0.93562   -0.93562    2.19871   -2.19871   -0.23391    0.23391    0.23391   -0.23391    0.69058    8.56315    1.62286   20.12341
   0.20000    0.10000    0.93562   -0.93562    2.19871   -2.19871   -0.46781    0.46781    0.46781   -0.46781    1.38115    7.87258    3.24571   18.50056
   0.20000    0.15000    0.93562   -0.93562    2.19871   -2.19871   -0.70172    0.70172    0.70172   -0.70172    2.07173    7.18200    4.86857   16.87770
   0.20000    0.20000    0.93562   -0.93562    2.19871   -2.19871   -0.93562    0.93562    0.93562   -0.93562    2.76231    6.49142    6.49142   15.25485
   0.20000    0.25000    0.93562   -0.93562    2.19871   -2.19871   -1.16953    1.16953    1.16953   -1.16953    3.45288    5.80085    8.11428   13.63199
   0.20000    0.30000    0.93562   -0.93562    2.19871   -2.19871   -1.40343    1.40343    1.40343   -1.40343    4.14346    5.11027    9.73714   12.00913
   0.20000    0.35000    0.93562   -0.93562    2.19871   -2.19871   -1.63734    1.63734    1.63734   -1.63734    4.83404    4.41969   11.35999   10.38628
   0.20000    0.40000    0.93562   -0.93562    2.19871   -2.19871   -1.87124    1.87124    1.87124   -1.87124    5.52462    3.72912   12.98285    8.76342
   0.20000    0.45000    0.93562   -0.93562    2.19871   -2.19871   -2.10515    2.10515    2.10515   -2.10515    6.21519    3.03854   14.60570    7.14057
   0.20000    0.50000    0.93562   -0.93562    2.19871   -2.19871   -2.33905    2.33905    2.33905   -2.33905    6.90577    2.34796   16.22856    5.51771
   0.20000    0.55000    0.93562   -0.93562    2.19871   -2.19871   -2.57296    2.57296    2.57296   -2.57296    7.59635    1.65738   17.85141    3.89485
   0.20000    0.60000    0.93562   -0.93562    2.19871   -2.19871   -2.80686    2.80686    2.80686   -2.80686    8.28692    0.96681   19.47427    2.27200
   0.20000    0.65000    0.93562   -0.93562    2.19871   -2.19871   -3.04077    3.04077    3.04077   -3.04077    8.97750    0.27623   21.09713    0.64914
   0.25000    0.00000    1.16953   -1.16953    1.96480   -1.96480   -0.00000    0.00000    0.00000   -0.00000    0.00000   11.56716    0.00000   19.43284
   0.25000    0.05000    1.16953   -1.16953    1.96480   -1.96480   -0.23391    0.23391    0.23391   -0.23391    0.86322   10.70394    1.45021   17.98262
   0.25000    0.10000    1.16953   -1.16953    1.96480   -1.96480   -0.46781    0.46781    0.46781   -0.46781    1.72644    9.84072    2.90042   16.53241
   0.25000    0.15000    1.16953   -1.16953    1.96480   -1.96480   -0.70172    0.70172    0.70172   -0.70172    2.58966    8.97750    4.35063   15.08220
   0.25000    0.20000    1.16953   -1.16953    1.96480   -1.96480   -0.93562    0.93562    0.93562   -0.93562    3.45288    8.11428    5.80085   13.63199
   0.25000    0.25000    1.16953   -1.16953    1.96480   -1.96480   -1.16953    1.16953    1.16953   -1.16953    4.31611    7.25106    7.25106   12.18178
   0.25000    0.30000    1.16953   -1.16953    1.96480   -1.96480   -1.40343    1.40343    1.40343   -1.40343    5.17933    6.38784    8.70127   10.73157
   0.25000    0.35000    1.16953   -1.16953    1.96480   -1.96480   -1.63734    1.63734    1.63734   -1.63734    6.04255    5.52462   10.15148    9.28135
   0.25000    0.40000    1.16953   -1.16953    1.96480   -1.96480   -1.87124    1.87124    1.87124   -1.87124    6.90577    4.66139   11.60169    7.83114
   0.25000    0.45000    1.16953   -1.16953    1.96480   -1.96480   -2.10515    2.10515    2.10515   -2.10515    7.76899    3.79817   13.05190    6.38093
   0.25000    0.50000    1.16953   -1.16953    1.96480   -1.96480   -2.33905    2.33905    2.33905   -2.33905    8.63221    2.93495   14.50212    4.93072
   0.25000    0.55000    1.16953   -1.16953    1.96480   -1.96480   -2.57296    2.57296    2.57296   -2.57296    9.49543    2.07173   15.95233    3.48051
   0.25000    0.60000    1.16953   -1.16953    1.96480   -1.96480   -2.80686    2.80686    2.80686   -2.80686   10.35865    1.20851   17.40254    2.03030
   0.25000    0.65000    1.16953   -1.16953    1.96480   -1.96480   -3.04077    3.04077    3.04077   -3.04077   11.22188    0.34529   18.85275    0.58008
   0.30000    0.00000    1.40343   -1.40343    1.73090   -1.73090   -0.00000    0.00000    0.00000   -0.00000    0.00000   13.88060    0.00000   17.11940
   0.30000    0.05000    1.40343   -1.40343    1.73090   -1.73090   -0.23391    0.23391    0.23391   -0.23391    1.03587   12.84473    1.27757   15.84184
   0.30000    0.10000    1.40343   -1.40343    1.73090   -1.73090   -0.46781    0.46781    0.46781   -0.46781    2.07173   11.80887    2.55513   14.56427
   0.30000    0.15000    1.40343   -1.40343    1.73090   -1.73090   -0.70172    0.70172    0.70172   -0.70172    3.10760   10.77300    3.83270   13.28670
   0.30000    0.20000    1.40343   -1.40343    1.73090   -1.73090   -0.93562    0.93562    0.93562   -0.93562    4.14346    9.73714    5.11027   12.00913
   0.30000    0.25000    1.40343   -1.40343    1.73090   -1.73090   -1.16953    1.16953    1.16953   -1.16953    5.17933    8.70127    6.38784   10.73157
   0.30000    0.30000    1.40343   -1.40343    1.73090   -1.73090   -1.40343    1.40343    1.40343   -1.40343    6.21519    7.66540    7.66540    9.45400
   0.30000    0.35000    1.40343   -1.40343    1.73090   -1.73090   -1.63734    1.63734    1.63734   -1.63734    7.25106    6.62954    8.94297    8.17643
   0.30000    0.40000    1.40343   -1.40343    1.73090   -1.73090   -1.87124    1.87124    1.87124   -1.87124    8.28692    5.59367   10.22054    6.89886
   0.30000    0.45000    1.40343   -1.40343    1.73090   -1.73090   -2.10515    2.10515    2.10515   -2.10515    9.32279    4.55781   11.49811    5.62130
   0.30000    0.50000    1.40343   -1.40343    1.73090   -1.73090   -2.33905    2.33905    2.33905   -2.33905   10.35865    3.52194   12.77567    4.34373
   0.30000    0.55000    1.40343   -1.40343    1.73090   -1.73090   -2.57296    2.57296    2.57296   -2.57296   11.39452    2.48608   14.05324    3.06616
   0.30000    0.60000    1.40343   -1.40343    1.73090   -1.73090   -2.80686    2.80686    2.80686   -2.80686   12.43039    1.45021   15.33081    1.78859
   0.30000    0.65000    1.40343   -1.40343    1.73090   -1.73090   -3.04077    3.04077    3.04077   -3.04077   13.46625    0.41435   16.60838    0.51103
   0.35000    0.00000    1.63734   -1.63734    1.49699   -1.49699   -0.00000    0.00000    0.00000   -0.00000    0.00000   16.19403    0.00000   14.80597
   0.35000    0.05000    1.63734   -1.63734    1.49699   -1.49699   -0.23391    0.23391    0.23391   -0.23391    1.20851   14.98552    1.10492   13.70105
   0.35000    0.10000    1.63734   -1.63734    1.49699   -1.49699   -0.46781    0.46781    0.46781   -0.46781    2.41702   13.77701    2.20985   12.59612
   0.35000    0.15000    1.63734   -1.63734    1.49699   -1.49699   -0.70172    0.70172    0.70172   -0.70172    3.62553   12.56850    3.31477   11.49120
   0.35000    0.20000    1.63734   -1.63734    1.49699   -1.49699   -0.93562    0.93562    0.93562   -0.93562    4.83404   11.35999    4.41969   10.38628
   0.35000    0.25000    1.63734   -1.63734    1.49699   -1.49699   -1.16953    1.16953    1.16953   -1.16953    6.04255   10.15148    5.52462    9.28135
   0.35000    0.30000    1.63734   -1.63734    1.49699   -1.49699   -1.40343    1.40343    1.40343   -1.40343    7.25106    8.94297    6.62954    8.17643
   0.35000    0.35000    1.63734   -1.63734    1.49699   -1.49699   -1.63734    1.63734    1.63734   -1.63734    8.45957    7.73446    7.73446    7.07151
   0.35000    0.40000    1.63734   -1.63734    1.49699   -1.49699   -1.87124    1.87124    1.87124   -1.87124    9.66808    6.52595    8.83939    5.96658
   0.35000    0.45000    1.63734   -1.63734    1.49699   -1.49699   -2.10515    2.10515    2.10515   -2.10515   10.87659    5.31744    9.94431    4.86166
   0.35000    0.50000    1.63734   -1.63734    1.49699   -1.49699   -2.33905    2.33905    2.33905   -2.33905   12.08510    4.10893   11.04923    3.75674
   0.35000    0.55000    1.63734   -1.63734    1.49699   -1.49699   -2.57296    2.57296    2.57296   -2.57296   13.29361    2.90042   12.15415    2.65182
   0.35000    0.60000    1.63734   -1.63734    1.49699   -1.49699   -2.80686    2.80686    2.80686   -2.80686   14.50212    1.69191   13.25908    1.54689
   0.35000    0.65000    1.63734   -1.63734    1.49699   -1.49699   -3.04077    3.04077    3.04077   -3.04077   15.71063    0.48340   14.36400    0.44197
   0.40000    0.00000    1.87124   -1.87124    1.26309   -1.26309   -0.00000    0.00000    0.00000   -0.00000    0.00000   18.50746    0.00000   12.49254
   0.40000    0.05000    1.87124   -1.87124    1.26309   -1.26309   -0.23391    0.23391    0.23391   -0.23391    1.38115   17.12631    0.93228   11.56026
   0.40000    0.10000    1.87124   -1.87124    1.26309   -1.26309   -0.46781    0.46781    0.46781   -0.46781    2.76231   15.74515    1.86456   10.62798
   0.40000    0.15000    1.87124   -1.87124    1.26309   -1.26309   -0.70172    0.70172    0.70172   -0.70172    4.14346   14.36400    2.79684    9.69570
   0.40000    0.20000    1.87124   -1.87124    1.26309   -1.26309   -0.93562    0.93562    0.93562   -0.93562    5.52462   12.98285    3.72912    8.76342
   0.40000    0.25000    1.87124   -1.87124    1.26309   -1.26309   -1.16953    1.16953    1.16953   -1.16953    6.90577   11.60169    4.66139    7.83114
   0.40000    0.30000    1.87124   -1.87124    1.26309   -1.26309   -1.40343    1.40343    1.40343   -1.40343    8.28692   10.22054    5.59367    6.89886
   0.40000    0.35000    1.87124   -1.87124    1.26309   -1.26309   -1.63734    1.63734    1.63734   -1.63734    9.66808    8.83939    6.52595    5.96658
   0.40000    0.40000    1.87124   -1.87124    1.26309   -1.26309   -1.87124    1.87124    1.87124   -1.87124   11.04923    7.45823    7.45823    5.03431
   0.40000    0.45000    1.87124   -1.87124    1.26309   -1.26309   -2.10515    2.10515    2.10515   -2.10515   12.43039    6.07708    8.39051    4.10203
   0.40000    0.50000    1.87124   -1.87124    1.26309   -1.26309   -2.33905    2.33905    2.33905   -2.33905   13.81154    4.69592    9.32279    3.16975
   0.40000    0.55000    1.87124   -1.87124    1.26309   -1.26309   -2.57296    2.57296    2.57296   -2.57296   15.19269    3.31477   10.25507    2.23747
   0.40000    0.60000    1.87124   -1.87124    1.26309   -1.26309   -2.80686    2.80686    2.80686   -2.80686   16.57385    1.93362   11.18735    1.30519
   0.40000    0.65000    1.87124   -1.87124    1.26309   -1.26309   -3.04077    3.04077    3.04077   -3.04077   17.95500    0.55246   12.11963    0.37291
   0.45000    0.00000    2.10515   -2.10515    1.02918   -1.02918   -0.00000    0.00000    0.00000   -0.00000    0.00000   20.82090    0.00000   10.17910
   0.45000    0.05000    2.10515   -2.10515    1.02918   -1.02918   -0.23391    0.23391    0.23391   -0.23391    1.55380   19.26710    0.75963    9.41947
   0.45000    0.10000    2.10515   -2.10515    1.02918   -1.02918   -0.46781    0.46781    0.46781   -0.46781    3.10760   17.71330    1.51927    8.65984
   0.45000    0.15000    2.10515   -2.10515    1.02918   -1.02918   -0.70172    0.70172    0.70172   -0.70172    4.66139   16.15950    2.27890    7.90020
   0.45000    0.20000    2.10515   -2.10515    1.02918   -1.02918   -0.93562    0.93562    0.93562   -0.93562    6.21519   14.60570    3.03854    7.14057
   0.45000    0.25000    2.10515   -2.10515    1.02918   -1.02918   -1.16953    1.16953    1.16953   -1.16953    7.76899   13.05190    3.79817    6.38093
   0.45000    0.30000    2.10515   -2.10515    1.02918   -1.02918   -1.40343    1.40343    1.40343   -1.40343    9.32279   11.49811    4.55781    5.62130
   0.45000    0.35000    2.10515   -2.10515    1.02918   -1.02918   -1.63734    1.63734    1.63734   -1.63734   10.87659    9.94431    5.31744    4.86166
   0.45000    0.40000    2.10515   -2.10515    1.02918   -1.02918   -1.87124    1.87124    1.87124   -1.87124   12.43039    8.39051    6.07708    4.10203
   0.45000    0.45000    2.10515   -2.10515    1.02918   -1.02918   -2.10515    2.10515    2.10515   -2.10515   13.98418    6.83671    6.83671    3.34239
   0.45000    0.50000    2.10515   -2.10515    1.02918   -1.02918   -2.33905    2.33905    2.33905   -2.33905   15.53798    5.28291    7.59635    2.58276
   0.45000    0.55000    2.10515   -2.10515    1.02918   -1.02918   -2.57296    2.57296    2.57296   -2.57296   17.09178    3.72912    8.35598    1.82312
   0.45000    0.60000    2.10515   -2.10515    1.02918   -1.02918   -2.80686    2.80686    2.80686   -2.80686   18.64558    2.17532    9.11562    1.06349
   0.45000    0.65000    2.10515   -2.10515    1.02918   -1.02918   -3.04077    3.04077    3.04077   -3.04077   20.19938    0.62152    9.87525    0.30385
   0.50000    0.00000    2.33905   -2.33905    0.79528   -0.79528   -0.00000    0.00000    0.00000   -0.00000    0.00000   23.13433    0.00000    7.86567
   0.50000    0.05000    2.33905   -2.33905    0.79528   -0.79528   -0.23391    0.23391    0.23391   -0.23391    1.72644   21.40789    0.58699    7.27868
   0.50000    0.10000    2.33905   -2.33905    0.79528   -0.79528   -0.46781    0.46781    0.46781   -0.46781    3.45288   19.68144    1.17398    6.69169
   0.50000    0.15000    2.33905   -2.33905    0.79528   -0.79528   -0.70172    0.70172    0.70172   -0.70172    5.17933   17.95500    1.76097    6.10470
   0.50000    0.20000    2.33905   -2.33905    0.79528   -0.79528   -0.93562    0.93562    0.93562   -0.93562    6.90577   16.22856    2.34796    5.51771
   0.50000    0.25000    2.33905   -2.33905    0.79528   -0.79528   -1.16953    1.16953    1.16953   -1.16953    8.63221   14.50212    2.93495    4.93072
   0.50000    0.30000    2.33905   -2.33905    0.79528   -0.79528   -1.40343    1.40343    1.40343   -1.40343   10.35865   12.77567    3.52194    4.34373
   0.50000    0.35000    2.33905   -2.33905    0.79528   -0.79528   -1.63734    1.63734    1.63734   -1.63734   12.08510   11.04923    4.10893    3.75674
   0.50000    0.40000    2.33905   -2.33905    0.79528   -0.79528   -1.87124    1.87124    1.87124   -1.87124   13.81154    9.32279    4.69592    3.16975
   0.50000    0.45000    2.33905   -2.33905    0.79528   -0.79528   -2.10515    2.10515    2.10515   -2.10515   15.53798    7.59635    5.28291    2.58276
   0.50000    0.50000    2.33905   -2.33905    0.79528   -0.79528   -2.33905    2.33905    2.33905   -2.33905   17.26442    5.86990    5.86990    1.99577
   0.50000    0.55000    2.33905   -2.33905    0.79528   -0.79528   -2.57296    2.57296    2.57296   -2.57296   18.99087    4.14346    6.45689    1.40878
   0.50000    0.60000    2.33905   -2.33905    0.79528   -0.79528   -2.80686    2.80686    2.80686   -2.80686   20.71731    2.41702    7.04389    0.82179
   0.50000    0.65000    2.33905   -2.33905    0.79528   -0.79528   -3.04077    3.04077    3.04077   -3.04077   22.44375    0.69058    7.63088    0.23480
   0.55000    0.00000    2.57296   -2.57296    0.56137   -0.56137   -0.00000    0.00000    0.00000   -0.00000    0.00000   25.44776    0.00000    5.55224
   0.55000    0.05000    2.57296   -2.57296    0.56137   -0.56137   -0.23391    0.23391    0.23391   -0.23391    1.89909   23.54867    0.41435    5.13789
   0.55000    0.10000    2.57296   -2.57296    0.56137   -0.56137   -0.46781    0.46781    0.46781   -0.46781    3.79817   21.64959    0.82869    4.72355
   0.55000    0.15000    2.57296   -2.57296    0.56137   -0.56137   -0.70172    0.70172    0.70172   -0.70172    5.69726   19.75050    1.24304    4.30920
   0.55000    0.20000    2.57296   -2.57296    0.56137   -0.56137   -0.93562    0.93562    0.93562   -0.93562    7.59635   17.85141    1.65738    3.89485
   0.55000    0.25000    2.57296   -2.57296    0.56137   -0.56137   -1.16953    1.16953    1.16953   -1.16953    9.49543   15.95233    2.07173    3.48051
   0.55000    0.30000    2.57296   -2.57296    0.56137   -0.56137   -1.40343    1.40343    1.40343   -1.40343   11.39452   14.05324    2.48608    3.06616
   0.55000    0.35000    2.57296   -2.57296    0.56137   -0.56137   -1.63734    1.63734    1.63734   -1.63734   13.29361   12.15415    2.90042    2.65182
   0.55000    0.40000    2.57296   -2.57296    0.56137   -0.56137   -1.87124    1.87124    1.87124   -1.87124   15.19269   10.25507    3.31477    2.23747
   0.55000    0.45000    2.57296   -2.57296    0.56137   -0.56137   -2.10515    2.10515    2.10515   -2.10515   17.09178    8.35598    3.72912    1.82312
   0.55000    0.50000    2.57296   -2.57296    0.56137   -0.56137   -2.33905    2.33905    2.33905   -2.33905   18.99087    6.45689    4.14346    1.40878
   0.55000    0.55000    2.57296   -2.57296    0.56137   -0.56137   -2.57296    2.57296    2.57296   -2.57296   20.88995    4.55781    4.55781    0.99443
   0.55000    0.60000    2.57296   -2.57296    0.56137   -0.56137   -2.80686    2.80686    2.80686   -2.80686   22.78904    2.65872    4.97215    0.58008
   0.55000    0.65000    2.57296   -2.57296    0.56137   -0.56137   -3.04077    3.04077    3.04077   -3.04077   24.68813    0.75963    5.38650    0.16574
   0.60000    0.00000    2.80686   -2.80686    0.32747   -0.32747   -0.00000    0.00000    0.00000   -0.00000    0.00000   27.76119    0.00000    3.23881
   0.60000    0.05000    2.80686   -2.80686    0.32747   -0.32747   -0.23391    0.23391    0.23391   -0.23391    2.07173   25.68946    0.24170    2.99710
   0.60000    0.10000    2.80686   -2.80686    0.32747   -0.32747   -0.46781    0.46781    0.46781   -0.46781    4.14346   23.61773    0.48340    2.75540
   0.60000    0.15000    2.80686   -2.80686    0.32747   -0.32747   -0.70172    0.70172    0.70172   -0.70172    6.21519   21.54600    0.72511    2.51370
   0.60000    0.20000    2.80686   -2.80686    0.32747   -0.32747   -0.93562    0.93562    0.93562   -0.93562    8.28692   19.47427    0.96681    2.27200
   0.60000    0.25000    2.80686   -2.80686    0.32747   -0.32747   -1.16953    1.16953    1.16953   -1.16953   10.35865   17.40254    1.20851    2.03030
   0.60000    0.30000    2.80686   -2.80686    0.32747   -0.32747   -1.40343    1.40343    1.40343   -1.40343   12.43039   15.33081    1.45021    1.78859
   0.60000    0.35000    2.80686   -2.80686    0.32747   -0.32747   -1.63734    1.63734    1.63734   -1.63734   14.50212   13.25908    1.69191    1.54689
   0.60000    0.40000    2.80686   -2.80686    0.32747   -0.32747   -1.87124    1.87124    1.87124   -1.87124   16.57385   11.18735    1.93362    1.30519
   0.60000    0.45000    2.80686   -2.80686    0.32747   -0.32747   -2.10515    2.10515    2.10515   -2.10515   18.64558    9.11562    2.17532    1.06349
   0.60000    0.50000    2.80686   -2.80686    0.32747   -0.32747   -2.33905    2.33905    2.33905   -2.33905   20.71731    7.04389    2.41702    0.82179
   0.60000    0.55000    2.80686   -2.80686    0.32747   -0.32747   -2.57296    2.57296    2.57296   -2.57296   22.78904    4.97215    2.65872    0.58008
   0.60000    0.60000    2.80686   -2.80686    0.32747   -0.32747   -2.80686    2.80686    2.80686   -2.80686   24.86077    2.90042    2.90042    0.33838
   0.60000    0.65000    2.80686   -2.80686    0.32747   -0.32747   -3.04077    3.04077    3.04077   -3.04077   26.93250    0.82869    3.14213    0.09668
   0.65000    0.00000    3.04077   -3.04077    0.09356   -0.09356   -0.00000    0.00000    0.00000   -0.00000    0.00000   30.07463    0.00000    0.92537
   0.65000    0.05000    3.04077   -3.04077    0.09356   -0.09356   -0.23391    0.23391    0.23391   -0.23391    2.24438   27.83025    0.06906    0.85632
   0.65000    0.10000    3.04077   -3.04077    0.09356   -0.09356   -0.46781    0.46781    0.46781   -0.46781    4.48875   25.58588    0.13812    0.78726
   0.65000    0.15000    3.04077   -3.04077    0.09356   -0.09356   -0.70172    0.70172    0.70172   -0.70172    6.73313   23.34150    0.20717    0.71820
   0.65000    0.20000    3.04077   -3.04077    0.09356   -0.09356   -0.93562    0.93562    0.93562   -0.93562    8.97750   21.09713    0.27623    0.64914
   0.65000    0.25000    3.04077   -3.04077    0.09356   -0.09356   -1.16953    1.16953    1.16953   -1.16953   11.22188   18.85275    0.34529    0.58008
   0.65000    0.30000    3.04077   -3.04077    0.09356   -0.09356   -1.40343    1.40343    1.40343   -1.40343   13.46625   16.60838    0.41435    0.51103
   0.65000    0.35000    3.04077   -3.04077    0.09356   -0.09356   -1.63734    1.63734    1.63734   -1.63734   15.71063   14.36400    0.48340    0.44197
   0.65000    0.40000    3.04077   -3.04077    0.09356   -0.09356   -1.87124    1.87124    1.87124   -1.87124   17.95500   12.11963    0.55246    0.37291
   0.65000    0.45000    3.04077   -3.04077    0.09356   -0.09356   -2.10515    2.10515    2.10515   -2.10515   20.19938    9.87525    0.62152    0.30385
   0.65000    0.50000    3.04077   -3.04077    0.09356   -0.09356   -2.33905    2.33905    2.33905   -2.33905   22.44375    7.63088    0.69058    0.23480
   0.65000    0.55000    3.04077   -3.04077    0.09356   -0.09356   -2.57296    2.57296    2.57296   -2.57296   24.68813    5.38650    0.75963    0.16574
   0.65000    0.60000    3.04077   -3.04077    0.09356   -0.09356   -2.80686    2.80686    2.80686   -2.80686   26.93250    3.14213    0.82869    0.09668
   0.65000    0.65000    3.04077   -3.04077    0.09356   -0.09356   -3.04077    3.04077    3.04077   -3.04077   29.17688    0.89775    0.89775    0.02762
\end{filecontents}
\pgfplotstableread{MerolegesNyomoeroCalc.dat}{\pistonkinetics}

\begin{tikzpicture}
\pgfplotsset{every axis plot/.append style={very thick}}
\setcaptionsubtype

\begin{groupplot}[%
            ,group style={%
                ,group name=my plots
                ,group size=2 by 2
                ,vertical sep=2cm,
                ,horizontal sep = 2cm,
                ,ylabels at=edge left
            }
            ,width=7cm
            ,height=6cm
            ,try min ticks=5
            ,xlabel={\bfseries{\emph{m}}}
            ,ylabel={\bfseries{\emph{m}}}
            ,zlabel={\bfseries{\emph{kg}}}
            ,grid=both
            ,every major grid/.style={gray, opacity=0.5},
            view={0}{90},
            %xmin=0,xmax=0.65,
            %ymin=0,ymax=0.65,
            zmin=-5,zmax=60,
            ]

\nextgroupplot%
\addplot3 [contour gnuplot={number=7},
        thick,mesh/rows=196, mesh/cols=14,mesh/check=false ] table [header=true, x = x, y=y, z=F_zFR] {\pistonkinetics};
\label{plots:InstC}

\nextgroupplot%
\addplot3 [contour gnuplot={number=7},
        thick,mesh/rows=196, mesh/cols=14, mesh/check=false,contour/label distance=50pt] table [header=true, x = x, y=y, z=F_zFL] {\pistonkinetics};

\nextgroupplot%
\addplot3 [contour gnuplot={number=1},
        thick,mesh/rows=196, mesh/cols=14,mesh/check=false ] table [header=true, x = x, y=y, z=F_zBR] {\pistonkinetics};
 
\nextgroupplot%       
\addplot3 [contour gnuplot={number=14},
        thick,mesh/rows=196, mesh/cols=14,mesh/check=false ] table [header=true, x = x, y=y, z=F_zBL] {\pistonkinetics};


\end{groupplot}

%\path [nodes={anchor=south,rotate=90,font=\large\bfseries,midway}]
%  (my plots c1r1.outer north west)--(my plots c1r2.outer south west)
%    node {Testing of Parameters 1}
%  (my plots c2r1.outer north west)--(my plots c2r2.outer south west)
%    node {Testing of Parameters 2};

% legend



\node[text width=.5\linewidth,align=center,anchor=south] at (my plots c1r1.north) {\caption[]{FL\label{subplot:one}}};
\node[text width=.5\linewidth,align=center,anchor=south] at (my plots c2r1.north) {\caption[]{FR\label{subplot:two}}};
\node[text width=.5\linewidth,align=center,anchor=south] at (my plots c1r2.north) {\caption[]{BL\label{subplot:three}}};
\node[text width=.5\linewidth,align=center,anchor=south] at (my plots c2r2.north) {\caption[]{BR\label{subplot:four}}};
\end{tikzpicture}
\caption[]{$SSMR-4W$ tipusu robot kereknyomoerok kerekenkeni változása a sulypont fuggvenyeben}
\label{fig:NyomoeroSzim4Wheel}

\end{figure}
\end{kep}




\subsection{Súlypont meghatározása mérésekkel}

A robot súlypont meghatározása egy mérleg segítségéével lemérve sorra minden kerék merőleges nyomóerőjét a talajra nézve. 
A mért adatok vízszintes helyzetben:

\begin{table}[H]
\center
\begin{tabular}{lll}
Node  & Nyomó erő & Mértékegység \\
FL &   11,8      & kg          \\
FR &   13,2      & kg          \\
BL &   17,1      & kg          \\
BR &   17,9      & kg            
\end{tabular}
\end{table}

A súlypont pozíciója: $b = 20$ és $c = 30 $

\subsection{Kerék Dinamikája}
Az $I_w \in \mathbb{R}^4$ tartalmazza a kerekek inerciáját a forgás tengelyükhöz képest. $\Omega \in \mathbb{R}^4$ a kerekek szögsebessége. A $W_r \in \mathbb{R}^4$ a kerekek sugara, $\tau \in \mathbb{R}^4$ a kerekek forgatónyomatéka.


\begin{equation}
I_w\dot\Omega =
\tau-W_rF_x
\end{equation}

\begin{equation*}
 I_w =   \begin{bmatrix}
I_{FL} & 0 & 0 & \\ 
 0 & I_{BL}  & 0 & 0 \\ 
 0 &  0 & I_{FR} & 0\\ 
 0 &  0 &  0 & I_{BR}
\end{bmatrix}
,\quad
W_r=\begin{bmatrix}
r_{FL} & 0 & 0 & 0\\ 
0 & r_{BL} & 0 &0 \\ 
0 & 0 & r_{FR} & 0\\ 
0 & 0 & 0 & r_{BR}
\end{bmatrix}
\end{equation*}

\begin{equation*}
\tau = \begin{bmatrix}
\tau _{FL}&& 
\tau _{BL}&& 
\tau _{FR}&& 
\tau _{BR}
\end{bmatrix}^T
,\quad
\Omega =\begin{bmatrix}
\omega _ {FL}&& 
\omega _ {BL}&& 
\omega _ {FR}&& 
\omega _ {BR}
\end{bmatrix}^T
\end{equation*}







\section{Kinematikai Modell} 

A \ref{fig:SMR4WKinematics} látható a $4W-SSMR$ kinematikai modellje. Néhány feltételezés : a robot minden kereke mindig érintkezik a talajjal, a kerekek nem csúsznak forgásuk közben, külön van kezelve a laterális és a longitudinális súrlódás, a robot egy tömeg központtal van jellemezve, az alacsony szintű szabályzok tökéletesen követik az előírt referenciát.

A robot a $ICR$ pont körül fordul, és csak a robothoz rendelt vonatkoztatási rendszer x tengelye menten tud elmozdulni. Az y irányú sebességeket azt okozza hogy a jobb és bal oldali kerekek forgási sebessége eltér és így létrejönne egy oldal irányú csuszás. 

Jelölje a rendre a $K_{ik}$ a kerekek a talajjal való érintkezési pontját, $^RV_{ik}$ a $K_{ik}$ pontok sebességét a robothoz rendelt $VNR$-ben, $^RV_{ikX}$ és $^RV_{ikY}$ rendre a $^RV_{ik}$ sebesség X és Y komponense robothoz rendelt $VNR$-ben. A $^RV_{ikX}$ megfelel a kerekek kerületei sebességének. A $^RV^{COM}_{ik}$ a $COM$ pont sebességet a robothoz rendelt $VNR$-ben, illetve a $^RV^{COM}_{ikX}$ és $^RV^{COM}_{ikY}$ az X és Y komponense. 

A robot és a globális $VNR$ x tengelye között bezárt szög $\theta$ valamint $X$ és $Y$ a robot pozíciója a $O$ ponthoz viszonyítva.

Az $ICR$ pont helyzete a $^RV_{ik}$ sebesség vektorokra merőleges egyenesek metszés pontjában található és mindig a robothoz rendelt $VNR$ $Y$ tengelyen helyezkedik el.

\renewcommand{\img}{SajatRobot/SzerkAbrak/robot4wSebModel_seb.jpg}
\renewcommand{\sources}{*}
\renewcommand{\captionn}{Kinematikai modell az $4W-SSMR$ típusú robotnak.}
\renewcommand{\figlabel}{SMR4WKinematics}
\begin{kep}
\begin{figure}[H]
\centering
\ifthenelse{\equal{\svg}{*}}
{
    \includegraphics[width=\aspectratioPic\textwidth,angle=\rotationAnglePic]{\img}
}
{
    \includesvg[width=\aspectratioPic\textwidth,angle=\rotationAnglePic]{\img}
}

 \ifthenelse{\equal{\sources}{*}}
    { \captionof{figure}{ \captionn}}
    { \captionof{figure}{ \mand{\mand{\captionn}{Forrás:}}{}} }
  	

\ifthenelse{\equal{\figlabel}{*}}
    {}
    {\label{fig:\figlabel}}%
    
\renewcommand{\figlabel}{*}



\end{figure}
\end{kep}
\renewcommand{\aspectratioPic}{1}
\renewcommand{\rotationAnglePic}{0}
\renewcommand{\svg}{*}


A $\dot q$ a $4W-SSMR$ síkban modellezett állapot vektora a globális $VNR$-ben. $^R \omega^{COG}$ az $COG$ pont körüli szögsebesség a robothoz rendelt $VNR$-ben. Az $\eta$ jelölje a bemeneti értékeket.
A $d$ a $COG$ és a $COM$ pontok közti távolság.

A $COM$ pontban mert értékek az \myeqref{eq:allapot} segítségével számolhatjuk a globális $VNR$ -be.
A $COM$ pont sebességének y komponense megadható az \myeqref{eq:SebComY} segítségével.
A nemholomonikus megkötés \myeqref{eq:nemholomonikusmegkotes} biztosítja azt hogy a robot nem tud oldal irányú mozgást végezni.

\begin{equation}
\label{eq:allapot}
\dot{q} = 
\begin{bmatrix}
\dot{X}\\ 
\dot{Y}\\
\dot{\theta}
\end{bmatrix}= \begin{bmatrix}
\cos \theta  & -\sin \theta & 0 \\
\sin\theta & \cos \theta &  0\\ 
 0 & 0  & 1
\end{bmatrix}
\begin{bmatrix}
^RV^{COM}_x\\ 
^RV^{COM}_y\\ 
^R\omega^{COG}
\end{bmatrix}
\end{equation}

\begin{equation}
\label{eq:SebComY}
^RV^{COM}_y = d\omega
\end{equation}

\begin{equation}
\label{eq:nemholomonikusmegkotes}
\begin{bmatrix}
-\sin\theta  & -\cos\theta  & -d
\end{bmatrix}
\begin{bmatrix}
\dot{X}\\ 
\dot{Y}\\
\dot{\theta}
\end{bmatrix} = A(q) \dot{q} =0
\end{equation}

\begin{equation}
    \label{eq:allapotegyszeru}
    \dot q = S(q)\eta 
\end{equation}

\begin{equation*}
S(q)=\begin{bmatrix}
\cos \theta  & -d\sin \theta  \\
\sin\theta & d\cos \theta \\ 
 0  & 1
\end{bmatrix},
\eta=\begin{bmatrix}
^RV^{COM}_x\\ 
^R\omega^{COG}
\end{bmatrix}
\end{equation*}

\begin{equation}
\label{eq:SxAeqZero}
    S^TA^T=0
\end{equation}







\section{Dinamikai Modell} 

A \ref{fig:SMR4WDinamicsModel} látható a $4W-SSMR$-ra ható erők rendszerre. Jelölje a $F_{ik}$ a $K_{ik}$ pontokban a kerekek a talajra kifejtett erőt, $F_{fxik}$ és a $F_{fyik}$ rendre az x és y irányba ható súrlódási erőket a $K_{ik}$ pontokban.

Az \myeqrefinterval{eq:eromodelglobal1}{eq:eromodelglobal3} leírják a robot mozgását a globális rendszerben, felhasználva a robot $VNR$ - ben mert erő hatásokat.

\renewcommand{\img}{SajatRobot/SzerkAbrak/robot4wDinamic.jpg}
\renewcommand{\sources}{*}
\renewcommand{\captionn}{Kinematikai modell az $SSMR$ típusú $MR$ robotnak.}
\renewcommand{\figlabel}{SMR4WDinamicsModel}
\begin{kep}
\begin{figure}[H]
\centering
\ifthenelse{\equal{\svg}{*}}
{
    \includegraphics[width=\aspectratioPic\textwidth,angle=\rotationAnglePic]{\img}
}
{
    \includesvg[width=\aspectratioPic\textwidth,angle=\rotationAnglePic]{\img}
}

 \ifthenelse{\equal{\sources}{*}}
    { \captionof{figure}{ \captionn}}
    { \captionof{figure}{ \mand{\mand{\captionn}{Forrás:}}{}} }
  	

\ifthenelse{\equal{\figlabel}{*}}
    {}
    {\label{fig:\figlabel}}%
    
\renewcommand{\figlabel}{*}



\end{figure}
\end{kep}
\renewcommand{\aspectratioPic}{1}
\renewcommand{\rotationAnglePic}{0}
\renewcommand{\svg}{*}


Az $F_x \in \mathbb{R}^4$ tartalmazza $F_{ik}$ kerekek által a talajra kifejtett erőket. Az $F_{sx} \in \mathbb{R}^4$ és $F_{sy} \in \mathbb{R}^4$ súrlódási erők x és y tengely menten a robot $VNR$-ben.A $F \in \mathbb{R}^2$ tartalmazza a jobb és bal oldali kerek által a talajra kifejtett erők összegét. Jelölje $I$ a robot inerciáját a z tengely körül, $M_a$ nyomatékok összege amelyeket a kerekek hoznak létre, $M_r$ a nyomatékok összege amelyeket a súrlódások hoznak létre. 

A $K_x\in \mathbb{R}^4$ és $K_y\in \mathbb{R}^4$ jelölje a súlypont pozíciója a kerek és talaj érintkezési pontokhoz viszonyítva a \ref{fig:SMR4WMerolegesNyomoero} alapján.

Az $N \in \mathbb{R}^4$ tartalmazza a merőleges nyomóerőket talajra nézve, a $K_{ik}$ pontokban. 

\begin{equation}
\label{eq:eromodelglobal1}
m\ddot{X} = \xi F_x\cos \theta -\xi F_{sx}\cos \theta + \xi F_{sy} \sin\theta
\end{equation}

\begin{equation}
\label{eq:eromodelglobal2}
m\ddot{Y} = \xi F_x\sin \theta -\xi F_{sx}\sin \theta - \xi F_{sy} \cos \theta
\end{equation}

\begin{equation}
\label{eq:eromodelglobal3}
I\ddot{\theta} = M_{a}+ M_{r},
\end{equation}

\begin{equation*}
\xi = \begin{bmatrix}
1 & 1 & 1 & 1
\end{bmatrix}
\end{equation*}

\begin{equation*}
F_{sx} = \begin{bmatrix}
F_{sxFL} & F_{sxBL} & F_{sxFR} & F_{sxBR}
\end{bmatrix}^T
, \quad
F_{sy} = \begin{bmatrix}
F_{syFL} & F_{syBL} & F_{syFR} & F_{syBR}
\end{bmatrix}^T
\end{equation*}

\begin{equation*}
    F_{sxik}=N_{ik} \mu_{xik} S_{xik}
    ,\quad
    F_{sy}=N_{ik} \mu_{yik} S_{y{ik}}
\end{equation*}

\begin{equation*}
\mu _x = \begin{bmatrix}
\mu _{xFL} & 
\mu _{xBL} & 
\mu _{xFR} & 
\mu _{xBR}
\end{bmatrix}^T
,\quad
\mu _y = \begin{bmatrix}
\mu _{yFL}& 
\mu _{yBL}& 
\mu _{yFR}&
\mu _{yBR}
\end{bmatrix}^T
\end{equation*}

\begin{equation*}
S_x=\begin{bmatrix}
sgn(V_{xFL})&& 
sgn(V_{xBL})&& 
sgn(V_{xFR})&& 
sgn(V_{xBR})
\end{bmatrix}^T
\end{equation*}

\begin{equation*}
S_y=\begin{bmatrix}
sgn(V_{yFL})&& 
sgn(V_{yBL})&& 
sgn(V_{yFR})&& 
sgn(V_{yBR})
\end{bmatrix}^T
\end{equation*}

\begin{equation}
M_r=M_{rx}+M_{ry}
\end{equation}

\begin{equation}
M_{rx}=K^T_xF_{sx},M_{ry}=K^T_yF_{sy}, M_a = K^T_xF_x
\end{equation}

\begin{equation*}
K_x =\begin{bmatrix}
a & a & b & b
\end{bmatrix}^T
,\quad
K_y =\begin{bmatrix}
c & d & c & d
\end{bmatrix}^T
\end{equation*}

\begin{equation*}
F_x= \begin{bmatrix}
F_{FL}&& 
F_{BL}&& 
F_{FR}&& 
F_{BR}
\end{bmatrix}^T
\end{equation*}

\begin{equation*}
    F = \begin{bmatrix}
    F_{FL}+F_{BL} &&
    F_{FR}+F_{BR}
    \end{bmatrix}^T
\end{equation*}



Általános formában a $4W-SSMR$ dinamikai modellje a \myeqref{eq:dinamiceqgeneral} adható meg a \cite{RobustMotionControl} alapján. Jelölje $M \in \mathbb{R}^{3x3}$ a tömegek és inerciák mátrixa, $R \in \mathbb{R}^{3}$ ellenálló nyomatékok és erők mátrixa,  $B \in \mathbb{R}^{3x2}$ bemeneti mátrix, $A$ a megkötések vektora \myeqref{eq:nemholomonikusmegkotes} alapján, $\lambda$ Lagrange együtthatók vektora. $F_d \in \mathbb{R}^{3}$ zajok vektora.

A \myeqref{eq:dinamiceqgeneral}  az állapotok gyorsulását megkapjuk ha az \myeqref{eq:allapotegyszeru} időben deriváljuk, így az \myeqref{eq:allapotokdupladerivalt}-t kapjuk.

Felhasználva a  \myeqref{eq:SxAeqZero} és \myeqref{eq:allapotokdupladerivalt} és \myeqref{eq:allapotegyszeru} a \myeqref{eq:dinamiceqgeneral} egyenletet egyszerűbb alakra írhatjuk azáltal hogy minden tagot beszorzunk balról $S^T$ -vel, így a \myeqref{eq:dinamicmodelgenericsimplified} kapjuk.


\begin{equation}
\label{eq:dinamiceqgeneral}
    M(q)\ddot q+
R(\dot q)+F_d = B(q)F + A^T\lambda 
\end{equation}

\begin{equation*}
M=\begin{bmatrix}
m & 0 & 0\\ 
0 & m & 0\\ 
0 & 0 & I
\end{bmatrix},\quad
B(q)=\begin{bmatrix}
\cos\theta  & \cos\theta \\
\sin\theta  & \sin\theta \\
-a &  b
\end{bmatrix},\quad
D=\begin{bmatrix}
1 & 1 & 0 & 0\\ 
0 & 0 & 1 & 1
\end{bmatrix}^T, \quad
R(\dot q)=\begin{bmatrix}
\xi F_{sx}\cos\theta-\xi F_{sy}\sin\theta \\ 
\xi F_{sx}\sin\theta-\xi F_{sy}\cos\theta\\ 
M_r
\end{bmatrix}
\end{equation*}

\begin{equation}
\label{eq:allapotokdupladerivalt}
    \ddot q = S(q) \dot\eta + \dot S(q)\eta 
\end{equation}

\begin{equation}
\label{eq:dinamicmodelgenericsimplified}
\bar{M}\dot\eta+\bar{C}\eta+\bar{R}+\bar{F_d}=\bar{B}F
\end{equation}

\begin{equation*}
\bar{M}=S^TMS, \quad
\bar{C}= S^TM \dot S, \quad
\bar{R}=S^T\dot R,  \quad
\bar{F_d}=S^TF_d,\quad
\bar{B}=S^TB 
\end{equation*}



\section{Robot Platform Sebesség Szabályzása}


\subsection{Előírt nyomatékkal}

A kerekek előírt nyomatékát megkapjuk ha a  \myeqref{eq:eloirtnyomatek} -t használva. Az $u$ szabályzó jelet kiszámíthatjuk ha az \myeqref{eq:controllertorque1}-t használjuk. Jelölje a $K_\eta$ a szabályzó paraméterei. Csuszás szabályzó  $\sigma_\eta$ paraméteri $\rho_v$ lineáris sebességért, és $\rho_w$ szögsebességért felelős. Mindkét paraméter nagyobb kell legyen mint a zaj $n$ megfelelő érteké.

\renewcommand{\img}{SajatRobot/SzerkAbrak/SebContRefNyom.tex}
\renewcommand{\sources}{*}
\renewcommand{\captionn}{Kinematikai modell az $SSMR$ típusú $MR$ robotnak.}
\renewcommand{\figlabel}{DinamicSpeedController}
\begin{kep}
\begin{figure}[H]
\centering
\input{\img}
 \ifthenelse{\equal{\sources}{*}}
    { \captionof{figure}{ \captionn}}
    { \captionof{figure}{ \captionn Forrás: \sources }}%
 

\ifthenelse{\equal{\figlabel}{*}}
    {}
    {\label{fig:\figlabel}}%
    
\renewcommand{\figlabel}{*}

\end{figure}
\end{kep}



\begin{equation}
\label{eq:eloirtnyomatek}
\tau = W_rD\overline{B}\begin{bmatrix}
\overline{M}u+\overline{C}\eta +\overline{R}
\end{bmatrix} + I_w \dot\Omega 
\end{equation}

\begin{equation}
    \label{eq:controllertorque1}
    u = \dot \eta_d + K_\eta e_\eta  + \sigma_\eta  
\end{equation}

\begin{equation}
    e_\eta = \eta _d - \eta
\end{equation}

\begin{equation}
    e_\eta = \begin{bmatrix}
e_v & e_w 
\end{bmatrix}^T \in \mathbb{R}^2
,\quad
\sigma_\eta = \begin{bmatrix}
\sigma_v & \sigma_w 
\end{bmatrix}^T \in \mathbb{R}^2
,\quad
\sigma_v = \rho _v sgn(e_v)
, \quad
\sigma_w = \rho _w sgn(e_w)
\end{equation}



\subsubsection{Mesterséges Erő Módszere}
A \cite{SSMRartificialForceMethod} cikk alapján egy másik megközelítést használva modellezi a robot. A $q$ állapotokat meg kiegészíti a jobb és bal oldali kereke szögsebességével. Feltételezi hogy a kerekek sugara $r$ mind a négy kereknél egyenlő, és a $COM$ pont a robot szimmetriatengelyén helyezkedik el. Jelölje $F$ a ellenálló erők és nyomatékok vektora. Hasonlóképen az  \myeqref{eq:dinamiceqgeneral} -hez a Lagrange egyenletet ír fel a dinamikai modellre. Az $\eta$ tartalmaz az előírt sebességek vektora, az $\eta_{3}$ a sebességek vektora, a 3-dik eleme tartalmazza a generált sebességet amelyet úgy kell előírnunk hogy a hozza tartozó előírt kerek sebesség nulla legyen.
Az $u \in \mathbb{R}^3$ a jobb és bal oldali kerekek előírt erőleadása a talajra.

\begin{equation}
A(q)\dot q = \begin{bmatrix}
\cos\theta  & \sin\theta & -c & -r &  0\\ 
\cos\theta & \sin\theta & c & 0 & -r
\end{bmatrix}
\begin{bmatrix}
\dot X\\ 
\dot Y\\ 
\dot \theta\\ 
\Omega_L\\ 
\Omega_R
\end{bmatrix}
\end{equation}

\begin{equation*}
M(q)=\begin{bmatrix}
m & 0 & 0 & 0 & 0\\ 
0 & m & 0 & 0 & 0\\ 
0 & 0 & I & 0 & 0\\ 
0 & 0 & 0 & I_{FL}+I_{BL} & 0\\ 
0 & 0 & 0 & 0 & I_{FR}+I_{BR}
\end{bmatrix}
,\quad
B=\begin{bmatrix}
0 & 0\\ 
0 & 0\\ 
0 & 0\\ 
1 & 0\\ 
0 & 1
\end{bmatrix}
\end{equation*}

\begin{equation}
F(q,\dot q) = \begin{bmatrix}
F_x \cos \theta - F_y \sin \theta & F_x \sin \theta - F_y \cos \theta & M_r & 0 & 0
\end{bmatrix}^T
\end{equation}

\begin{equation}
\dot q = G_e(q)\eta = \begin{bmatrix}
\cos \theta & \cos \theta & -\sin \theta\\ 
\sin \theta & \sin \theta & \cos \theta\\ 
\frac{1}{c} & -\frac{1}{c}  & 0\\ 
0 & \frac{2}{r} & 0\\ 
\frac{2}{r} & 0 & 0
\end{bmatrix}
\begin{pmatrix}
\eta_1 \\ 
\eta_2\\ 
\eta_3
\end{pmatrix}
\end{equation}

\begin{equation}
\underset{M^*}{\underbrace{G^T_eMG_e}}\dot\eta + \underset{C^*}{\underbrace{G^T_e(M\dot G_e + CG_e)}}\dot \eta
 + \underset{F^*}{\underbrace{G^T_eF}} = \underset{B^*}{\underbrace{G^T_eB}}u
\end{equation}

\begin{equation*}
M^*(q)=\begin{bmatrix}
m+\frac{I}{c}+4\frac{I_k}{r^2} & m-\frac{I}{c^2}  & 0 \\ 
m-\frac{I}{c^2} & m+\frac{I}{c^2} + 4\frac{I_k}{r^2}& 0\\ 
0 & 0 & m
\end{bmatrix}
,\quad
F^*(q,\dot q)=\begin{bmatrix}
-F_x-\frac{M_r}{c} && - F_x+\frac{M_r}{c} && -F_y
\end{bmatrix}^T
\end{equation*}

\begin{equation*}
C^*(q)=\begin{bmatrix}
0 & 0 & -m\dot \theta \\ 
0 & 0 & -m\dot \theta \\
m\dot \theta & m\dot \theta & 0
\end{bmatrix}
\end{equation*}

\begin{equation*}
B^*=\begin{bmatrix}
\frac{2}{r} & 0 & 0\\ 
0 & \frac{2}{r} & 0\\ 
0 & 0 & 1
\end{bmatrix}
,\quad
(B^*)^{-1}=\begin{bmatrix}
\frac{r}{2} & 0 & 0\\ 
0 & \frac{r}{2} & 0\\ 
0 & 0 & 1
\end{bmatrix}
,\quad
\eta_r=\begin{pmatrix}
\eta_{r_1}\\ 
\eta_{r_2}\\ 
\eta_{r_3}
\end{pmatrix}
\end{equation*}


\begin{equation}
u=(B^*)^{-1}
\begin{Bmatrix}
M^*\dot\eta_r + C^*\eta_r + F^* -K_de
\end{Bmatrix}
\end{equation}



\begin{equation}
    \tau =W_rD[u_1, u_2]^T
\end{equation}

\begin{equation}
e_{\eta_{i}} = \eta_{i} - \eta_{r_{i}}
\end{equation}

\begin{equation}
\dot\eta_{r_{3}} = \frac{m\dot\theta(\eta_{r_{1}}+\eta_{r_{2}})-F_y-K_d(\eta_3-\eta_{r_3})}{m}
\end{equation}

\begin{equation*}
\eta_{r_1} = \frac{^RV^{COM}_x- ^R\omega^{COG}_rc}{2}
,\quad
\eta_{r_2} = \frac{^RV^{COM}_x+ ^R\omega^{COG}_rc}{2} 
\end{equation*}





\subsection{Elirt kerekszogsebessegekkel}
A \cite{Campa2014} cikkben a kerekek sebességét szabályozza, A jobb és bal oldali kerekek modelljét ARX becsléssel meghatározza a matematikai modellt és pólusáthelyezéses  módszerrel a kívánt modellt állítja elő. 

\renewcommand{\img}{SajatRobot/SzerkAbrak/SebContRefOmega.tex}
\renewcommand{\sources}{*}
\renewcommand{\captionn}{Kinematikai modell az $SSMR$ típusú $MR$ robotnak.}
\renewcommand{\figlabel}{DinamicSpeedController}
\begin{kep}
\begin{figure}[H]
\centering
\input{\img}
 \ifthenelse{\equal{\sources}{*}}
    { \captionof{figure}{ \captionn}}
    { \captionof{figure}{ \captionn Forrás: \sources }}%
 

\ifthenelse{\equal{\figlabel}{*}}
    {}
    {\label{fig:\figlabel}}%
    
\renewcommand{\figlabel}{*}

\end{figure}
\end{kep}



