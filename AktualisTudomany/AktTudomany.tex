\FloatBarrier

\chapter{Bibliográfiai tanulmány}

\section{Robotok}
A olyan gepek amelyeket arra terveztek hogy bizonyos feladatokat automatikusan elvégezzenek gyorsabban és pontosabban mit az emberek. Manapsag mar sok tipusu robot letezik, ezekozul nehany repul, foldon gurul vagy maszik, de leteznek mar emberhez hasonlok is.

A robotok legeloszor a masodik vilaghaboru alatt jelentek meg mint iranyitott bombak. A haboru utan W.Grey Walter neurologus fejlesztett ki egy kismeretu robotot (Elmer and Elsie) mely fenyszenzorral, es nyomasszenzorral volt elatva. 1961-1963 Johns Hopkins Beast robot kepes volt a fal menten végigmenni és megtalalni a toltoalomast. 1970 -ben Shakey the robot kepes volt kapera segitsegevel egy vonalat kovetni. A 1990 utan a fejlesztesek felgyorsultak es azota mar robot jar a naprendszer mas bolygojan is.

Ezen dolgozat csak a $MR$ típusú robotokkal fog reszletesebben kiterni. A $MR$ tipusu robotok kepesek a sajat poziciojukat megvaltoztani a kornyezetukhoz kepest. Az $AMR$ tipusu robotok kepesek sajat kornyezetuk felderitesere ismeretlen kornyezetben is kulombozo syenzorokat hasznalva melyekkel kepések a kornyezetuk parametereinek a megmeresere. Az $AMR$ tipusu robotokat manapsag leginkabb tarolo helysegekben hasznaljak ahol anyagokat paletakat vagy kulombozo dolgokat kel egyik helyszinrol a masikra szalitani pl(Mobile Industrial Robots ApS valalat fejlesztesei).

\section{Mobilis Robotok Helyváltoztatása}
A $MR$ kerekekkel, hernyotalpakkal, vagy labakkal kepesek helyuket megvaltoztatni. A leginkabb hasznalatos es robusztusnak bizonyult kulteri terepen a kerekek es a herbyutalpak.
Legismertebb robot felepitesek: packboot pl: iRobo510,
negy hagyomanyos kerek pl: husky, negy kormanyozhato kerek, hat kerek amibol negy kormanyozhato pl: Curiosity Mars Rover.
Az utobi evekben a kerek fogalma is ujraertelmezodot, a hagyomanyos kerekek pl: RHex Rough-Terrain amely a kerek es a lebak kevereke.


\section{Kültéri mobilis robotok}


\renewcommand{\xname}{iRobo510}
\renewcommand{\x}{0.521}
\renewcommand{\y}{0.686}
\renewcommand{\z}{0.178}
\renewcommand{\weight}{20}
\renewcommand{\img}{MobilisRobotok/iRobo/iRobo510.jpg}
\renewcommand{\sources}{Forrás: https://www.army-technology.com}
\renewcommand{\captionn}{iRobo 510 lánctalpas packboot}
\renewcommand{\watherProf}{Igen -3m ig.}
\renewcommand{\sebesseg}{9.3}
\renewcommand{\weight}{10.89}
\renewcommand{\AcAndGy}{Igen}
\renewcommand{\GPS}{Igen}
\subsection*{iRobo510}
 A robot külterén és belterén is egyaránt használható, felderítésekre és kisebb beavatkozásra alkalmas a manipulátor karát használva.A robotot 2007-ben dobták piacra. A katonaság használta bombák hatástalanítására vagy felderítésre. 

\renewcommand{\aspectratioPic}{0.5}
\input{picture.tex}

\begin{center}
\center
\begin{tabular}{ llll  }
 \hline
 Tulajdonság              &           &          & Mértékegység \\ \hline
\multirow{3}{*}{Méretek} & X         &    \x      & m  \\
                         & Y         &    \y      & m  \\
                         & Z         &    \z      & m  \\
Önsuly                   & \multicolumn{2}{l}{\weight} & kg \\
Max Sebesség             & \multicolumn{2}{l}{\sebesseg} & km/h \\
Gyorsulásmérő és Giroszkóp & \multicolumn{2}{l}{\AcAndGy} & 
\end{tabular}
\end{center}


\renewcommand{\xname}{Husky}
\renewcommand{\x}{0.99}
\renewcommand{\y}{0.67}
\renewcommand{\z}{0.39}
\renewcommand{\weight}{50 + 75}
\renewcommand{\img}{MobilisRobotok/clearpathrobotics/husky.jpg}
\renewcommand{\sources}{Forrás: https://robots.ros.org/husky}
\renewcommand{\captionn}{Négykerekű mobilis platform.}
\renewcommand{\watherProf}{Igen}
\renewcommand{\sebesseg}{3.6}
\renewcommand{\AcAndGy}{Igen}
\renewcommand{\GPS}{Igen}

\subsection*{Husky}
Nagy teherbírású kültéri mobilis robot, leginkább kutatási célokra használják, nagy keréknyomatéka miatt nehéz terepen is jól boldogul. Nagy felbontású inkrementális szenzorokkal szerelték fel, alacsony mozgási sebességere is kepés. Működése 3 ora átlagos használat mellett, támogatja teljes mertekben ROS-t.

\renewcommand{\aspectratioPic}{0.6}
\input{picture.tex}
\begin{center}
\center
\begin{tabular}{ llll  }
 \hline
 Tulajdonság              &           &          & Mértékegység \\ \hline
\multirow{3}{*}{Méretek} & X         &    \x      & m  \\
                         & Y         &    \y      & m  \\
                         & Z         &    \z      & m  \\
Önsuly                   & \multicolumn{2}{l}{\weight} & kg \\
Max Sebesség             & \multicolumn{2}{l}{\sebesseg} & km/h \\
Gyorsulásmérő és Giroszkóp & \multicolumn{2}{l}{\AcAndGy} & 
\end{tabular}
\end{center}


\renewcommand{\xname}{ecorobotix}
\renewcommand{\x}{2.2}
\renewcommand{\y}{1.70}
\renewcommand{\z}{1.30}
\renewcommand{\img}{MobilisRobotok/EcoRobotix/ecorobotix.jpg}
\renewcommand{\sources}{Forrás: https://www.ecorobotix.com/}
\renewcommand{\captionn}{ecorobotix mezőgazdasági robot}
\renewcommand{\watherProf}{Igen}
\renewcommand{\sebesseg}{1.4}
\renewcommand{\weight}{130}
\renewcommand{\AcAndGy}{Igen}
\renewcommand{\GPS}{Igen, GPS RTK}

\subsection*{Ecorobotix}

Mezőgazdaság számára kifejlesztett robot, négy kerékkel rendelkezik amelyek közül kettőt motor hajt, és a másik kettő ön-beálló kerék. Kamera segítségével ismeri fel a növényeket és a pozíciójukat, és ha szükséges akkor be is avatkozik. 

A lokalizációra egy nagy pontosságú GPS RTS alkalmaznak.

\renewcommand{\aspectratioPic}{0.5}
\input{picture.tex}
\begin{center}
\center
\begin{tabular}{ llll  }
 \hline
 Tulajdonság              &           &          & Mértékegység \\ \hline
\multirow{3}{*}{Méretek} & X         &    \x      & m  \\
                         & Y         &    \y      & m  \\
                         & Z         &    \z      & m  \\
Önsuly                   & \multicolumn{2}{l}{\weight} & kg \\
Max Sebesség             & \multicolumn{2}{l}{\sebesseg} & km/h \\
Gyorsulásmérő és Giroszkóp & \multicolumn{2}{l}{\AcAndGy} & 
\end{tabular}
\end{center}



\renewcommand{\xname}{Spirit}
\renewcommand{\x}{1.6}
\renewcommand{\y}{2.3}
\renewcommand{\z}{1.5}
\renewcommand{\weight}{35 (felszereléssel 180)}
\renewcommand{\img}{MobilisRobotok/Spirit/spirit.jpg}
\renewcommand{\sources}{Forrás:
https://en.wikipedia.org/wiki/Mars_Exploration_Rover}
\renewcommand{\captionn}{Spirit nevu Marsjáró robot.}
\renewcommand{\watherProf}{Igen -3m ig.}
\renewcommand{\sebesseg}{0.05(avg 0.01)}
\renewcommand{\AcAndGy}{Igen}
\renewcommand{\GPS}{Igen}
\subsection*{Spirit}
 Marsi körülményekre tervezett robot 6 kerekkel rendelkezett amelyből 4 kormányzott
 négy sztereo kamerapárral elátva,30\degree lejtöt volt képes megmaszni. 
Működési ideje 6 év és 2 hónap volt a Marson, a végen homokba süllyedve egy sziklára akadva
érte a marsi tél, amely ahhoz vezetett hogy a nap elemei nem szolgáltattak elegendő
energiát és így végleg leált és kihült.

\renewcommand{\aspectratioPic}{0.5}
\input{picture.tex}

\begin{center}
\center
\begin{tabular}{ llll  }
 \hline
 Tulajdonság              &           &          & Mértékegység \\ \hline
\multirow{3}{*}{Méretek} & X         &    \x      & m  \\
                         & Y         &    \y      & m  \\
                         & Z         &    \z      & m  \\
Önsuly                   & \multicolumn{2}{l}{\weight} & kg \\
Max Sebesség             & \multicolumn{2}{l}{\sebesseg} & km/h \\
Gyorsulásmérő és Giroszkóp & \multicolumn{2}{l}{\AcAndGy} & 
\end{tabular}
\end{center}



%\lhead{Robot Operációs Rendszer}

\section{Robot Operációs Rendszer} 
A ROS 2007-ben jelent meg, gyorsan elterjedt az egész világon, manapság szinte minden robotokkal foglalkozó cég termékei kapcsolódnak a ROS-hoz.

A ROS működéséhez szükséges egy gazda operációs rendszer, UNIX vagy Windows alapú, amelyen a központi node fut (rosmaster). A master kezeli a többi node közti kommunikációt, paramétereket, szervízhívásokat.

A kommunikáció a nodok közözt TCP protokolra épül, amely XML/PRC technoloiát használ, RPC távoli eljáráshívást jelent \cite{xmlrpc}.
Minden node jelzi a rosmasternek, milyen adatokat szeretne megosztani, ezeket az advertise függvényhívások jelzik. A nodeok ugyanakkor feliratkozhatnak, ezeket a subscribe függvényekkel valósíthatjuk meg pl.: \cite{rossubpubexample}.
A szervízhívások 5 lépéses folyamatból állnak, látható a \ref{fig:rosmech1}, valamint az adatfolyamok 8 lépésből \ref{fig:rosmech2}.



\renewcommand{\img}{AktualisTudomany/rosmech1.jpg}
\renewcommand{\sources}{Forrás: http://answers.ros.org}
\renewcommand{\captionn}{ROS kommunikációs mechanizmus szervizhívásokra}
\renewcommand{\aspectratioPic}{0.7}
\renewcommand{\figlabel}{rosmech1}
\input{picture.tex}


\renewcommand{\img}{AktualisTudomany/rosmech2.jpg}
\renewcommand{\sources}{Forrás: http://answers.ros.org}
\renewcommand{\captionn}{ROS kommunikációs mechanizmus adatfolyamokra}
\renewcommand{\aspectratioPic}{0.7}
\renewcommand{\figlabel}{rosmech2}
\input{picture.tex}

A \cite{lentin2015} \cite{NagyPirosKonyv} segítségével megalapozhatjuk a tudásunkat. Számos előnnyel rendelkezik a ROS használata egy új robot fejlesztésében, mert már elkészített eszközöket használhatunk pl: SLAM \footnote{ Simultaneous Localization and Mapping egyidejű térképezés és lokalizáció}, $AMCL$, vagy előre elkészített eszközök segítenek a fejlesztésben pl: Rviz  \footnote{ROS környezet vizualizációs eszköze}, Gazebo \footnote{ROS környezet szimulációs eszköze}, interfészt biztosít a szenzoroknak és beavatkozóknak pl: $LIDAR$, $IMU$ .

A \cite{lentin2015} említést tesz arról, hogy más hasonló keret-rendszerekhez képest a ROS stabilabb pl: ha egy modulban futás-idejű hiba lép fel, az nem terjed ki a többi csomópontra.
Egyszerűbb fejlesztés lehetséges azáltal, hogy kisebb modulokat fejlesztünk és nem egy nagy, több szálon futó kódot. Annak ellenére, hogy a forrás-kódja nyílt, a keret-rendszernek nagyon jó szupportja van, rengeteg fórumon keresztül kaphatunk megoldásokat az esetleges hibákra. Több technológiát képes összekapcsolni mint pl.: tensorflow, matlab, simulink, opencv, V-Rep...

Hátrányai között említhető a Gazebo szimulátor, mert a használata nem egyszerű, ellentétben a V-Rep \footnote{Robot szimulációs szoftverrel, amely támogatja a ROS-t} programmal. A robot modellezése nem egyszerű dolog, $URDF$ fájl szükséges hozzá, csak SolodWork \footnote{3D modellező szoftver} biztosít  lehetőséget arra, hogy modellt exportálhassuk.

\subsection{Új robot integrációja ROS-hoz}
Egy új robot integrációja során meg kell vizsgálni, hogy milyen mérési adatok állnak rendelkezésünkre alacsony szinten, a rotációs csuklók paraméterei lehetnek pl: szögpozíció, szögsebesség, kifejtett nyomaték, ezeket a paramétereket mérhetjük, illetve referenciaként is előírhatjuk. 

Az integrációra több megoldás is lehetséges:
\begin{enumerate}[label=(\alph*)]
	\item ROS Serialon keresztül.
	\item ROS control használata.
	\item Osztott Rendszer.
\end{enumerate}

\subsubsection*{Ros Serial}
Egy megoldás a hardver integrációjára a ROS Serial, amely UART vagy TCP protokollra épülő, soros vagy hálózati kommunikációt használva. Korlátai miatt \cite{RosSerial} nem képes nagy méretű üzentek használatára, valamint a nodoek száma is korlátozott. Szükséges a ROS csomagok használata a hardveren, ami nem mindig előnyös, függőség alakul ki a hardver fejlesztése közben a szoftver irányába.

A \cite{ROSArduino2013} cikkben egy arduino típusú fejlesztő lappal valósítja meg a robot alacsony szintű szabályzását, megemlíti, hogy a rosserial-t nem tudja használni a limitációk miatt. A feldolgozó oldalon elkészít egy saját szoftveres modult, amely megvalósítja a ROS és a hardver közti kapcsolatot.A kommunikációra soros $UART$ protokollt használ.

A \cite{ROSRoboticsByExampleSecEd} könyv 8. fejezetben leírja, hogyan lehet használni a rosserial-t, arduino, valamint Raspberry Pi fejlesztő lapokon, viszont nem tesz említést a hátrányairól.

\subsubsection*{Ros Control}

A ros controller  \cite{roscontrol} használatával összeköthetjük az alacsony szintű hardvert a ROS keretrendszerben fejlesztett modulokkal, implementálva \cite{ROSControlExample} a hardware\_interface::RobotHW interfészt és létrehozva minden egyes rotációs csuklónak egy \newline hardware\_interface::JointStateHandle-t. A \ref{fig:RosControlDiagram} látható read() és write() függvényeken keresztül kell megvalósítani a kommunikációt a hardverrel, ez történhet hálózaton vagy soros porton keresztül.

\renewcommand{\img}{AktualisTudomany/roscontrol.png}
\renewcommand{\sources}{Forrás: https://www.army-technology.com}
\renewcommand{\captionn}{Ros Control modulok}
\renewcommand{\figlabel}{RosControlDiagram}
\input{picture.tex}

A \ref{fig:RosControlDiagram} ábrán látható az interfészek kapcsolatai és a fontosabb függvényhívások. A hardverrel való integráció write() és read() fűvényhívásokkal valósul meg, ebben a két függvényben kell elkészíteni a programokat, amelyek kepések kiolvasni és beírni az eszközben a kívánt jeleket. Itt használhatunk több típusú, alacsony szintü kommunikációs protokolt pl: TCP,UART.., vagy bármilyen interfészt, amit a gazda operációs rendszer elismer.

\subsection*{Osztott Rendszer}

A  \cite{DistributedRealTimeControlROS} cikkben leír egy megoldást arra, hogyan lehetne smart eszközként tekinteni a szenzorokra és beavatkozó modulokra. Osztott rendszert $DSC$ -t alkalmaz, ahol minden szenzornak saját mastere van, ezáltal minden node független lesz a hálózaton. Enélkül a hálózat konfigurációját teljes mértékben ismerni kell minden nodenak a többi node IP címét, de ezzel a megoldással futásidőben módosíthatjuk a hálózatot, $GUID$ -t használ a új maszter bejelentésre a hálózaton, valamint ezzel oldja meg az információk áramlását is.
A kommunikációt a masterek között ROS MultiPeer Architecture-nek (RMPA) nevezett architektúrával oldja meg. Más rendszerekhez képest kétszer jobb időkésést valósított meg. 
Ami a hátránya, a szenzorok mellett egy mcu is szükséges, amely kepés kell legyen egy operációs rendszert futtatni és egy ros mastert. Valóban robusztusabb, modulárisabb lesz a rendszer ezáltal, de drágább is. Kisebb rendszereknél inkább hátrány mint előny, viszont komplex, térben nagy területet lefedő rendszernél előnyös.

Egy másik hasonló megoldás, ahol több ROS mastert lehet összekötni és felügyelni, a multimaster\_fkie megoldja a futásidőben való új master szinkronizációját, a topikok és a szervizek kezelését is. 


 \newpage

\input{AktualisTudomany/RobotModel.tex}


