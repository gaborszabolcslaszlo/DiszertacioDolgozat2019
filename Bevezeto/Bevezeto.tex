\chapter{Bevezető}
A robotokat széles körben alkalmazzák, egyre több feladatra és most már az átlag emberek életében is. Néhány nagyobb vállat mint pl: ABB, Kuka nagy területet foglalt el az iparban robot karok gyártásában.Emellett egyre több kisebb vállalat jelent meg, amelyek háztartási vagy fél katonai eszközöket gyártanak pl.: iRobo. A mezőgazdaságban is próbálnak alkalmazni autonóm robotokat pl.: echorobotics amelyek segítségével hatóanyag mentes élelmiszereket állíthatnak elő.

A dolgozat célja hogy felderítse az aktuális legmodernebb technikákat amelyeket robotokon alkalmaznak, és ezeket alkalmazza egy kültéri mobilis robot megépítése során.
Az előzőleg már egy hasonló rendszer kivitelezésekor elteszitek modulok továbbfejlesztése és integrálása az új rendszerbe. A robot mozgása négy csiga áttétel és négy DC motor segítségével valósul meg. A motrok szögsebességét és  felvett arámokat merjük, inkrementális szenzor és elektromágneses jelenségre alapuló aramerő modul segítségével.

A beavatkozás feszültség formájában történik, amelyet H-híddal állítunk elő azáltal hogy PWM jelet generálunk.

Az inkrementális és aramerő szenzorok jeleit FPGA alapú fejlesztőlapokkal oldjuk meg. A roboton helyet foglal egy kis méretű számítógép is amely USB vezetékeken keresztül csatlakozik az FPGA lapokhoz és a roboton található más szenzorokhoz pl: LIDAR, IMU, GPS..
Az FPGA modulok segitsegevel valosulnak meg a kerekek szogsebesseg szabalyzasa, a dolgozatban targyalasra kerul az egyedi hardver integracioja ROS keretrendszerben, amelyet egy koztes nodal valositunk meg.
A robot a csiga áttelek miatt nagy nyomatékot tud előállítani, ami a mozgási sebesség rovására vált. A robot kepés 0.3 m/s sebességél mozogni előre, és 20 \degree/s forgási sebességet generálni súlypontja körül.

A dolgozat célja az elkészített kültéri mobilis robottal meréseket végezni különböző terepviszonyok között és azokat összehasonlítási. A terepviszonyok mit pl: füves, kavicsos, aszfalt, márvány stb. A meréseket szeretnek összehasonlítani a terepviszonyok függvényében: a robot mozgását és a szükséges energia nagyságát, a környezetében okozott behatások.

A robot SLAM algoritmus  dolgozza fel a LIDAR által mért adatokat és egy 2D térképet állít elő, amelyen kepés egyidőben behatárolnia pozícióját és a térképet építeni is. Ezen meglevő programok használatával kültéri meréseket végezni, térképeket készíteni és majd egy előírt pályán végigmenni, ezen terepeken a térképet használva. Az esetleges zavao tenyezok felderites amelyek befolyasoljak a terkepezes es a lokalizacio pontosagat.
