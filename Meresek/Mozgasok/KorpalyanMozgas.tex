\subsection{Kavicsos talajon korpalyan 7 5}
A \ref{fig:KorP0705a} megfigyelhető amint a robot kavicsos talajon differenciálisan fordul 60 másodpercen keresztül, ezalatt háromszor teljen korbefordul. A palyat tekintve letrejon egy oldaliranyu mozgás is igy 0.4m kerul odebb. Az oldaliranyu mozgas a nem egyenlo surlodasok es eroeloszlasok miatt jon letre.

\renewcommand{\GlobalPath}{Meresek/Mozgasok/NormalMukodes/Korpalya_07_05_Kavicsos/}
\renewcommand{\secondImage}{*}

%kep a talajrol
%

\renewcommand{\sources}{}
\renewcommand{\captionn}{Kep a felszinrol}
\renewcommand{\figlabel}{figm}


\begin{kep}
    \begin{figure}[H]%
    \begin{center}
    
    \subfloat[label a]{
        {\includegraphics[width=9cm]{\mand{\GlobalPath}{talaj1.jpg}} }
        \label{fig:ex3-a}
    }%
    
    \ifthenelse{\equal{\secondImage}{*}}
    {}
    {
        \qquad
        \subfloat[label b]{{\includegraphics[width=9cm]{\mand{\GlobalPath}{talaj1.jpg}} }}%
    }
  
    \label{fig:example}%
    \end{center}
\end{figure}
\end{kep}

\renewcommand{\secondImage}{*}



%1
%\input{Meresek/Mozgasok/FirstV1.tex}

\begin{figure}[H]
  \includegraphics{tikz/KorP0705x.pdf}
  \caption{$SSMR-4W$ típusú robot mozgása, tengelyekre bontva, kereksebessegek BL=FL=0 es a FR=BR= 50\degree/s}
  \label{fig:KorP0705x}  
\end{figure}


\begin{figure}[H]
  \includegraphics{tikz/KorP0705a.pdf}
  \caption{$SSMR-4W$ típusú robot mozgása, tengelyekre bontva, kereksebessegek BL=FL=0 es a FR=BR= 50\degree/s }
  \label{fig:KorP0705a}  
\end{figure}


\begin{figure}[H]
  \includegraphics{tikz/KorP0705b.pdf}
  \caption{$SSMR-4W$ típusú robot altal leirt palya, kereksebessegek BL=FL=0 es a FR=BR= 50\degree/s}
  \label{fig:KorP0705b}
\end{figure}


\begin{figure}[H]
  \includegraphics{tikz/KorP0705c.pdf}
  \caption{$SSMR-4W$ típusú robot orientacioja, kereksebessegek BL=FL=0 es a FR=BR= 50\degree/s}
  \label{fig:KorP0705c}
\end{figure}


\begin{figure}[H]
  \includegraphics{tikz/KorP0705d.pdf}
  \caption{$SSMR-4W$ típusú robot fordulasi szogsebessege, kereksebessegek BL=FL=0 es a FR=BR= 50\degree/s}
  \label{fig:KorP0705d}
\end{figure}


\begin{figure}[H]
  \includegraphics{tikz/KorP0705e.pdf}
  \caption{$SSMR-4W$ típusú robot egyenesvonalu sebessegei, kereksebessegek BL=FL=0 es a FR=BR= 50\degree/s}
  \label{fig:KorP0705e}
\end{figure}


A mozgas nyilt hurokban tortenik nincsen a pozicio vagy a rogsebeseg szabalyozva.
A robot $ICR$ pontja a $K_{BL}$ es $K_{FL}$ pontokat oszekoto tengelyen helyezkedik el \ref{fig:SMR4WKinematics}.



















\subsection{Korpalya 7 3 Kavicsos talajon}


\renewcommand{\GlobalPath}{Meresek/Mozgasok/NormalMukodes/Korpalya_07_03_Kavicsos/}
\renewcommand{\secondImage}{*}

%kep a talajrol
%

\renewcommand{\sources}{}
\renewcommand{\captionn}{Kep a felszinrol}
\renewcommand{\figlabel}{figm}


\begin{kep}
    \begin{figure}[H]%
    \begin{center}
    
    \subfloat[label a]{
        {\includegraphics[width=9cm]{\mand{\GlobalPath}{talaj1.jpg}} }
        \label{fig:ex3-a}
    }%
    
    \ifthenelse{\equal{\secondImage}{*}}
    {}
    {
        \qquad
        \subfloat[label b]{{\includegraphics[width=9cm]{\mand{\GlobalPath}{talaj1.jpg}} }}%
    }
  
    \label{fig:example}%
    \end{center}
\end{figure}
\end{kep}

\renewcommand{\secondImage}{*}



%1
%\input{Meresek/Mozgasok/FirstV1.tex}



\begin{figure}[H]
  \includegraphics{tikz/KorP0703x.pdf}
  \caption{$SSMR-4W$ típusú robot mozgása, tengelyekre bontva, kereksebessegek BL=FL=0 es a FR=BR= 50\degree/s}
  \label{fig:KorP0703x}
\end{figure}


\begin{figure}[H]
  \includegraphics{tikz/KorP0703a.pdf}
  \caption{$SSMR-4W$ típusú robot mozgása, tengelyekre bontva, kereksebessegek BL=FL=0 es a FR=BR= 50\degree/s }
  \label{fig:KorP0703a}
\end{figure}



\begin{figure}[H]
  \includegraphics{tikz/KorP0703b.pdf}
  \caption{$SSMR-4W$ típusú robot altal leirt palya, kereksebessegek BL=FL=0 es a FR=BR= 50\degree/s}
  \label{fig:KorP0703b}
\end{figure}


\begin{figure}[H]
  \includegraphics{tikz/KorP0703c.pdf}
  \caption{$SSMR-4W$ típusú robot orientacioja, kereksebessegek BL=FL=0 es a FR=BR= 50\degree/s}
  \label{fig:KorP0703c}
\end{figure}


\begin{figure}[H]
  \includegraphics{tikz/KorP0703d.pdf}
  \caption{$SSMR-4W$ típusú robot fordulasi szogsebessege, kereksebessegek BL=FL=0 es a FR=BR= 50\degree/s}
  \label{fig:KorP0703d}
\end{figure}

\begin{figure}[H]
  \includegraphics{tikz/KorP0703e.pdf}
  \caption{$SSMR-4W$ típusú robot egyenesvonalu sebessegei, kereksebessegek BL=FL=0 es a FR=BR= 50\degree/s}
  \label{fig:KorP0703e}
\end{figure}



A mozgas nyilt hurokban tortenik nincsen a pozicio vagy a rogsebeseg szabalyozva.
A robot $ICR$ pontja a $K_{BL}$ es $K_{FL}$ pontokat oszekoto tengelyen helyezkedik el \ref{fig:SMR4WKinematics}.











\subsection{Körpályák összehasonlítása}


\begin{table}[H]
\begin{tabular}{llllll}
\hline L & R & Delta & R mért & R számitott & Külömbség \\ \hline
 0 &  50 &    50   &        &             &           \\
 14 & 49 &    35   &        &             &           \\
 70 & 40 &    30   &        &             &           \\
 70 & 20 &    50   &        &             &           \\
 25 & 50 &    25   &        &             &          
\end{tabular}
\end{table}



A lenti ábrán látható a kavicsos illetve füves terepen mert körpályák ábrázolása kétfelé körpályát irtunk elő, a füves talajon nagyobb sebességél haladtunk. Füves talajon a 

\begin{figure}[H]
  \includegraphics{tikz/KorpalyakKavicsos.pdf}
  \caption{Kulombozo korpalyak}
  \label{fig:KorpalyakKavicsos}  
\end{figure}