\subsection{Lepcson}




\renewcommand{\GlobalPath}{Meresek/Mozgasok/Lepcso}
\renewcommand{\secondImage}{*} 

%kep a talajrol
%

\renewcommand{\sources}{}
\renewcommand{\captionn}{Kep a felszinrol}
\renewcommand{\figlabel}{figm}


\begin{kep}
    \begin{figure}[H]%
    \begin{center}
    
    \subfloat[label a]{
        {\includegraphics[width=9cm]{\mand{\GlobalPath}{talaj1.jpg}} }
        \label{fig:ex3-a}
    }%
    
    \ifthenelse{\equal{\secondImage}{*}}
    {}
    {
        \qquad
        \subfloat[label b]{{\includegraphics[width=9cm]{\mand{\GlobalPath}{talaj1.jpg}} }}%
    }
  
    \label{fig:example}%
    \end{center}
\end{figure}
\end{kep}

\renewcommand{\secondImage}{*}



%1
%\input{Meresek/Mozgasok/FirstV1.tex}


A következő mérések során a robot egy 40 \degree lépcsőn lefele és felfele is mozog, a lépcsőre merőleges irányban. A robot viselkedése a mozgás során, lefele könnyedén megy gond nélkül, felfele viszont a kerekek a következő lépcsőfok éléről lecsúszva visszaesnek.

A \ref{fig:LepcsoLexx} latható amit a lépcsőn lefele, a mozgato motrok mért értékei. és a \ref{fig:LepcsoFelxx} viszafele mozgás során a mért értékek. A beavatkozó jel nagysága 10\% -al  nagyobb viszafele mozgas során. A mérések elvegzésekor a hajtást végző motrok a kisebbik átételfokozatban voltak, igy nagyobb forgatónyomatékot adtak le.

\begin{figure}[H]
  \includegraphics{tikz/LepcsoLexx.pdf}
  \caption{Lépcsőn lefele mozgás.}
  \label{fig:LepcsoLexx}
\end{figure}

A roboton IMU szenzora által mért értékek mutatják amint a $g=9.81 m/s^2$ gravitácios gyorsulás megjelenik a $aZ$ tengelyen \ref{fig:ImuLepcsoLe1}. Kezdetben a robot vizszinteshez közeli állapotban van X és Y.  A lépcson lefele mozgás során a $g$ fokozatosan átevődik az $aX$ tengelyre is amiatt, hogy a robot előre dől. A robot három lépcsőfokon halad át ami látható az ábrán is.

\begin{figure}[H]
  \begin{center}
  	\includegraphics[scale=0.8]{tikz/ImuLepcsoLe1.pdf}
  \end{center}
  \caption{Lépcsön lefele mozgás, három lépcsőfok.}
  \label{fig:ImuLepcsoLe1}
\end{figure}

\begin{figure}[H]
  \begin{center}
  	\includegraphics[scale=0.8]{tikz/ImuLepcsoFel1.pdf}
  \end{center}
  \caption{Lépcsőn felfele mozgás, kétlépcsőfok.}
  \label{fig:ImuLepcsoFel1}
\end{figure}

A lépcsőn felfele mozgás során a robot az előző állapotból indul visszafele. Azokban a pillanatokban, amikor a kerekek lecsúsznak a lépcső éléről, a kerekek szögsebessége megnő, mert a súrlódási erő lecsökken.  

Az \ref{fig:LepcsoFelxx} az $FL$ és $FR$ kerekeken nagyobb beavatkozó jel esik, amiatt hogy megnő a merőleges nyomóerő.

\begin{figure}[H]
  \includegraphics{tikz/LepcsoFelxx.pdf}
  \caption{Lépcsőn felfele mozgás}
  \label{fig:LepcsoFelxx}
\end{figure}















