\subsection{Feloldali kerekek blokolva kavicsos talajon}

A robot baloldali kerekei leblokkolva és a jobboldali kerekei 50\degree/s szögsebességgel forognak. Az eredmények alapján a \ref{fig:Left0Right50b} latható a robot által le1írt pálya. A mozgás soran több mint 360 \degree -t fordul és mondhatni körpályat írt le. A talajjal való surlodások miatt a robot nem tökéletesen fordul ez látható abból is hogy a másodszori fordulás már az előzőhöz képest más középontal rendelkezik. 

\renewcommand{\GlobalPath}{Meresek/Mozgasok/HibasMukodes/R_0_L_1/}
\renewcommand{\secondImage}{*}

%kep a talajrol
%

\renewcommand{\sources}{}
\renewcommand{\captionn}{Kep a felszinrol}
\renewcommand{\figlabel}{figm}


\begin{kep}
    \begin{figure}[H]%
    \begin{center}
    
    \subfloat[label a]{
        {\includegraphics[width=9cm]{\mand{\GlobalPath}{talaj1.jpg}} }
        \label{fig:ex3-a}
    }%
    
    \ifthenelse{\equal{\secondImage}{*}}
    {}
    {
        \qquad
        \subfloat[label b]{{\includegraphics[width=9cm]{\mand{\GlobalPath}{talaj1.jpg}} }}%
    }
  
    \label{fig:example}%
    \end{center}
\end{figure}
\end{kep}

\renewcommand{\secondImage}{*}



%1
%\input{Meresek/Mozgasok/FirstV1.tex}




\begin{figure}[H]
  \includegraphics{tikz/Left0Right50a.pdf}
  \caption{$SSMR-4W$ típusú robot poziciója, X és Y tengelyekre bontva, kereksebességek BL=FL=0 és a FR=BR= 50\degree/s}
    \label{fig:Left0Right50a}
\end{figure}


\begin{figure}[H]
  \includegraphics{tikz/Left0Right50b.pdf}
  \caption{$SSMR-4W$ típusú robot által leírt pálya, kereksebességek BL=FL=0 és a FR=BR= 50\degree/s}
  \renewcommand{\figlabel}{Left0Right50b}
  \label{fig:Left0Right50b}
\end{figure}

A mérés során a fordulási szögsebesség 9\degree/s látható a \ref{fig:Left0Right50c} ábran. A LIDAR és HectorMap segítségével mért abszolut szögsebesség zajosabb mint a giroszkóp által mért. A LIDAR-al mért szögsebesség előnyösebb mert a zajokat nem kell integrálni ahoz hogy megkapjuk a szögsebességet a giroszkóppal ellentétben.

A lineáris sebességeket tekintve \ref{fig:Left0Right50d} szinuszosan változnak, az X és Y tengelyeken, megfigyelhető egy 90\degree eltolódás az X és Y tengelyeken mért szinuszos mozgásban. A kerületi sebesség 0.1 m/s körül adható meg.

\begin{figure}[H]
  \includegraphics{tikz/Left0Right50c.pdf}
  \caption{$SSMR-4W$ típusú robot orientácíőja,ha a kerékszögsebességek BL=FL=0 és a FR=BR=50\degree/s}
  \label{fig:Left0Right50c}
\end{figure}


\begin{figure}[H]
  \label{fig:Left0Right50d}
  \includegraphics{tikz/Left0Right50d.pdf}
  \caption{$SSMR-4W$ típusú robot fordulási szögsebessége Giroszkóp és LIDAR által mért, ha a kerékszögsebességek BL=FL=0 és a FR=BR= 50\degree/s}
  \label{fig:Left0Right50d}
\end{figure}


\begin{figure}[H]
  \includegraphics{tikz/Left0Right50e.pdf}
  \caption{$SSMR-4W$ típusú robot súlypontjának sebessége X és Y tengelyekre bontva, ha a kerékszögsebességek BL=FL=0 és a FR=BR= 50\degree/s}
  \label{fig:Left0Right50e}  
\end{figure}










