\subsection{Kavicsos talajon körpályán 50/25}

Az alábbi méréseknél 80-90s között a robot jobboldali kerekeit vezérlő H-híd túlmelegedése miatt a beléjük épített védelmi funkciónak köszönhetően leálltak így a jobboldali kerek leblokkoltak, így a mozgás pályája is megváltozott.


\renewcommand{\GlobalPath}{Meresek/Mozgasok/NormalMukodes/Korpalya_07_03_Kavicsos/}
\renewcommand{\secondImage}{*}

%kep a talajrol
%

\renewcommand{\sources}{}
\renewcommand{\captionn}{Kep a felszinrol}
\renewcommand{\figlabel}{figm}


\begin{kep}
    \begin{figure}[H]%
    \begin{center}
    
    \subfloat[label a]{
        {\includegraphics[width=9cm]{\mand{\GlobalPath}{talaj1.jpg}} }
        \label{fig:ex3-a}
    }%
    
    \ifthenelse{\equal{\secondImage}{*}}
    {}
    {
        \qquad
        \subfloat[label b]{{\includegraphics[width=9cm]{\mand{\GlobalPath}{talaj1.jpg}} }}%
    }
  
    \label{fig:example}%
    \end{center}
\end{figure}
\end{kep}

\renewcommand{\secondImage}{*}



%1
%\input{Meresek/Mozgasok/FirstV1.tex}



\begin{figure}[H]
  \includegraphics{tikz/KorP0703x.pdf}
  \caption{$SSMR-4W$ típusú robot motorvezérlő jelei, ha kerékszögsebességek BL=FL=25\degree/s és a FR=BR=50\degree/s}
  \label{fig:KorP0703x}
\end{figure}


\begin{figure}[H]
  \includegraphics{tikz/KorP0703a.pdf}
  \caption{$SSMR-4W$ típusú robot mozgása, tengelyekre bontva, ha kerékszögsebességek BL=FL=25\degree/s és a FR=BR=50\degree/s }
  \label{fig:KorP0703a}
\end{figure}



\begin{figure}[H]
  \includegraphics{tikz/KorP0703b.pdf}
  \caption{$SSMR-4W$ típusú robot által leírt pálya, ha kerékszögsebességek BL=FL=25\degree/s és a FR=BR=50\degree/s}
  \label{fig:KorP0703b}
\end{figure}


%\begin{figure}[H]
%  \includegraphics{tikz/KorP0703c.pdf}
%  \caption{$SSMR-4W$ típusú robot orientációja, ha kerékszögsebességek %BL=FL=25\degree/s és a FR=BR=50\degree/s}
%  \label{fig:KorP0703c}
%\end{figure}


\begin{figure}[H]
  \begin{center}
  	\includegraphics[scale=1]{tikz/KorP0703d.pdf}
  \end{center}
  \caption{$SSMR-4W$ típusú robot fordulási szögsebessége, ha kerékszögsebességek BL=FL=25\degree/s és a FR=BR=50\degree/s}
  \label{fig:KorP0703d}
\end{figure}

%\begin{figure}[H]
%  \includegraphics{tikz/KorP0703e.pdf}
%  \caption{$SSMR-4W$ típusú robot egyenesvonalu sebessegei, ha %kerékszögsebességek BL=FL=25\degree/s és a FR=BR=50\degree/s}
%  \label{fig:KorP0703e}
%\end{figure}


