\subsection{Kavicsos talajon korpalyan 7 5}
A \ref{fig:KorP0705a} megfigyelhető amint a robot kavicsos talajon differenciálisan fordul 60 másodpercen keresztül, ezalatt háromszor teljen korbefordul. A palyat tekintve letrejon egy oldaliranyu mozgás is igy 0.4m kerul odebb. Az oldaliranyu mozgas a nem egyenlo surlodasok es eroeloszlasok miatt jon letre.

\renewcommand{\GlobalPath}{Meresek/Mozgasok/NormalMukodes/Korpalya_07_05_Kavicsos/}
\renewcommand{\secondImage}{*}

%kep a talajrol
%

\renewcommand{\sources}{}
\renewcommand{\captionn}{Kep a felszinrol}
\renewcommand{\figlabel}{figm}


\begin{kep}
    \begin{figure}[H]%
    \begin{center}
    
    \subfloat[label a]{
        {\includegraphics[width=9cm]{\mand{\GlobalPath}{talaj1.jpg}} }
        \label{fig:ex3-a}
    }%
    
    \ifthenelse{\equal{\secondImage}{*}}
    {}
    {
        \qquad
        \subfloat[label b]{{\includegraphics[width=9cm]{\mand{\GlobalPath}{talaj1.jpg}} }}%
    }
  
    \label{fig:example}%
    \end{center}
\end{figure}
\end{kep}

\renewcommand{\secondImage}{*}



%1
%\input{Meresek/Mozgasok/FirstV1.tex}

\begin{figure}[H]
  \includegraphics{tikz/KorP0705x.pdf}
  \caption{$SSMR-4W$ típusú robot mozgása, tengelyekre bontva, kereksebessegek BL=FL=0 es a FR=BR= 50\degree/s}
  \label{fig:KorP0705x}  
\end{figure}


\begin{figure}[H]
  \includegraphics{tikz/KorP0705a.pdf}
  \caption{$SSMR-4W$ típusú robot mozgása, tengelyekre bontva, kereksebessegek BL=FL=0 es a FR=BR= 50\degree/s }
  \label{fig:KorP0705a}  
\end{figure}


\begin{figure}[H]
  \includegraphics{tikz/KorP0705b.pdf}
  \caption{$SSMR-4W$ típusú robot altal leirt palya, kereksebessegek BL=FL=0 es a FR=BR= 50\degree/s}
  \label{fig:KorP0705b}
\end{figure}


\begin{figure}[H]
  \includegraphics{tikz/KorP0705c.pdf}
  \caption{$SSMR-4W$ típusú robot orientacioja, kereksebessegek BL=FL=0 es a FR=BR= 50\degree/s}
  \label{fig:KorP0705c}
\end{figure}


\begin{figure}[H]
  \includegraphics{tikz/KorP0705d.pdf}
  \caption{$SSMR-4W$ típusú robot fordulasi szogsebessege, kereksebessegek BL=FL=0 es a FR=BR= 50\degree/s}
  \label{fig:KorP0705d}
\end{figure}


\begin{figure}[H]
  \includegraphics{tikz/KorP0705e.pdf}
  \caption{$SSMR-4W$ típusú robot egyenesvonalu sebessegei, kereksebessegek BL=FL=0 es a FR=BR= 50\degree/s}
  \label{fig:KorP0705e}
\end{figure}


A mozgas nyilt hurokban tortenik nincsen a pozicio vagy a rogsebeseg szabalyozva.
A robot $ICR$ pontja a $K_{BL}$ es $K_{FL}$ pontokat oszekoto tengelyen helyezkedik el \ref{fig:SMR4WKinematics}.

















