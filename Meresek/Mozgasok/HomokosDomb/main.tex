\subsection{Homokos Lejtodfdsfds}


\renewcommand{\GlobalPath}{Meresek/Mozgasok/HomokosDomb/}
\renewcommand{\secondImage}{*}

A lejto meredeksege 45\degree, a szerkezete homokos, nagyobb meretu szilard darakobbal maelyek elerik a 4cm atmerot is.
A lejtot az FL es FR kerekekkel kozelitjuk meg, igy a BL es BR kerekekre nagyobb nyomoero jut.
A \ref{fig:HlejtoKerek} lathato a PWM beavatkozo jelek a BL es BR kerekeken 20\% nagyobb peavatkozo jelet szukseges ugyanannak a referencia jelnek a kovetesere.

A kereke eloirt forasi sebessege 15\degree/s, torekedtunk hogy a kereke neassak be magukat a homokba de 0.5m \ref{fig:Hlejto2}  megtetele utan
a BL es a BR kereke 10 cm melyen a homokba sulyetek es a robot elakadt \ref{fig:Hlejto1}. Egy masik eszrevetel hogy a FL es FR kereke egyaltalan nem astak be a talajba magukat forgas kozben ami a kisebb meroleges nyomoeronek tulajdonithato. 

%kep a talajrol
%

\renewcommand{\sources}{}
\renewcommand{\captionn}{Kep a felszinrol}
\renewcommand{\figlabel}{figm}


\begin{kep}
    \begin{figure}[H]%
    \begin{center}
    
    \subfloat[label a]{
        {\includegraphics[width=9cm]{\mand{\GlobalPath}{talaj1.jpg}} }
        \label{fig:ex3-a}
    }%
    
    \ifthenelse{\equal{\secondImage}{*}}
    {}
    {
        \qquad
        \subfloat[label b]{{\includegraphics[width=9cm]{\mand{\GlobalPath}{talaj1.jpg}} }}%
    }
  
    \label{fig:example}%
    \end{center}
\end{figure}
\end{kep}

\renewcommand{\secondImage}{*}



%1
%\input{Meresek/Mozgasok/FirstV1.tex}



\renewcommand{\img}{Meresek/Mozgasok/HomokosDomb/kep1.jpg}
\renewcommand{\aspectratioPic}{0.7}
\renewcommand{\figlabel}{Hlejto1}
\renewcommand{\captionn}{Homokos domb 1}
\input{picture.tex}


\renewcommand{\img}{Meresek/Mozgasok/HomokosDomb/kep2.jpg}
\renewcommand{\aspectratioPic}{0.7}
\renewcommand{\figlabel}{Hlejto2}
\renewcommand{\captionn}{Homokos domb 1}
\input{picture.tex}


\begin{figure}[H]
  \includegraphics{tikz/HlejtoKerek.pdf}
  \caption{}
  \label{fig:HlejtoKerek}
\end{figure}











