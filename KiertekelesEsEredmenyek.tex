\chapter{Eredmények Kiértékelése}
\section{Megvalósítások}
Ezzel a dolgozattal négy év folyamatos munkájának egy szakaszát szeretném lezárni, a következő eredményeket sikerült felmutatni, kronológiai sorrendben: 
A mechanikai szerkezet fejlesztése, a 2011 be fejlesztett verzióhoz képest, amely sokkal robusztusabb, kültérre alkalmasabb mint az előző, és egyszerűbb is. Ugyan megmaradt az a tendencia hogy csiga-áttételeket alkalmazzak a kerekek meghajtására. Az áttételeket magam terveztem és gyártattam le, a kezdeti alacsony költségvetés miatt az 3D nyomtatóval elkészített inkrementális szenzort nem lehetett alkalmazni robusztusan, ami nem az jeleni, hogy nem is lehetséges csak túl sok időt igényelt volna, a csiga tengely kotyogása miatt.

Az alkatrészeket 3D tervező programban elkészítettem, és 3D nyomtatóval elkészítettük, a tapasztalatom ezekkel az alkatrészekkel: nagy mechanikai terhelés elviselésére nem alkalmasak hosszútávon, ezért történt meg hogy a csiga tengely csapágyháza terhelés alatt széttört.

A következő lépésben a robot alacsonyrendű vezérlóáramköröket terveztem meg és viteleztem ki CNC marógép segítségével.

A vezérlő logika implementálására FPGA alapú fejlesztőlapokat használtam mert.

Vivado környezetet használtam az FPGA fejlesztésére, megvalósítottam egy uBlaze processzorrendszert kialakítását és több hardveres modult is amelyek a következők: PWM modul, UART protokoll csomag értelmező amely támogatja a nagy sebességű kommunikációt, globális engedélyező jel, ezeket a modulokat System Generator-ban valósítottam meg és IP mag készült ezekből.

Robotokhoz kapcsolódó keretrendszer használata lett szükséges így került sor a ROS keretrendszer ismereteinek az elsajátítására.

Megterveztem az integrációt a ROS és FPGA UART alapú kommunikációjának kié pitésére a jelen pillanatban a robot specifikus, hasznosulható dinamikusan ujra más robotoknál anélkül hogy ne kellene számottevő programkódot írni.

A rosz használata számos előnyel járt, sikerült bekonfigurálni és elindítani a ROS keretrendszerben levő egyeb eszközöket is pl: logolás a nodok között, eclipse fejlesztőkörnyezet bekonfigurálása a fejlesztéshez, HectorMap térképez és lokalizálás LIDAR alapján, robotmodell elkészítése Rviz 3D megjelenítő számára. A alacsonyrendű paraméterek szinkronizációja a ROS rendszerben levő paraméterekkel.

Sikeresen elsajátítottam a ROS alapjait, és megterveztem egy sajátos kommunikációt FPGA alapú rendszer és a ROS között. Az integráció a robot és a ROS között jól működik, minden egyes szenzor mért adata bekerül a ROS környezetbe.

A rendszer elindítása után, a megbecsültem MATLAB segítségével a kereke átviteli függvényeit, és majd ezekre PID szabályozót terveztem PID tuning toolbox segítségével.

A hectormap segítségével, az ismeretlen terep térképezésével, a roboton lokalizálva, sikeresen bekonfiguráltam a movebase nevű eszközt, amely segítségével a robotot egy adott pozícióba és irányba tudjuk elvinni. A move base megoldja az akadályok kikerülését is.

A térképezéssel kapcsolatban a tapasztalatok a mérések során, a LIDAR és HectorMap zajosak külső terepen egyrészt a környezeti tényezők miatt amelyek befolyásolják a robot dőlésszöget, így zajosítva a méréseket, valamit a lokalizáció nem pontos ha üvegen keresztülhalad a lézersugár.

A robottal végzet mérések során a robot szerkezete és alacsonyrendű szabályzása megbízhatónak bizonyult, képes több legalább 100kg függőleges irányú terhelést elviselni és akar egy személygépkocsit is elmozdítani.

A mérések alapján a lépcsőre könnyebben tud felmenni ha egy 90 \degree nál kisebb szög alatt közelítjük meg.

A környezetre fordulás esetén gyakorolja a robot a legnagyobb behatást, pl helyben forgás esetné a súrlódások miatt a füves talajt a fekete földig leszedi. Mezőgazdasági alkalmazásra előnyösebb volna ha mind a négy kereke kormányozhatnák.

A lépcsőn és a lejtőt is előnyösebb a SSMR al úgy megközelíteni hogy a súlypont a robot elején legyen ahogy felfele haladunk így a merőleges nyomóerők is egyenletesebben eloszlanak. Lépcső esetében ügyelni kell a hátsó kerekek lépcsőfokba való beragadása miatt,mert a kerekek nagy forgatónyomatéka, a robot hátra billentheti, ez abban az esetben állhat fent ha a lépcsőt 90\degree szög alatt közelítjük meg.


\section{Hasonló rendszerekkel való összehasonlítás}

\begin{center}
\begin{table}[]
\begin{tabular}{llll}
\hline Tulajdonság & Husky Robot & Előny & SapRoover \\ \hline
 széleség           &    0.67m    &    ?        &   0.78m        \\
   hosszuság        &    0.99m    &   ?        &   0.80m       \\
  magaság           &    0.39m    &   ?        &   0.40m        \\
 ROS 
 kompatibilitás           &    Igen         &   =    &   Igen        \\
   Max sebesség        &    1m/s  &     >   & 0.25 m/s    \\
   Bépitett számitogép        &    IGEN  &   =    & IGEN      \\
   Önsúly + hasznos teher & 50+75 kg &  <   & 60+100 kg \\
   Hátraborulhat & NEM &  <   & IGEN \\
\end{tabular}
\end{table}
\end{center}


\section{További fejlesztési irányok}

A robotplatformot a következő lépésekben felkelne szerelni egy nagyobb méréstartományú 3D LIDAR, sztereó kamera, és nagyobb pontóságú GPS vevővel.
A platform  anonimitása lenne a fő cél a növénytermesztésben, legyen képes eljutni A pontból B pontba autonóm módon anélkül hogy kárt tenne a haszonnövényekben. Eközben legyen képes kiszolgálni a majdan rászerelhető pl: robotkar kéréseit.

A mechanika tovább fejlesztése: célszerű lenne a robot mind a négy kerekét kormányozóvá tenni egyedileg, így a önkormányzással járó károk megszőnének és energia fogyasztása is hatékonyabb lenne.

