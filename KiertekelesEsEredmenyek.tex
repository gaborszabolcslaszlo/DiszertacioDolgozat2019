\chapter{Eredmények Kiértékelése}
\section{Megvalósítások}
Ezzel a dolgozattal négy év folyamatos munkájának egy szakaszát szeretném lezárni, a következő eredményeket sikerült felmutatni, kronológiai sorrendben: 
A mechanikai szerkezet fejlesztése, a 2011 be fejlesztett verzióhoz képest, amely sokkal robusztusabb, kültérre alkalmasabb mint az előző, és egyszerűbb is. Ugyan megmaradt az a tendencia hogy csiga-áttételeket alkalmazzak a kerekek meghajtására. Az áttételeket magam terveztem és gyártattam le, a kezdeti alacsony költségvetés miatt az 3D nyomtatóval elkészített inkrementális szenzort nem lehetett alkalmazni robusztusan, ami nem az jeleni, hogy nem is lehetséges csak túl sok időt igényelt volna, a csiga tengely kotyogása miatt.
Az alkatrészeket 3D tervező programban elkészítettem, és 3D nyomtatóval elkészítettük, a tapasztalatom ezekkel az alkatrészekkel: nagy mechanikai terhelés elviselésére nem alkalmasak hosszútávon, ezért történt meg hogy a csiga tengely csapágyháza terhelés alatt széttört.
Vivado környezetet használtam az FPGA fejlesztésére, megvalósítottam egy uBlaze processzorrendszert kialakítását és több hardveres modult is amelyek a következők: PWM modul, UART protokoll csomag értelmező amely támogatja a nagy sebességű kommunikációt, globális engedélyező jel, ezeket a modulokat System Generator-ban valósítottam meg és IP mag készült ezekből.
Sikeresen elsajátítottam a ROS alapjait, és megterveztem egy sajátos kommunikációt FPGA alapú rendszer és a ROS között. Az integráció a robot és a ROS között jól működik, minden egyes szenzor mért adata bekerül a ROS környezetbe.

A hectormap segítségével, az ismeretlen terep térképezésével, a roboton lokalizálva, sikeresen bekonfiguráltam a movebase nevű eszközt, amely segítségével a robotot egy adott pozícióba es irányba tudjuk elvinni. A move base megoldja az akadályok kikerülését is.


\section{Hasonló rendszerekkel való összehasonlítás}
\section{További fejlesztési irányok}
