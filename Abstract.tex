\begin{titlepage}

\section*{Abstract}

The purpose the document is design and implement a mobile robot platform which applicable outdoor, like agriculture, and measure whit robot real scenarios. The robot is a four wheeled robot, the wheels was driven by a worm gearbox 1:70. Some parameters: radius of wheel 0,2m, maximum angular speed of wheel 90 \degree/s. The control angular speed use PID controller, which was implemented in the FPGA based soft core processor(uBlaze). The tuning off PID controller parameter, use the Matlab toolbox for identification system transfer function, after that use Matlab/PidTuning toolbox for calculate optimal PID parameters. During design of robot, to pay attention of possibilities for development other software or hardware based controllers. 

The robot has a miniPC whit Ubuntu Linux based operation system, propose of this the synchronization all of nodes in the robot, collect measured data from sensor and FPGAs and calculate controller signal for position. ROS framework was use for the development of code, based C/C++ language.

The wheels angular speed measure whit incremental rotary encoder, the driven DC motors current was measure whit hall effect based sensor, the measured values processing whit FPGA, and send to miniPC whit UART protokol.

The on of main purpose of this document research the integration methods of FPGA and ROS. The ROS offer capability for integration unique designed hardware used ROS Serial, but has some limitations, because of this, we try this, but has some performance problem when use unique messages, the problem is the size of message buffer limitation. Unable to use for integration FPGA and Ros but successful applied for communication between IMU+ESP8266(arduino based) and ROS use standard ROS messages.

The best practice of the integration is own node for ROS side which handle UART or TCP based unique protocol based communication, and convert messages for the ROS, on hardware side process of messages on FPGA hardware and after that generate an interrupt for processor which handle the request. In this design able to use ROS and FPGA+uBlaze whit 1ms sample time period, and use 200byte size of message.

When we try to use communication without hardware acceleration found limitation of uBlaze processor because of many interrupt request from uart module, and the maximum sample time is 50ms.

Advantage of hardware preprocessing of message is the speed of communication and we able to use some command addressed to direct hardware, with this method implement some safety methods for increase robust of communication. Implement the HeartBeat functionality, which 
responsible for detection failure of important nodes and generate global enable signals for all of the nodes on ROS system. The HeartBeat use live line communication which means the FPGA send periodical message to the PC, the heart beat compare this message whit before received message and decide the hardware is live or not live. When HeartBeat not receive message form hardware since 300ms, this mean the hardware is offline. The FPGA communication module wait for periodical enable signal, when not receive this message disable the controller and PWM module outputs.
The communication module use shared memory whit the uBlaze processor which increase the communication speed.

The implementation of communication and other nodes like PWM module, use Simulink/System Generator, and generate IP core for Vivado designed suit.

The robot has many sensor like IMU, 2D LIDAR, GPS, Battery Voltage sensor, this models was connected to miniPc use USB<->Uart transformer. Use the LIDAR measured data and HectoSlam create a map, in this map the robot able to localize itself. Basically, the LIDAR successful applicable to the mapping task but has problem when the environment contains window, at this case the map have mistaken because the measurements of lidar is noisy.

\subsection{Measurements whit robots}

I make some measurements whit real outdoor robot in various surface conditions like: green grass, marble, pebbly surfaces to study the circle orbit motion and the on place turning, and robot motion on stair. Result of this measurements is the motion on stair is most effective when the center of gravity point is front and the angle between the stair and robot orientation is less than 90\degree. The motion of pebbly surface is most effective when the wheel rotation speed is smaller, because avoided the wheel sinking into sand.

\subsection{Robot Position Controller}

The position controller algorithm divided by two subprocess: turn to goal, and move to goal. Use PI controller to control robot angular position on the map, and PI controller for control robot linear velocity of X axis to reach the desired distance.


Keywords: SSMR,FPGA,ROS,Robot,SLAM

\end{titlepage}

